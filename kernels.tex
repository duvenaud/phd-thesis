%
% A header that lets you compile a chapter by itself, or inside a larger document.
% Adapted from http://stackoverflow.com/questions/3655454/conditional-import-in-latex
%
%
%Use \inbpdocument and \outbpdocument in your individual files, in place of \begin{document} and \end{document}. In your main file, put in a \def \ismaindoc {} before including or importing anything.
%
% David Duvenaud
% June 2011
% 
% ======================================
%
%


\ifx\ismaindoc\undefined
	\newcommand{\inbpdocument}{
		\def \ismaindoc {}
		% Use this header if we are compiling by ourselves.
		\documentclass[a4paper,12pt,authoryear,index]{common/PhDThesisPSnPDF}
		\input{common/official-preamble.tex}
		% All my custom preamble stuff.  Shouldn't overlap with anything in official-preamble

\usepackage{nth}
\usepackage{rotating}
\usepackage{array}
%\usepackage{gantt}
\usepackage[hyperpageref]{backref}

		% ************************ Thesis Information & Meta-data **********************

%% The title of the thesis
%\title{Structured Gaussian Process Models} 
%\title{Automatic Model Construction \\ through \\ Structured Gaussian Processes}
%\title{Automatic Model-Building \\ through \\ Structured Gaussian Processes}
%\title{Automatic Modeling \\ with \\ Structured Gaussian Processes}    
\title{Automatic Model Construction \\ with Gaussian Processes}
%\title{Automatic Model Construction}
%\title{Automating Statistical Model Construction}


%\texorpdfstring is used for PDF metadata. Usage:
%\texorpdfstring{LaTeX_Version}{PDF Version (non-latex)} eg.,
%\texorpdfstring{$sigma$}{sigma}

%% The full name of the author
\author{David Kristjanson Duvenaud}

%% Department (eg. Department of Engineering, Maths, Physics)
\dept{Department of Engineering}

%% University and Crest
\university{University of Cambridge}
\crest{\includegraphics[width=0.25\textwidth]{University_Crest}}

%% You can redefine the submission text:
% Default as per the University guidelines: This dissertation is submitted for
% the degree of Doctor of Philosophy
%\renewcommand{\submissiontext}{change the default text here if needed}

%% Full title of the Degree 
\degree{Doctor of Philosophy}
 
%% College affiliation (optional)
\college{Pembroke College}

%% Submission date
\degreedate{June 2014} 

%% Meta information
\subject{LaTeX} \keywords{{LaTeX} {PhD Thesis} {Engineering} {University of
Cambridge}}



		\begin{document}
	}	
	\newcommand{\outbpdocument}[1]{
		%\bibliographystyle{common/CUEDthesis}
		\bibliographystyle{plainnat}
		\bibliography{references.bib}
		\end{document}
	}	
\else
	%If we're inside another document, no need to re-start the document.
	\ifx\inbpdocument\undefined
		\newcommand{\inbpdocument}{}
		\newcommand{\outbpdocument}[1]{}
	\fi
\fi

\inbpdocument

\chapter{Expressing Structure with Kernels}
\label{ch:kernels}

This chapter shows how to use kernels to build models of functions with many different kinds of structure: additivity, symmetry, periodicity, interactions between variables, and changepoints.
We will also show several ways to encode group invariants into kernels.
Combining a few simple kernels through addition and multiplication will give us a rich, open-ended language of models.

The properties of kernels discussed in this chapter are mostly known in the literature.
The original contribution of this chapter is to gather them into a coherent whole and to offer a tutorial showing the implications of different kernel choices, and some of the structures which can be obtained by combining them.
To the best of our knowledge, the different types of structure that could be obtained through sums and products of kernels had not been systematically explored before \citet{DuvLloGroetal13}.
%TODO: Add more about contribution here.


\section{Definition}

%Since we'll be discussing kernels at length, we now give a precise definition.
A kernel (also called a covariance function, kernel function, or covariance kernel), is a positive-definite function of two inputs $\vx, \vx'$. % in some space $\InputSpace$. 
%Formally, we write $\kernel(\vx, \vx'): \InputSpace \times \InputSpace \to \Reals$.
In this chapter, $\vx$ and $\vx'$ are usually vectors in a Euclidean space, but kernels can also be defined on graphs, images, discrete or categorical inputs, or even text.

Gaussian process models use a kernel to define the prior covariance between any two function values:
%
\begin{align}
\textrm{Cov}\left(f(\vx), f(\vx') \right) = \kernel(\vx,\vx')
\end{align}
%
Colloquially, kernels are often said to specify the similarity between two objects.
This is slightly misleading in this context, since what is actually being specified is the similarity between two values of a \emph{function} evaluated on each object.
The kernel specifies which functions are likely under the \gp{} prior, which in turn determines the generalization properties of the model.





\section{A Few Basic Kernels}
\label{sec:basic-kernels}

To begin understanding the types of structures expressible by \gp{}s, we will start by briefly examining the priors on functions encoded by some commonly used kernels:
the squared-exponential (\kSE), periodic (\kPer), and linear (\kLin) kernels.
These kernels are defined in \cref{fig:basic_kernels}.
%
\begin{figure}[h]%
\centering
\begin{tabular}{r|ccc}
Kernel name: & Squared-exp (\kSE) & Periodic (\kPer) & Linear (\kLin) \\[10pt]
$k(x, x') =$ & $\sigma_f^2 \exp\left(-\frac{(\inputVar - \inputVar')^2}{2\ell^2}\right)$ &
$\sigma_f^2 \exp\left(-\frac{2}{\ell^2} \sin^2 \left( \pi \frac{\inputVar - \inputVar'}{p} \right)\right)$ &
$\sigma_f^2 (\inputVar - c)(\inputVar' - c)$ \\[14pt]
\raisebox{1cm}{Plot of kernel:} & \kernpic{se_kernel} & \kernpic{per_kernel} & \kernpic{lin_kernel}\\
& $x -x'$ & $x -x'$ & \fixedx \\
%& & & \\
 & \large $\downarrow$ & \large $\downarrow$ & \large $\downarrow$  \\
\raisebox{1cm}{\parbox{2.5cm}{Samples from \gp{} prior:}} & \kernpic{se_kernel_draws} & \kernpic{per_kernel_draws_s2} & \kernpic{lin_kernel_draws} \\
Type of structure: & local variation & repeating structure & linear functions
\end{tabular}
\vspace{6pt}
\caption[Examples of structures expressible by some basic kernels]
{Examples of structures expressible by some basic kernels.
%Left and third columns: base kernels $k(\cdot,0)$.
%Second and fourth columns: draws from a \sgp{} with each repective kernel.
%The x-axis has the same range on all plots.
}
\label{fig:basic_kernels}
\end{figure}
%
%There is nothing special about these kernels in particular, except that they represent a diverse set.

Each covariance function corresponds to a different set of assumptions made about the function we wish to model.
For example, using a squared-exp ($\kSE$) kernel implies that the function we are modeling has infinitely many derivatives.
There exist many variants of ``local'' kernels similar to the $\kSE$ kernel, each encoding slightly different assumptions about the smoothness of the function being modeled.

\paragraph{Kernel parameters}
Each kernel has a number of parameters which specify the precise shape of the covariance function.
These are sometimes referred to as \emph{hyper-parameters}, since they can be viewed as specifying a distribution over function parameters, instead of being parameters which specify a function directly.


\paragraph{Stationary and Non-stationary}
The $\kSE$ and $\kPer$ kernels are \emph{stationary}, meaning that their value only depends on the difference $x-x'$.  This implies that the probability of observing a particular dataset remains the same even if we move all the $\vx$ values over by some amount.
In contrast, the linear kernel $\kLin$ is non-stationary, meaning that the corresponding \gp{} model will produce different answers if the data is moved around while the kernel parameters are kept fixed.




\section{Combining Kernels}

What if the kind of structure we need is not expressed by any existing kernel?
For many types of structure, it is possible to build a ``made to order'' kernel with the desired properties.
The next few sections of this chapter will explore ways in which kernels can be combined to create new ones with different properties.
This will allow us to include as much high-level structure as necessary into our models.
%For an overview, see \cite[Chapter~4]{rasmussen38gaussian}.


\subsection{Notation}

Below, we will focus on two ways of combining kernels: addition and multiplication.
We will often write these operations in shorthand, without arguments:
%
\begin{align}
k_a + k_b =& \,\, k_a(\vx, \vx') + k_b( \vx, \vx')\\
k_a \times k_b =& \,\, k_a(\vx, \vx') \times k_b(\vx, \vx')
\end{align}

All of the basic kernels we considered in \cref{sec:basic-kernels} are one-dimensional, but kernels over multidimensional inputs can be constructed by adding and multiplying between kernels on different dimensions.
The dimension on which a kernel operates is denoted by a subscripted integer.
For example, $\SE_2$ represents an \kSE{} kernel over the second dimension of $\vx$.
To remove clutter, we will usually refer to kernels without specifying their parameters.



\subsection{Combining Properties through Multiplication}

%Multiplying kernels allows us to account for interactions between different input dimensions, or to combine different notions of similarity.

%Loosely speaking, multip

Multiplying two positive-definite kernels together always results in another positive-definite kernel.
But what properties do these new kernels have?
%
\begin{figure}
\centering
\begin{tabular}{cccc}
$\kLin \times \kLin$ & $\kSE \times \kPer$ & $\kLin \times \kSE$ & $\kLin \times \kPer$ \\
\kernpic{lin_times_lin} & \kernpic{longse_times_per} & \kernpic{se_times_lin} & \kernpic{lin_times_per}\\
\fixedx & $x -x'$ & \fixedx & \fixedx\\
%& & & \\
\large $\downarrow$ & \large $\downarrow$ & \large $\downarrow$ & \large $\downarrow$  \\
\kernpic{lin_times_lin_draws}  & \kernpic{longse_times_per_draws_s2} & \kernpic{se_times_lin_draws_s2} & \kernpic{lin_times_per_draws_s2} \\
quadratic functions & locally \newline periodic & increasing variation  & growing amplitude \\[10pt]
\end{tabular}
\caption[Examples of structures expressible by multiplying kernels]
{ Examples of one-dimensional structures expressible by multiplying kernels.  
%The x-axis has the same scale for all plots.
Plots have same meaning as in figure \ref{fig:basic_kernels}.}
\label{fig:kernels_times}
\end{figure}
%
\Cref{fig:kernels_times} shows some more interesting kernels that one can obtain by multiplying two basic kernels together.

Working with kernels rather than the parametric form of the function itself allows us to express high-level properties of functions that do not necessarily have a simple parametric form.
Here, we discuss a few examples:

\begin{itemize}
\item {\bf Polynomial Regression.}
By multiplying together $T$ linear kernels, we obtain a prior on polynomials of degree $T$.
The first column of \cref{fig:kernels_times} shows a quadratic kernel.

\item {\bf Locally Periodic Functions.}
In univariate data, multiplying a kernel by \kSE{} gives a way of converting global structure to local structure.
For example, $\Per$ corresponds to exactly periodic structure, whereas $\Per \kerntimes \SE$ corresponds to locally periodic structure, as shown in the second column of \cref{fig:kernels_times}.

\item {\bf Functions with Growing Amplitude.}
Multiplying by a linear kernel means that the marginal standard deviation of the function being modeled will grow linearly away from the origin.
The third and fourth columns of \cref{fig:kernels_times} show two examples.
\end{itemize}

We can multiply any number of kernels together in this way to produce kernels combining several high-level properties.
For example, the kernel $\kSE \kerntimes \kLin \kerntimes \kPer$ is a prior on functions which are locally periodic with linearly growing amplitude.
We will see examples of real datasets with this kind of structure in \cref{ch:grammar}.


\subsection{Building Flexible Multidimensional Models}

%For instance, in multidimensional data, the multiplicative kernel $\SE_1 \kerntimes \SE_2$ represents a smoothly varying function of dimensions 1 and 2 which is not constrained to be additive.

We can build flexible models of functions of more than one input simply by multiplying kernels between the different inputs.
For example, a product of $\kSE$ kernels over different dimensions, each having a different lengthscale parameter, is called the $\seard$ kernel:
%
\begin{align}
\seard( \vx, \vx')
 = \prod_{d=1}^D \exp \sigma_d^2 \left( -\frac{1}{2} \frac{\left( x_d - x'_d \right)^2}{\ell_d^2} \right)
 = \sigma_f^2 \exp \left( -\frac{1}{2} \sum_{d=1}^D \frac{\left( x_d - x'_d \right)^2}{\ell_d^2} \right)
\end{align}
%
\Cref{fig:product-of-se-kernels} illustrates the \seard{} kernel in two dimensions.
%
\begin{figure}[ht!]
\centering
\begin{tabular}{ccccccc}
%$\kSE_1$ & \hspace{-0.4cm} $\kerntimes$ & $\kSE_2$ &  $=$ & \hspace{-0.4cm} $\kSE_1 \kerntimes \kSE_2$ & & \hspace{-0.5cm} $f \sim \GPt{\vzero}{\kSE_1 \kerntimes \kSE_2}$ \\
\includegraphics[width=0.2\textwidth]{\additivefigsdir/2d-kernel/additive_kernel_sum_p1} \hspace{-0.3cm}
& \hspace{-0.3cm} \raisebox{1cm}{$\times$} & 
\hspace{-0.3cm}\includegraphics[width=0.2\textwidth]{\additivefigsdir/2d-kernel/additive_kernel_sum_p2} \hspace{-0.3cm}
& \hspace{-0.2cm} \raisebox{1cm}{=} \hspace{-0.2cm} & \hspace{-0.3cm}
\includegraphics[width=0.2\textwidth]{\additivefigsdir/2d-kernel/sqexp_kernel} 
& \hspace{-0.3cm} \raisebox{1cm}{$\rightarrow$} \hspace{-0.3cm} & \hspace{-0.3cm}
\includegraphics[width=0.2\textwidth]{\additivefigsdir/2d-kernel/sqexp_draw}\\
%$k_1(x_1, x_1')$ & & $k_2(x_2, x_2')$ & & \hspace{-0.4cm}\parbox{4cm}{$k_1(x_1,x_1') \kerntimes k_2(x_2,x_2')$} & & $f(x_1, x_2)$ \\[1em]
$\kSE_1(x_1, x_1')$ &  & $\kSE_2(x_2, x_2')$ &  & \hspace{-0.4cm} $\kSE_1 \kerntimes \kSE_2$ & & \hspace{-0.5cm} \parbox{0.24\columnwidth}{$f(x_1, x_2)$ drawn from $\GPt{0}{\kSE_1 \kerntimes \kSE_2}$} \\
\end{tabular}
\caption[A product of squared-exponential kernels across different dimensions]{
%\emph{Top left:} An additive kernel is a sum of kernels.
%\emph{Bottom left:}  A draw from an additive kernel corresponds to a sum of draws from independent \gp{} priors with the corresponding kernels.
%\emph{Top right:} A product kernel.
%\emph{Right:}  A \gp{} prior with a product of kernels does not correspond to a product of draws from \gp{}s.
%In this example, both kernels are composed of one dimensional squared-exponential kernels, but this need not be the case in general.
A product of two one-dimensional kernels gives rise to a prior on functions which depend on both dimensions.
}
\label{fig:product-of-se-kernels}
\end{figure}

\ARD{} stands for Automatic Relevance Determination, so named because estimating the lengthscale parameters $\ell_1, \ell_2, \dots, \ell_D$, implicitly determines the ``relevance'' of each dimension.
Input dimensions with relatively large lengthscales indicate relatively little variation along those dimensions in the function being modeled.

$\seard$ kernels are the default kernel in most applications of \gp{}s.
This may be partly because they have relatively few parameters to estimate, and those parameters are relatively interpretable.
In addition, there is a theoretical reason to use them: they are \emph{universal} kernels~\citep{micchelli2006universal}, capable of learning any continuous function given enough data, under some conditions.

However, this flexibility means that they can sometimes be relatively slow to learn, due to the \emph{curse of dimensionality} \citep{bellman1956dynamic}.
In general, the more structure we account for, the less data we need - the \emph{blessing of abstraction} \citep{goodman2011learning} counters the curse of dimensionality.
Below, we will investigate ways to encode more structure into kernels.



\section{Modeling Sums of Functions}

An additive function is one which can be expressed as $f(\vx) = f_a(\vx) + f_b(\vx)$.
%
%TODO: make the se_long + se_short figure a bit more obvious
\begin{figure}
\centering
\begin{tabular}{cccc}
%Composite & Draws from \gp{} & \gp{} posterior \\ \toprule
$\kLin + \kPer$ & $\kSE + \kPer$ & $\kSE + \kLin$ & $\kSE^{(\textnormal{long})} + \kSE^{(\textnormal{short})}$ \\
\kernpic{lin_plus_per} & \kernpic{se_plus_per} & \kernpic{se_plus_lin} & \kernpic{longse_plus_se}\\
\fixedx & $x -x'$ & \fixedx & $x -x'$\\
%& & & \\
\large $\downarrow$ & \large $\downarrow$ & \large $\downarrow$ & \large $\downarrow$  \\
\kernpic{lin_plus_per_draws} & \kernpic{se_plus_per_draws_s7} & \kernpic{se_plus_lin_draws_s5} & \kernpic{longse_plus_se_draws_s7}\\
periodic plus trend & periodic plus noise & linear plus variation & slow \& fast variation \\[10pt]
\end{tabular}
\caption[Examples of structures expressible by adding kernels]
{ Examples of one-dimensional structures expressible by adding kernels.  
%The x-axis has the same scale for all plots.
Rows have the same meaning as in \cref{fig:basic_kernels}.
$\kSE^{(\textnormal{long})}$ denotes a $\kSE$ kernel whose lengthscale is long relative to that of $\kSE^{(\textnormal{short})}$
}
\label{fig:kernels_plus}
\end{figure}
%
Additivity is a very useful modeling assumption in a wide variety of contexts, especially if it allows us to make strong assumptions about the individual components which make up the sum.
Restricting the flexibility of the component functions often aids in building interpretable models, and sometimes enables extrapolation in high dimensions.

Fortunately, it's easy to encode additivity into \gp{} models.
Suppose functions ${\function_a, \function_b}$ are drawn independently from \gp{} priors:
%
\begin{align}
\function_a \dist \GP(\mu_a, \kernel_a)\\
\function_b \dist \GP(\mu_b, \kernel_b)
\end{align}
%
Then the sum of those functions is simply another \gp{}:
%
\begin{align}
%\function := 
\function_a + \function_b \dist \GP(\mu_a + \mu_b, \kernel_a + \kernel_b).
\end{align}
%
Kernels $\kernel_a$ and $\kernel_b$ can be kernels of different types, allowing us to model the data as a sum of independent functions, each possibly representing a different type of structure.
We can also sum any number of components this way.


\subsection{Additivity Across Multiple Dimensions}
\label{sec:additivity-multiple-dimensions}

When modeling functions of multiple dimensions, summing kernels can give rise to additive structure across different dimensions.
To be more precise, if the kernels being added together are functions of only a subset of input dimensions, then the implied prior over functions decomposes in the same way.
%
\begin{figure}
\centering
\begin{tabular}{ccccc}
\hspace{-0.2cm}\includegraphics[width=0.2\textwidth]{\additivefigsdir/2d-kernel/additive_kernel_sum_p2} 
& \hspace{-0.4cm} \raisebox{1cm}{+} \hspace{-0.4cm} & 
\includegraphics[width=0.2\textwidth]{\additivefigsdir/2d-kernel/additive_kernel_sum_p1} 
& \hspace{-0.4cm} \raisebox{1cm}{=} \hspace{-0.4cm} & 
\includegraphics[width=0.2\textwidth]{\additivefigsdir/2d-kernel/additive_kernel} \\
$k_1(x_1, x_1')$ & & $k_2(x_2, x_2')$ & & $k_1(x_1,x_1') + k_2(x_2,x_2')$ \\[1em]
\large $\downarrow$ & & \large $\downarrow$ & & \large $\downarrow$ \\[-0.2em]
\hspace{-0.2cm}\includegraphics[width=0.2\textwidth]{\additivefigsdir/2d-kernel/additive_kernel_draw_sum_p1}
& \hspace{-0.4cm} \raisebox{1cm}{+} \hspace{-0.4cm} & 
\includegraphics[width=0.2\textwidth]{\additivefigsdir/2d-kernel/additive_kernel_draw_sum_p2}
& \hspace{-0.4cm} \raisebox{1cm}{=} \hspace{-0.4cm} &
\includegraphics[width=0.2\textwidth]{\additivefigsdir/2d-kernel/additive_kernel_draw_sum} \\
$f_1(x_1) \sim \GP\left(0, k_1\right)$ & & $f_2(x_2) \sim \GP\left(0, k_2\right)$ & & $f_1(x_1) + f_2(x_2)$ \\[1em]
\end{tabular}
\caption[Additive kernels correspond to additive functions]{
A sum of two orthogonal one-dimensional kernels.
\emph{Top row:} An additive kernel is any sum of kernels.
\emph{Bottom row:} A draw from an additive kernel corresponds to a sum of draws from independent \gp{} priors, each having the corresponding kernel.
}
\label{fig:sum-of-kernels}
\end{figure}
%
For example, if
%
\begin{align}
%\function := 
f(x_1, x_2) \dist \GP(0, \kernel_1(x_1, x_1') + \kernel_2(x_2, x_2'))
\end{align}
%
Then this is equivalent to the model
%
\begin{align}
\function_1(x_1) & \dist \GP(0, \kernel_1(x_1, x_1'))\\
\function_2(x_2) & \dist \GP(0, \kernel_2(x_2, x_2'))\\
f(x_1, x_2) & = f_1(x_1) + f_2(x_2)
\end{align}
%
%Then we can model the sum of those functions through another \gp{}:
%

\Cref{fig:sum-of-kernels} illustrates a decomposition of this form.
Note that the product of two kernels does not have an analogous interpretation as the product of two functions.


\subsection{Long-range Extrapolation through Additivity}
\label{sec:additivity-extrapolation}

Additive structure often allows us to make predictions far from the training data.
%
\begin{figure}
\centering
\begin{tabular}{ccc}
 & \gp{} mean using & \gp{} mean using \\
True function: & sum of $\kSE$ kernels: & product of $\kSE$ kernels: \\
\parbox{0.33\columnwidth}{$f(x_1, x_2) = \sin(x_1) + \sin(x_2)$} & $k_1(x_1, x_1') \kernplus k_2(x_2, x_2')$ &  $k_1(x_1, x_1') \kerntimes k_2(x_2, x_2')$ \\[0.2em]
\hspace{-0.1in}\includegraphics[width=0.32\textwidth]{\additivefigsdir/1st_order_censored_truth} &
\hspace{-0.1in}\includegraphics[width=0.32\textwidth]{\additivefigsdir/1st_order_censored_add} & 
\hspace{-0.1in}\includegraphics[width=0.32\textwidth]{\additivefigsdir/1st_order_censored_ard}\\[1em]
\end{tabular}
\caption[Long-range inference in functions with additive structure]
{\emph{Left:} A function with additive structure.
%Inference in functions with additive structure.
\emph{Center:} When the function being modeled has additive structure, a \gp{} with an additive kernel can exploit this fact to extrapolate far from the training data.
\emph{Right:} The product kernel allows a different function value for every combination of inputs, and so is uncertain about function values away from the training data.  This causes the predictions to revert to the mean.
%  The additive GP is able to discover the additive pattern, and use it to fill in a distant mode.  The ARD kernel can only interpolate, and thus predicts the mean in locations missing data.
}
\label{fig:synth2d}
\end{figure}
%
\Cref{fig:synth2d} compares the extrapolations made by additive versus product-kernel \gp{} models, conditioned on data from a sum of two axis-aligned sine functions.
The training data is confined to a small, L-shaped area.
In this example, the additive model is able to correctly predict the height of the function at unseen combinations of inputs.
The product-kernel model is more flexible, and so remains uncertain about the function away from the data.
%The ability of additive GPs to discover long-range structure suggests that this model may be well-suited to deal with covariate-shift problems.

These types of additive models have been well-explored in the statistics literature.
For example, generalized additive models~\citep{hastie1990generalized} have seen wide adoption.
In high dimensions, we can also consider sums of functions of more than one dimension.
\Cref{ch:additive} considers this model class in more detail.




\subsection{Example: An Additive Model of Concrete Strength}
\label{sec:concrete}

To illustrate how additive kernels give rise to interpretable models, we will build an additive model of the strength of concrete as a function of the amount of seven different ingredients, plus the age of the concrete \citep{yeh1998modeling}.
%We model measurements of the compressive strength of concrete, as a function of the concentration of 7 ingredients, plus the age of the concrete.

Our simple additive model looks like
%
\begin{align}
f(\vx) & = 
f_1(\textnormal{cement}) + f_2(\textnormal{slag}) + f_3(\textnormal{fly ash}) + f_4(\textnormal{water}) \nonumber \\
& \quad + f_5(\textnormal{plasticizer}) + f_6(\textnormal{coarse}) + f_7(\textnormal{fine}) + f_8(\textnormal{age}) + \textnormal{noise}
\label{eq:concrete}
\end{align}
%
where $\textnormal{noise} \simiid \Nt{0}{\sigma^2_n}$.
After learning the kernel parameters by maximizing the marginal likelihood of the data, we can visualize the predictive distribution of each component of the model.
%
%
\newcommand{\concretepic}[1]{\includegraphics[width=0.31\columnwidth]{\decompfigsdir/concrete-component-#1}}
\newcommand{\concretelegend}[0]{\raisebox{5mm}{\includegraphics[trim=56mm 1mm 2mm 34.1mm, clip, width=0.24\columnwidth]{\decompfigsdir/concrete-component-8-legend}}}
%
\begin{figure}[h!]
\centering
\begin{tabular}{ccc}
%\includegraphics[width=0.3\textwidth]{\additivefigsdir/interpretable_1st_order1.pdf} &
%\includegraphics[width=0.3\textwidth]{\additivefigsdir/interpretable_1st_order2.pdf}& 
Cement & Slag & Fly Ash\\
\concretepic{1} & \concretepic{2} & \concretepic{3} \\
 Water & Plasticizer & Coarse\\
\concretepic{4} & \concretepic{5} & \concretepic{6} \\
 Fine & Age \\
 \concretepic{7} & \concretepic{8} & \concretelegend \\
\end{tabular}
\caption[Decomposition of posterior into interpretable one-dimensional functions]
{The predictive distribution of each component of a multidimensional additive model.
Blue crosses indicate the original data projected on to each dimension, red indicates the marginal posterior density, and colored lines are samples from the marginal posterior of each component.
The vertical axis is the same for all plots.
%The black line is the posterior mean of a \sgp{} with only one term in its kernel.
%Right:  the posterior mean of a \sgp{} with only one second-order term in its kernel.
%rs = 1 - var(complete_mean - y)/ var(y)
%R-squared = 0.9094
}
\label{fig:interpretable functions}
\end{figure}
%

\Cref{fig:interpretable functions} shows the marginal posterior distribution of each of the eight components in \cref{eq:concrete}.
We can see that the parameters controlling the variance of two of the components, Coarse and Fine, were set to zero, meaning that the marginal likelihood preferred a parsimonious model which did not depend on these dimensions.
This is an example of the automatic sparsity that arises by maximizing marginal likelihood in \gp{} models, and another example of automatic relevance determination (\ARD) \citep{neal1995bayesian}.

The ability to learn kernel parameters in this way is much more difficult when using non-probabilistic methods such as Support Vector Machines \citep{cortes1995support}, for which cross-validation is often the best method to select kernel parameters.



\subsection{Posterior Variance of Additive Components}
\label{sec:posterior-variance}


%\footnote{Code is available at \url{github.com/duvenaud/phd-thesis/}}
Here we derive the posterior variance and covariance of all of the additive components of a \gp{}.
These formulas allow us to make plots such as \cref{fig:interpretable functions}.
%These formulas let us plot the marginal variance of each component separately.  These formulas can also be used to examine the posterior covariance between pairs of components.

First, we will write down the joint prior over the sum of two functions drawn from \gp{} priors.
We distinguish between $\vecf(\vX)$ (the function values at the training locations) and  $\vecf(\vX^\star)$ (the function values at some set of query locations).
% so that it's clear which matrices to use to extrapolate.

If $f_1$ and $f_2$ are \emph{a priori} independent, and $f_1 \sim \gp( \mu_1, k_1)$ and $f_2 \sim \gp( \mu_2, k_2)$, then
%
\begin{align}
\left[ \begin{array}{l} 
\vf_1(\vX) \\
\vf_1(\vX^\star) \\
\vf_2(\vX) \\
\vf_2(\vX^\star) \\
\vf_1(\vX) + \vf_2(\vX) \\
\vf_1(\vX^\star) + \vf_2(\vX^\star)
\end{array} \right]
\sim
\Nt{
\left[ \begin{array}{c} \vmu_1 \\ \vmu_1^\star \\ \vmu_2 \\ \vmu_2^\star \\ \vmu_1 + \vmu_2 \\ \vmu_1^\star + \vmu_2^\star \end{array} \right]
}
{\left[ \begin{array}{cccccc} 
\vK_1 & \vK_1^\star & 0 & 0 & \vK_1 & \vK_1^\star \\ 
\vK_1^\star & \vK_1^{\star\star} & 0 & 0 & \vK_1^\star & \vK_1^{\star\star} \\
0 & 0 & \vK_2 & \vK_2^\star & \vK_2 & \vK_2^\star \\ 
0 & 0 & \vK_2^\star & \vK_2^{\star\star} & \vK_2^\star & \vK_2^{\star\star} \\
\vK_1 & \vK_1^\star & \vK_2 & \vK_2^\star & \vK_1 + \vK_2 & \vK_1^\star + \vK_2^\star \\ 
\vK_1^\star & \vK_1^{\star\star}  & \vK_2^\star & \vK_2^{\star\star}  & \vK_1^\star + \vK_2^\star & \vK_1^{\star\star} + \vK_2^{\star\star}\\
\end{array} \right]
}
\end{align}
%
where we represent the Gram matrices, evaluated at all pairs of vectors in bold capitals as ${\vK_{i,j} = k(\vx_i, \vx_j)}$.  So 
%
\begin{align}
\vK_1 & = k_1( \vX, \vX ) \\
\vK_1^\star & = k_1( \vX^\star, \vX ) \\
\vK_1^{\star\star} & = k_1( \vX^\star, \vX^\star )
\end{align}

By the formula for Gaussian conditionals, (given by \cref{eq:gauss_conditional}), we get that the conditional distribution of a \gp{}-distributed function conditioned on its sum with another \gp{}-distributed function is given by
%
\begin{align}
\vf_1(\vX^\star) | \vf_1(\vX) + \vf_2(\vX) \sim \Normal \Big( 
& \vmu_1^\star + \vK_1^{\star\tra} (\vK_1 + \vK_2)\inv \left[ \vf_1(\vX) + \vf_2(\vX) - \vmu_1 - \vmu_2 \right] \nonumber \\
& \vK_1^{\star\star} - \vK_1^{\star\tra} (\vK_1 + \vK_2)\inv \vK_1^{\star} \Big)
%\vf_1(\vx^\star) | \vf(\vx) \sim \mathcal{N}\big( 
%& \vmu_1(\vx^\star) + \vk_1(\vx^{\star}, \vx)  \left[ \vK_1(\vx, \vx) + \vK_2(\vx, \vx) \right]\inv \left( \vf(\vx) - \vmu_1(\vx) - \vmu_2(\vx) \right) , \nonumber \\
%& \vk_1(\vx^{\star}, \vx^{\star}) - \vk_1(\vx^{\star}, \vx)  \left[ \vK_1(\vx, \vx) + \vK_2(\vx, \vx) \right] \inv \vk_1(\vx, \vx^{\star}) 
%\big).
\end{align}
%
These formulas express the model's posterior uncertainty about the different components of the signal, integrating over the possible configurations of the other components.
To extend these formulas to a sum of more than two functions, the term $\vK_1 + \vK_2$ can simply be replaced by $\sum_i \vK_i$ everywhere.




\subsubsection{Posterior Covariance of Additive Components}

%One of the advantages of using a generative, model-based approach is that we can examine any aspect of the model we wish to.
We can also compute the posterior covariance between any two components, conditioned on their sum:
if this quantity is negative, it means that there is ambiguity about which of the two components explains the data at that location.
\begin{figure}
\centering
\renewcommand{\tabcolsep}{1mm}
\def\incpic#1{\includegraphics[width=0.1\columnwidth]{../figures/decomp/concrete-#1}}
\begin{tabular}{p{2mm}*{6}{c}}
 & {Cement} & {Slag} & {Fly Ash} & {Water} & \parbox{0.1\columnwidth}{Plasticizer} & {Age} \\ 
 \rotatebox{90}{{Cement}} & \incpic{Cement-Cement} & \incpic{Cement-Slag} & \incpic{Cement-Fly-Ash} & \incpic{Cement-Water} & \incpic{Cement-Plasticizer} & \incpic{Cement-Age} \\ 
 \rotatebox{90}{{Slag}} & \incpic{Slag-Cement} & \incpic{Slag-Slag} & \incpic{Slag-Fly-Ash} & \incpic{Slag-Water} & \incpic{Slag-Plasticizer} & \incpic{Slag-Age} \\ 
 \rotatebox{90}{{Fly Ash}} & \incpic{Fly-Ash-Cement} & \incpic{Fly-Ash-Slag} & \incpic{Fly-Ash-Fly-Ash} & \incpic{Fly-Ash-Water} & \incpic{Fly-Ash-Plasticizer} & \incpic{Fly-Ash-Age} \\ 
 \rotatebox{90}{{$\quad$Water}} & \incpic{Water-Cement} & \incpic{Water-Slag} & \incpic{Water-Fly-Ash} & \incpic{Water-Water} & \incpic{Water-Plasticizer} & \incpic{Water-Age} \\ 
 \rotatebox{90}{\parbox{0.1\columnwidth}{Plasticizer}} & \incpic{Plasticizer-Cement} & \incpic{Plasticizer-Slag} & \incpic{Plasticizer-Fly-Ash} & \incpic{Plasticizer-Water} & \incpic{Plasticizer-Plasticizer} & \incpic{Plasticizer-Age} \\ 
 \rotatebox{90}{{Age}} & \incpic{Age-Cement} & \incpic{Age-Slag} & \incpic{Age-Fly-Ash} & \incpic{Age-Water} & \incpic{Age-Plasticizer} & \incpic{Age-Age} \\
 \end{tabular}
 \fbox{
\begin{tabular}{c}
 Correlation \\[1ex]
\includegraphics[width=0.1\columnwidth, clip, trim=6.2cm 0cm 0cm 0cm]{../figures/decomp/concrete-colorbar}
\end{tabular}
}
\caption[Visualizing posterior correlations between components]
{Posterior correlations between the components explaining the concrete dataset.
Each plot shows the additive model's posterior correlations between two components, plotted over the domain of the data $\pm 5\%$.
%Color indicates the amount of correlation between the function value of the two components.
Red indicates high correlation, teal indicates no correlation, and blue indicates negative correlation.
Plots on the diagonal show posterior correlations within each component.
%Off-diagonal plots show posterior covariance between each pair of functions, as a function of both inputs.
%Negative correlation means that one function is high and the other low, but which one is uncertain.
Dimensions `Coarse' and `Fine' are not shown, because their variance is zero.
}
\label{fig:interpretable interactions}
\end{figure}
%
%
\begin{align}
\covargs{ \vf_1(\vX^\star)}{\vf_2(\vX^\star) | \vf(\vX) } 
& = - \vK_1^{\star\tra} (\vK_1 + \vK_2)\inv \vK_2^\star
%\covargs{\vf_1(\vx^\star)}{\vf_2(\vx^\star) | \vf(\vx) }
%& = - \vk_1(\vx^{\star}, \vx)  \left[ \vK_1(\vx, \vx) + \vK_2(\vx, \vx) \right] \inv \vk_1(\vx, \vx^{\star}) 
\label{eq:post-component-cov}
\end{align}
%
\Cref{fig:interpretable interactions} shows the posterior correlation between all non-zero components of the concrete model.
Most of the correlation occurs within components, but there is also negative correlation between the ``Cement'' and ``Slag'' variables.
%We can see that there is negative correlation between the ``Cement'' and ``Slag'' variables.
This reflects an ambiguity in the model about which one of these functions is high and the other low.



%\subsection{Interpreting Additive Models in High Dimensions}
%As noted by \citet{plate1999accuracy}, o
%One advantage of additive models is their interpretability.
% since a high-dimensional function is decomposed into a series of one- or two-dimensional functions.  
%\citet{plate1999accuracy} demonstrated that by allowing high-order interactions as well as low-order interactions, one can trade off interpretability with predictive accuracy.
%In the case where the kernel parameters indicate that most of the variance in a function can be explained by low-order interactions, it is informative to plot the corresponding low-order functions, as in Figure \ref{fig:interpretable functions}. 








\section{Changepoints}
A simple example of how combining kernels can give rise to more structured priors is given by changepoint kernels.
Changepoints kernels can be defined through addition and multiplication with sigmoidal functions such as $\sigma(x) = \nicefrac{1}{1 + \exp(-x)}$:
%
\begin{align}
%\kCP(\kernel_1, \kernel_2) = \kernel_1 \kerntimes \boldsymbol\sigma \kernplus \kernel_2 \kerntimes \boldsymbol{\bar\sigma}
%\kCP(\kernel_1, \kernel_2) & = \begin{array}{rl} (1-\sigma(x)) & k_2(x,x')(1-\sigma(x')) \\
%  + \sigma(x) & k_1(x,x')\sigma(x') \end{array} \\
\kCP(\kernel_1, \kernel_2)(x,x') & = \sigma(x) k_1(x,x')\sigma(x') + (1-\sigma(x)) k_2(x,x')(1-\sigma(x'))
\label{eq:cp}
\end{align}
%
which can be written in shorthand as
%
\begin{align}
\kCP(\kernel_1, \kernel_2)& = \kernel_1 \kerntimes \boldsymbol\sigma \kernplus \kernel_2 \kerntimes \boldsymbol{\bar\sigma}
\end{align}
where $\boldsymbol\sigma = \sigma(x)\sigma(x')$ and $\boldsymbol{\bar\sigma} = (1-\sigma(x))(1-\sigma(x'))$.

This compound kernel expresses a change from one kernel to another.
The parameters of the sigmoid determine where, and how rapidly, this change occurs.

\newcommand{\cppic}[1]{\includegraphics[height=2.2cm,width=3.3cm]{\grammarfiguresdir/changepoint/#1}}%
\begin{figure}[h]
\centering
\begin{tabular}{rcccc}
 & $\kCP(\kSE, \kPer)$ & $\kCP(\kSE, \kPer)$ & $\kCP(\kSE, \kSE)$ & $\kCP(\kPer, \kPer)$ \\
\raisebox{1cm}{f(x)} \hspace{-0.4cm} & \cppic{draw_1} & \cppic{draw_2} & \cppic{draw_3} & \cppic{draw_4} \\
& $x$ & $x$ & $x$ & $x$
\end{tabular}
\caption[Draws from changepoint priors]
{Draws from different priors on using changepoint kernels, constructed by adding and multiplying together base kernels with sigmoidal functions.
}
\label{fig:changepoint_examples}
\end{figure}

We can also express a function whose structure changes within some interval -- a \emph{change window} -- by replacing $\sigma(x)$ with a product of two sigmoids, one increasing and one decreasing.

\subsection{Multiplication by a Known Function}

More generally, we can model an unknown function that's been multiplied by some fixed, known function $a(x)$, by multiplying the kernel by $a(\vx) a(\vx')$.
Formally,
%
\begin{align}
f(\iva) = a(\iva) g(\iva), \quad g \sim \GPdist{g}{\vzero}{k(\iva, \iva')} \quad
\iff
\quad f \sim \GPdist{f}{\vzero}{ a(\iva) k(\iva, \iva') a(\iva')}
\end{align}




\section{Feature Representation}
%
By Mercer's theorem \citep{mercer1909functions},
%Due to \citet{mercer1909functions}, we know that
any positive-definite kernel can be represented as the inner product between a fixed set of features, evaluated at $\vx$ and at $\vx'$:
%
\begin{align}
k(\vx, \vx') = \feat(\vx)\tra \feat(\vx')
\end{align}

As a simple example, the squared-exponential kernel ($\kSE$) on the real line has a representation in terms of infinitely many radial-basis functions of the form ${h_i(x) \propto \exp( -\frac{1}{2} \frac{(x - c_i)^2}{2\ell^2})}$.
More generally, any stationary kernel
% (one which only depends on the distance between its inputs) 
on the real line can be represented by a set of sines and cosines - a Fourier representation \citep{bochner1959lectures}.
In general, any particular feature representation of a kernel is not unique, and depends on which space $\InputSpace$ is being considered \citep{minh2006mercer}.

In some cases, $\InputSpace$ can even be the infinite-dimensional feature mapping of another kernel.  Composing feature maps in this way leads to \emph{deep kernels}, a topic explored in \cref{ch:deep-limits}.



\subsection{Relation to Linear Regression}

Surprisingly, \gp{} regression is equivalent to Bayesian linear regression on the implicit features $\feat(\vx)$ which give rise to the kernel:
%
\begin{align}
f(\iva) = \vw\tra \feat(\iva), \quad \vw \sim \N{\vw}{\vzero}{\vI} \quad
\iff
\quad f \sim \GPdist{f}{\vzero}{\feat(\iva) \tra \feat(\iva)}
\end{align}
%
The link between Gaussian processes, linear regression, and neural networks is explored further in \cref{ch:deep-limits}.


\subsection{Feature-space View of Combining Kernels}

\def\feata{\va}
\def\featb{\vb}

%Many architectures for learning complex functions, such as convolutional networks \cite{lecun1989backpropagation} and sum-product networks \cite{poon2011sum}, include units which compute \texttt{and}-like and \texttt{or}-like operations.
%Composite kernels can be viewed in this way too.
%A sum of kernels can be understood as an \texttt{or}-like operation: two points are considered similar if either kernel has a high value.
%Similarly, multiplying kernels is an \texttt{and}-like operation, since two points are considered similar only if both kernels have high values.


We can also view kernel addition and multiplication as a combination of the features of the original kernels.
%Viewing kernel addition from this point of view, if
For example, if we have two kernels
%
\begin{align}
k_a(\iva, \iva') & = \feata(\iva)\tra \feata(\iva')\\
k_b(\iva, \iva') & = \featb(\iva)\tra \featb(\iva')
\end{align}
%
then their addition has the form:
%
\begin{align}
k_a(\iva, \iva') + k_b(\iva, \iva')
& = \feata(\iva)\tra \featb(\iva') + \feata(\iva)\tra \featb(\iva') 
 = \colvec{\feata(\iva)}{\featb(\iva)}\tra \colvec{\feata(\iva')}{\featb(\iva')}
\end{align}
%
meaning that the features of $k_a + k_b$ are the concatenation of the features of each kernel.

We can examine kernel multiplication in a similar way:
%
\begin{align}
k_a(\iva, \iva') \times k_b(\iva, \iva')
& = \left[ \feata(\iva)\tra \feata(\iva') \right] \times \left[ \featb(\iva)\tra \featb(\iva') \right] \\
%& = \left[ \begin{array}{c} \feat_1(\iva) \\ \feat_2(\iva) \end{array} \right]^\tra \left[ \begin{array}{c} \feat_1(\iva') \\ \feat_2(\iva') \end{array} \right]
& = \sum_i a_i(\iva) a_i(\iva') \times \sum_j b_j(\iva) b_j(\iva') \\
%& = \sum_i \sum_j a_i(\iva) a_i(\iva') b_j(\iva) b_j(\iva') \\
& = \sum_{i,j} \big[ a_i(\iva) b_j(\iva) \big] \big[ a_i(\iva') b_j(\iva') \big] \\
& = \vecop{ \feata(\iva) \otimes \featb(\iva') } \tra \vecop{ \feata(\iva) \otimes \featb(\iva')} 
\end{align}
%
In other words, the features of $k_a \times k_b$ are just the Cartesian product (all possible combinations) of the original two sets of features.
For example, the Cartesian product of the features of two one-dimensional $\kSE$ kernels covers the plane with two-dimensional radial-basis functions of the form
%
\begin{align}
h_{ij}(x_1, x_2) \propto \exp \left( -\frac{1}{2} \frac{(x_1 - c_i)^2}{2\ell_1^2} \right) \exp \left( -\frac{1}{2} \frac{(x_2 - c_j)^2}{2\ell_2^2} \right)
\end{align}





\section{Expressing Symmetries and Invariants}
\label{sec:expressing-symmetries}

\def\gswitch{G_\textnormal{swap}}

When modeling functions, encoding known symmetries can improve predictive accuracy. 
This section looks at different ways to encode symmetries into a prior on functions.
Many types of symmetry can be enforced through operations on the kernel.

We will demonstrate the properties of the resulting models by sampling functions from their priors.
By using these functions to define warpings from $\Reals^2 \to \Reals^3$, we will show how to build a nonparametric prior on an open-ended family of topological manifolds, such as cylinders, toruses, and M\"{o}bius strips.

\subsection{Three Recipes for Invariant Priors}

Consider the scenario where we have a finite set of transformations of the input space $\{g_1, g_2, \ldots \}$ to which we wish our function to remain invariant:
%
\begin{align}
f(\vx) = f(g(\vx))  \quad \forall \vx \in \mathcal{X}, \quad \forall g \in G
\end{align}
%
As an example, imagine we want to build a model of functions invariant to swapping their inputs: $f(x_1, x_2) = f(x_2, x_1)$.
Being invariant to a set of operations is equivalent to being invariant to all compositions of those operations, the set of which form a group. \citep[chapter 21]{armstrong1988groups}.
In our example, the elements of the group $\gswitch$ containing the operations the functions are invariant to has two elements:%, and is an example of the symmetric group $S2$:
%
\begin{align}
g_1([x_1, x_2]) & = [x_2, x_1] \qquad \textnormal{(swap)} \\
g_2([x_1, x_2]) & = [x_1, x_2] \qquad \textnormal{(identity)}
\end{align}
%
How can we construct a prior on functions which respect these symmetries?

\citet{ginsbourger2012argumentwise} and \citet{Invariances13} showed that the only way to construct a \gp{} prior on functions which respect a set of invariants is to construct a kernel which respects the same invariants with respect to each of its two inputs:
%
\begin{align}
k(\vx, \vx') = k( g(\vx), g(\vx')), \quad \forall \vx, \vx' \in \InputSpace, \quad \forall g, g' \in G
\end{align}
%
Formally, given a finite group $G$ whose elements are operations to which we wish our function to remain invariant, and $f \sim \GPt{0}{k(\vx,\vx')}$, then every $f$ is invariant under $G$ (up to a modification) if and only if $k(\cdot, \cdot)$ is argument-wise invariant under $G$.

It might not always be clear how to construct a kernel respecting such argument-wise invariances.
%Fortunately, for finite groups, there are a few simple ways to transform any kernel into one which is argument-wise invariant to actions under any finite group:
Fortunately, there are a few simple ways to do this for any finite group:
%
%
\begin{figure}
\renewcommand{\tabcolsep}{1.5mm}
\begin{tabular}{ccc}
Additive method & Projection method & Product method \\[0.5ex]
\includegraphics[width=0.3\columnwidth]{\symmetryfigsdir/symmetric-xy-naive-sample} &
\includegraphics[width=0.3\columnwidth]{\symmetryfigsdir/symmetric-xy-projection-sample} &
\includegraphics[width=0.3\columnwidth]{\symmetryfigsdir/symmetric-xy-prod-sample}\\
%$k(x, y, x', y') + k(x, y, y', x')$ & $k(x, y, x', y') \times k(x, y, y', x')$ & $k( \min(x, y), \max(x,y),$ \\
% $+ k(y, x, x', y') + k(y, x, y', x')$ & $\times k(y, x, x', y') \times k(y, x, y', x')$ & $\min(x', y'), \max(x',y') )$
$\begin{array}{r@{}l@{}}
& \kSE(x_1, x_2, x_1', x_2') \\ + \,\, & \kSE(x_1, x_2, x_2', x_1')
\end{array}$
&
$\begin{array}{r@{}l@{}}
\kSE( \!\! &{}\min(x_1, x_2), \max(x_1, x_2), \\
           &{}\min(x_1', x_2'), \max(x_1',x_2') )
\end{array}$
&
$\begin{array}{r@{}l@{}}
& \kSE(x_1, x_2, x_1', x_2') \\ \times \,\, & \kSE(x_1, x_2, x_2', x_1')
\end{array}$
\end{tabular}
\caption[Three ways to introduce symmetry]{Three methods of introducing symmetry, illustrated through draws from the corresponding priors.
%Left:  The additive method.
%Center: The product method.
%Right: The projection method.
%The additive method has half the marginal variance away from $y = x$, but the min method introduces a non-differentiable seam along $y = x$.
All three methods introduce a different type of nonstationarity.
}
\label{fig:add_vs_min}
\end{figure}
%
\begin{enumerate}

\item {\bf Sum over the Orbit.} 
\citet{ginsbourger2012argumentwise} and \citet{kondor2008group} suggest enforcing invariants through a double sum over the orbits of $\vx$ and $\vx'$ with respect to G:
%
\begin{align}
k_\textnormal{sum}(\vx, \vx') = \sum_{g, \in G} \sum_{g' \in G} k( g( \vx ), g'( \vx') )
\end{align}

For the group $\gswitch$, this operation results in the kernel:
%
\begin{align}
k_\textnormal{switch}(\vx, \vx')
& = \sum_{g, \in \gswitch} \sum_{g' \in \gswitch} k( g( \vx ), g'( \vx') ) \\
& = k(x_1, x_2, x_1', x_2') + k(x_1, x_2, x_2', x_1')  \nonumber \\ 
& \quad + k(x_2, x_1, x_1', x_2') + k(x_2, x_1, x_2', x_1')
\end{align}
%
For stationary kernels, some pairs of elements in this sum will be identical, and can be ignored.
\Cref{fig:add_vs_min}(left) shows a draw from a \gp{} prior with an $\kSE$ kernel symmetrized in this way.
This construction has the property that the marginal variance is doubled near $x = y$, which may or may not be desirable.



\item {\bf Project onto a Fundamental Domain.}
\citet{Invariances13} also explore the possibility of projecting each datapoint into a fundamental domain of the group, using a mapping $A_G$:
%
\begin{align}
k_\textnormal{proj}(\vx, \vx') = k( A_G(\vx), A_G( \vx') )
\end{align}
%
For the group $\gswitch$, a fundamental domain is $\{x, y : x < y\}$, which can be mapped to using $A_{\gswitch}( x, y ) = \big[ \min(x,y), \max(x,y) \big]$.
Constructing a kernel using this method introduces a non-differentiable  ``seam'' along $x = y$, as shown in \cref{fig:add_vs_min}(center).
%The projection method also works for infinite groups, as we shall see below.

\item {\bf Multiply over the Orbit.}
\citet{adams2013product} suggests a construction enforcing invariants through products over the orbits:
%
\begin{align}
k_\textnormal{sum}(\vx, \vx') = \prod_{g, \in G} \prod_{g' \in G} k( g( \vx ), g'( \vx') )
\end{align}
%
This method can sometimes produce \gp{} priors with zero variance in some regions of the space, as in \cref{fig:add_vs_min}(right).
%We include it here to show that each of these methods for enforcing symmetries modifies the resulting model in other ways as well.
\end{enumerate}
%
There are many possible ways to achieve a given symmetry, but we must be careful to do so without compromising other qualities of the model we are constructing.
For example, simply setting $k(\vx, \vx') = 0$ gives rise to a \gp{} prior which obeys \emph{all possible} symmetries, but this is presumably not a model we wish to use.




%In this section, we give recipes for expressing several classes of symmetries.  Later, we will show how these can be combined to produce more interesting structures.


\subsection{Example: Periodicity}

%We can enforce periodicity on any subset of the dimensions:
Periodicity in a one-dimensional function corresponds to the invariance
%
\begin{align}
f(x) = f( x + \tau)
\label{eq:periodic_invariance}
\end{align}
%
where $\tau$ is the period.

The most popular method for building a periodic kernel is due to \citet{mackay1998introduction}, who used the projection method in combination with an $\kSE$ kernel.
A fundamental domain of the symmetry group is a circle, so the kernel
%
%The representer transformation for periodicity is simply $A(x) = [\sin(x), \cos(x)]$:
%
\begin{align}
\kPer(x, x') = \kSE \left( \big[ \sin(x), \cos(x) \big], \big[ \sin(x'), \cos(x') \big] \right)
\end{align}
%
%We can also apply rotational symmetry repeatedy to a single dimension.
achieves the invariance in \cref{eq:periodic_invariance}.
Simple algebra reduces this kernel to the form shown in \cref{fig:basic_kernels}.

We could also build a periodic kernel with period $\tau$ by the mapping $A(x) = \mod(x, \tau)$.
However, samples from this prior would be discontinuous at every integer multiple of $\tau$.

\subsection{Example: Symmetry About Zero}

Another simple example of an easily-enforceable symmetry is symmetry about zero:
%
\begin{align}
f(x) = f( -x)
\end{align}
%
using the sum over orbits method, by the transform
%
\begin{align}
k_{\textnormal{reflect}}(x, x') & = k(x, x') + k(x, -x') + k(-x, x') + k(-x, -x')
\end{align}

%This transformation can be applied to any subset of dimensions

%\paragraph{Spherical Symmetry}

%We can also enforce that a function expresses the symmetries obeyed by $n-spheres$ by simply transforming a set of $n - 1$ coordinates by:
%
%\begin{align}
%x_1 & = \cos(\phi_1) \nonumber \\
%x_2 & = \sin(\phi_1) \cos(\phi_2) \nonumber \\
%x_3 & = \sin(\phi_1) \sin(\phi_2) \cos(\phi_3) \nonumber \\
%& \vdots \nonumber \\
%x_{n-1} & = \sin(\phi_1) \cdots \sin(\phi_{n-2}) \cos(\phi_{n-1}) \nonumber \\
%x_n & = \sin(\phi_1) \cdots \sin(\phi_{n-2}) \sin(\phi_{n-1})
%\end{align}

%\cite{flanders1989}



\subsection{Example: Translation Invariance in Images}

Most models of images are invariant to spatial translations \citep{lecun1995convolutional}.
Similarly, most models of sounds are also invariant to translation through time.

This sort of translation invariance is completely distinct from the stationarity of kernels such as $\kSE$ or $\kPer$.
A stationary kernel implies that the prior is invariant to translations of the entire training and test set.
In contrast, we use translation invariance to refer to the situation where the signal has been discretized, and each pixel (or the audio equivalent) corresponds to a different input dimension.
We are interested in creating priors on functions that are invariant to swapping pixels in a manner that corresponds to shifting in some direction:
%
\begin{align}
f \Bigg( \raisebox{-2.5ex}{ \includegraphics[width=1cm]{\topologyfiguresdir/grid2} } \Bigg) 
= f \Bigg( \raisebox{-2.5ex}{ \includegraphics[width=1cm]{\topologyfiguresdir/grid3} } \Bigg)
\end{align}
%
%In this setting, translation is equivalent to swapping dimensions of the input vector $\vx$.
For example, in a one-dimensional image or audio signal, translation of an input vector by $i$ pixels can be defined as
%
\begin{align}
\shift(\vx, i) = \big[ x_{\mod( i + 1, D )}, x_{\mod( i + 2, D )}, \dots, x_{\mod( i + D, D )} \big]\tra
\end{align}
%
As above, translation invariance in one dimension can be achieved by the transformation
%
\begin{align}
%k \left( (x_1, x_2, \dots, x_D ), (x_1', x_2', \dots, x_D' ) \right) & = %\nonumber \\
%\sum_{i=1}^D \prod_{j=1}^D k( x_j, x_{ i + j \textnormal{mod $D$} }' )
k_\textnormal{invariant} \left( \vx, \vx' \right) = %\nonumber \\
\sum_{i=1}^D \sum_{j=1}^D k( \shift(\vx, i), \shift(\vx, j) ) \,,
\end{align}
%
simply defining the covariance between two signals to be the sum of all covariances between all translations of those two signals.

The extension to two dimensions, $\shift(\vx, i, j)$, is straightforward, but notationally cumbersome.
\citet{kondor2008group} built a more elaborate kernel between images, approximately invariant to both translation and rotation by using the projection method.

%Is there a pathology of the additive construction that appears in the limit?

%\subsection{Max-pooling}
%What we'd really like to do is a max-pooling operation.  However, in general, a kernel which is the max of other kernels is not PSD [put counterexample here?].  Is the max over co-ordinate switching PSD?





\section{Generating Topological Manifolds}
\label{sec:topological-manifolds}

In this section we give a geometric illustration of the symmetries encoded by different compositions of kernels.
The work presented in this section is based on a collaboration with David Reshef, Roger Grosse, Joshua B. Tenenbaum, and Zoubin Ghahramani.
The derivation of the M\"obius kernel was an original contribution by myself.

Priors on functions exhibiting symmetries can be used to create a prior on topological manifolds, by warping a simply-connected surface into a higher-dimensional space using a mapping expressing symmetries. %from a latent space to an observed space.%surface $f(\vx)$.
%The distribution on $\vf$ allows us to put mass on 
%If we set the input $\vx$ to be a 2-dimensional plane, intro 3 dimensions using a \gp{} encoding certain symmetries, the resulting surfaces will correspon
To build a prior on 2-dimensional manifolds embedded in 3-dimensional space, we simply need a prior on mappings from $\mathbb{R}^2$ to $\mathbb{R}^3$, which we can construct using three independent functions $[f_1(\vx), f_2(\vx), f_3(\vx)]$, each mapping from $\mathbb{R}^2$ to $\mathbb{R}$.
\gp{} priors on the mapping functions implictly give rise to a prior on warped surfaces.
Symmetries in $[f_1, f_2, f_3]$ will connect different parts of the manifolds, giving rise to non-trivial topologies on the sampled surfaces.
%
\begin{figure}
\renewcommand{\tabcolsep}{1mm}
\begin{tabular}{ccc}
Euclidian $( \SE_1 \times \SE_2 )$  & Cylinder $( \SE_1 \times \Per_2 )$ & Toroid $( \Per_1 \times \Per_2 )$\\
\hspace{-0.5cm}\includegraphics[width=0.33\columnwidth]{\topologyfiguresdir/manifold} &
\includegraphics[width=0.33\columnwidth]{\topologyfiguresdir/cylinder} &
\includegraphics[width=0.33\columnwidth]{\topologyfiguresdir/torus} \\
\end{tabular}
\caption[Generating 2D manifolds with different topological structures]{
Generating 2D manifolds with different topologies.
By enforcing that the functions mapping from $\mathbb{R}^2$ to $\mathbb{R}^3$ obey the appropriate symmetries, the surfaces created have the corresponding topologies, ignoring self-intersections.
}
\label{fig:gen_surf}
\end{figure}
%
\Cref{fig:gen_surf} shows 2D meshes warped into 3D by functions drawn from \gp{} priors with different kernels, giving rise to a variety of different topologies.
%
Higher-dimensional analogues of these shapes can be constructed by increasing the latent dimension and including corresponding terms in the kernel.
For example, an $N$-dimensional space with kernel $\kPer_1 \kerntimes \kPer_2 \kerntimes \ldots \kerntimes \kPer_N$ will give rise to a prior on $N$-dimensional toruses.

This construction is similar in spirit to the \gp{} latent variable model (\gplvm{}) of \citet{lawrence2005probabilistic}, which learns a latent embedding of the data into a low-dimensional space, using a \gp{} prior on the mapping from the latent space to the observed space.%and constructs a fixed kernel structure over that space.
%The GP-LVM will be discussed in further detail in chapter ...

%\subsection{Surfaces, Cylinders and Torii}




\subsection{M\"{o}bius Strips}

A prior on functions on M\"{o}bius strips can be constructed by enforcing the symmetries:
%
\begin{align}
f(x, y) & = f( x, y + \tau) \\
f(x, y) & = f( x + \tau, y)  \\
f(x, y) & = f( y, x )
\end{align}
%
Moving along the diagonal $x = y$ of a function drawn from the corresponding \gp{} prior is equivalent to moving along the edge of a notional M\"{o}bius strip which has had the function mapped on to its surface.
\Cref{fig:mobius}(a) shows an example of a function drawn from such a prior.
%
\begin{figure}
\begin{tabular}[t]{ccc}
%\begin{columns}
\centering
Draw from \gp{} prior: &  &  \\
%$( \Per_1 \times \Per_2 )$ and $f(x,y) = f(y,x)$ & generated parametrically\\
$\begin{array}{l@{}l@{}}
&\Per(x_1, x_1') \kerntimes \Per(x_2, x_2') \\
          \kernplus & \Per(x_1, x_2') \kerntimes \Per(x_2, x_1')
\end{array}$
& $\begin{array}{c} \textnormal{ M\"{o}bius strip}  \\ \textnormal{drawn from manifold prior}  \end{array}$
 & $\begin{array}{c} \textnormal{Sudanese M\"{o}bius strip}  \\ \textnormal{generated parametrically}  \end{array}$\\
%$ + \Per(x_1, x_2') \times \Per(x_2, x_1')$ & \\
%\includegraphics[width=0.45\columnwidth, height=0.45\columnwidth, clip=true,trim=2cm 2cm 2cm 1cm]{\topologyfiguresdir/mobius_regression} \\
\includegraphics[width=0.3\columnwidth, height=0.3\columnwidth, clip=true,trim=3cm 2cm 2cm 2cm]{\topologyfiguresdir/mobius_field} & 
\includegraphics[width=0.3\columnwidth,clip=true,trim=1cm 0cm 0cm 0cm]{\topologyfiguresdir/mobius} &
%\includegraphics[width=0.33\columnwidth]{\topologyfiguresdir/mobius} 
%\begin{minipage}{0.33\columnwidth}
%{\begin{tabular}[t]{p{.3\columnwidth}}
%\includegraphics[width=0.3\columnwidth,clip=true,trim=0cm 0cm 10cm 0cm]{\topologyfiguresdir/sudanese-wikipedia}
%\\
%\includegraphics[width=0.3\columnwidth,clip=true,trim=10.55cm 0cm 0cm 0cm]{\topologyfiguresdir/sudanese-wikipedia}
%\end{tabular}}
%\end{minipage}
\raisebox{1cm}{\includegraphics[width=0.25\columnwidth,clip=true,trim=0cm 0cm 14.6cm 0cm]{\topologyfiguresdir/sudanese-wikipedia}}
\end{tabular}
\caption[Generating M\"{o}bius strips]{Generating M\"{o}bius strips.
\emph{Left:} A function drawn from a \sgp{} prior obeying the same symmetries as a M\"{o}bius strip.
\emph{Center:} By enforcing that the functions mapping from $\mathbb{R}^2$ to $\mathbb{R}^3$ obey the appropriate symmetries, surfaces sampled from the prior have topology corresponding to a M\"{o}bius strip.
Surfaces generated this way do not have the familiar shape of a flat surface connected to itself with a half-twist.
Instead, they tend to look like \emph{Sudanese} M\"{o}bius strips \citep{sudanese1984}, whose edge has a circular shape.
\emph{Right:} A Sudanese projection of a M\"{o}bius strip.
Image adapted from \citep{sudanesepict}.
}
\label{fig:mobius}
\end{figure}

\Cref{fig:mobius}(b) shows an example of a 2D mesh mapped to 3D by functions drawn from such a prior.
This surface doesn't resemble the typical representation of a M\"{o}bius strip,
%, because the edge of the M\"{o}bius strip is in roughly circular shape, as opposed to the double-loop that one obtains by gluing a strip of paper with a single twist.
but instead resembles an embedding known as the Sudanese M\"{o}bius strip \citep{sudanese1984}, shown in \Cref{fig:mobius}(c).

%Another classic example of a function living on a Mobius strip is the auditory quality of 2-note intervals.  The harmony of a pair of notes is periodic (over octaves) for each note, and the 



\section{Kernels on Categorical Variables}

%Kernels can be defined over all types of data structures: Text, images, matrices, and even kernels . Coming up with a kernel on a new type of data used to be an easy way to get a NIPS paper.

%\subsection{}

%There is a simple way to do \gp{} regression over categorical variables:
Categorical variables are variables which can take values only from a discrete, unordered set, such as $\{\texttt{blue}, \texttt{green}, \texttt{red}\}$.
A flexible way to construct a kernel over categorical variables is to represent that variable by a set of binary variables, using a one-of-k encoding.
For example, if $\vx$ can take one of four values, $x \in \{ \texttt{A}, \texttt{B}, \texttt{C}, \texttt{D}\}$, then a one-of-k encoding of $x$ will correspond to four binary inputs, and $\oneofk(\texttt{C}) = [0, 0, 1, 0]$.
Given a one-of-k encoding, we can place any multidimensional kernel on that space, such as the \seard{}:
%
\begin{align}
k_{\textnormal{categorical}}( x, x') = \seard( \oneofk(x), \oneofk(x') )
\end{align}
%
Short lengthscales on any particular dimension in the $\seard$ kernel indicate that the function value corresponding to that category is uncorrelated with the others.

A more flexible parameterization suggested by \citet{swersky2013categorical} allows complete flexibility about which pairs of categories are similar to one another, replacing the $\seard$ kernel with a fully-parameterized kernel, $\sefull$:
%
\begin{align}
\sefull( \vx, \vx') = \exp \left( -\frac{1}{2} \vx\tra \vL \vx' \right)
\end{align}
%
where $L$ is a symmetric matrix, individually parameterizing the covariance between each pair of function values of categories.

%Then, simply put a product of kernels on those dimensions.
%This is the same as putting one SE ARD kernel on all of them.
%Learning the lengthscale on each dimension of the $\kSE$ kernel will now encode how similar the value of the different categories are to one another.
%The lengthscale hyperparameter will now encode whether, when that coding is active, the rest of the function changes.
%If you notice that the estimated lengthscales for your categorical variables is short, your model is saying that it's not sharing any information between data of different categories. 









\iffalse

\section{Worked example: building a structured kernel for a time-series}

%\subsection{Modeling multiple periodicities}

\begin{figure}[h]
\begin{tabular}{ccc}
Long-term trend & Weekly periodicity &Yearly periodicity \\
\includegraphics[width=0.31\columnwidth]{\examplefigsdir/births-component-1} &
\includegraphics[width=0.31\columnwidth]{\examplefigsdir/births-component-3} & 
\includegraphics[width=0.31\columnwidth]{\examplefigsdir/births-component-2-zoom} 
\end{tabular}
\caption[Composite model of births data]{A composite \gp{} model of births data. (blue)}
\label{fig:quebec-decomp}
\end{figure}





\iffalse
\begin{figure}
\renewcommand{\tabcolsep}{1mm}
\def \incpic#1{\includegraphics[width=0.200\columnwidth]{../figures/worked-example/births-#1}}
\begin{tabular}{*{5}{c}}
 & {Long-term} & {Weekly} & {Yearly} & {Short-term} \\ 
 \rotatebox{90}{{Long-term}} & \incpic{Long-term-Long-term} & \incpic{Long-term-Weekly} & \incpic{Long-term-Yearly} & \incpic{Long-term-Short-term} \\ 
 \rotatebox{90}{{Weekly}} & \incpic{Weekly-Long-term} & \incpic{Weekly-Weekly} & \incpic{Weekly-Yearly} & \incpic{Weekly-Short-term} \\ 
 \rotatebox{90}{{Yearly}} & \incpic{Yearly-Long-term} & \incpic{Yearly-Weekly} & \incpic{Yearly-Yearly} & \incpic{Yearly-Short-term} \\ 
 \rotatebox{90}{{Short-term}} & \incpic{Short-term-Long-term} & \incpic{Short-term-Weekly} & \incpic{Short-term-Yearly} & \incpic{Short-term-Short-term} \\ 
 \end{tabular}
\caption[Two-way interactions in births data]{Two-way interactions in births data}
\label{fig:quebec-decomp}
\end{figure}
\fi
%\subsection{Incoportating discrete covariates}

%\subsection{Breaking down the predictions, examining different parts of the model}
\fi





\section{Building a Kernel in Practice}

%\subsection{Learning Kernel Parameters}

One difficulty in building \gp{} models is choosing, or integrating over, the kernel parameters.
%Fortunately, typical kernels only have $\mathcal{O}(D)$ parameters, meaning that if $N$ is reasonably large, these parameters can be estimated by maximum marginal likelihood.
If the kernel we construct has relatively few parameters, these parameters can be estimated by maximum marginal likelihood, using gradient-based optimizers.
The kernel parameters estimated in the examples above were optimized using the \GPML{} toolbox, available at \url{gaussianprocess.org/gpml/code}.

%\subsection{Choosing the Kernel Form}
%The marginal likelihood of a model is useful for choosing among parameters.
%The marginal likelihood can also be used to selecting which type of kernel to use.
% the form of the kernel.
%For example, we might not know whether a particular structure or symmetry is present in the function we are trying to model.
%Again, the fact that we can compare marginal likelihoods in \gp{}s means that we can 
%Because \gp{}s let us build models both with and without certain symmetries, 
%By building kernels with and without such structure, we can compute the marginal likelihoods of the corresponding \gp{} models.
%The quantities represent the relative amount of evidence that the data provide for each of these possibilities, providing the assumptions of the model are correct.
%To do so, we simple need to compare the marginal likelihood of the data
%We demonstrate that marginal likeihood an be used to automatically search over such structures.
The next chapter will show how to perform such a search not just over the kernel parameters, but also over the open-ended space of kernel expressions.

\subsubsection{Source Code}
Source code to produce all figures is available at \url{github.com/duvenaud/phd-thesis}.

%\section{Conclusion}

%We've seen that kernels are a flexible and powerful language for building models of different types of functions.
%However, for a given problem, it can difficult to specify an appropriate kernel, even after looking at the data.
%A better procedure would be to compare the predictive performance, or marginal likelihood, of a few different kernels.
%However, it might be difficult to enumerate all plausible kernels, and tedious to search over them.
%In fact, choosing the kernel can be considered one of the main difficulties in doing inference.

%Analogously, we usually don't expect to simply guess the best value of some parameter.
%Rather, we specify a search space and an objective, and ask the computer to the search this space for us. 



\outbpdocument{
\bibliographystyle{plainnat}
\bibliography{references.bib}
}



\iffalse


\subsection{Example: Computing Molecular Energies}

\begin{figure}
\begin{center}
\begin{tabular}{cc}
Function on M\"{o}bius strip & \\
\includegraphics[width=0.3\columnwidth, height=0.3\columnwidth, clip=true,trim=3cm 2cm 2cm 2cm]{\topologyfiguresdir/mobius_field} & 
  \begin{tikzpicture}

%\pgfmathsetmacro{\r}{3cm}
%\pgfmathsetmacro{\ho}{70}
%\pgfmathsetmacro{\ht}{30}
\newcommand{\radius}{3}
\newcommand{\hone}{120}
\newcommand{\htwo}{70}
\newcommand{\hthree}{30}

	\coordinate (O) at (0, 0);
	\coordinate (left) at ({\radius*cos(\hone)}, {\radius*sin(\hone)});
	\coordinate (right) at ({\radius*cos(\htwo)}, {\radius*sin(\htwo)});
	\coordinate (zero) at ({\radius*cos(\hthree)}, {\radius*sin(\hthree)});

	\draw[fill] (left) circle (2pt);
	\draw (left) node[below, left] {H};
	
	\draw[fill] (right) circle (2pt);
	\draw (right) node[right] {H};

	\draw[fill] (zero) circle (2pt);
	\draw (zero) node[right] {H};

	\draw[fill] (O) circle (3pt);
	\draw (O) node[below] {C};

	\draw (left) -- (O);
	\draw (right) -- (O);
	\draw (zero) -- (O);

	\begin{scope}
	\path[clip] (O) -- (right) -- (zero);
	\fill[red, opacity=0.5, draw=black] (O) circle (2);
	\node at ($(O)+(50:1.6)$) {$\theta_1$};	
	\end{scope}
	
	\begin{scope}
	\path[clip] (O) -- (left) -- (right);
	\fill[green, opacity=0.5, draw=black] (O) circle (1.8);
	\node at ($(O)+(90:1.4)$) {$\theta_2$};	
	\end{scope}	
  \end{tikzpicture}
\end{tabular}
\end{center}
\caption[The energy of a molecular configuration obeys the same symmetries as a M\"{o}bius strip]{An example of a function expressing the same symmetries as a M\"{o}bius strip in two of its arguments.  The energy of a molecular configuration $f(\theta_1, \theta_2)$ depends only on the relative angles between atoms, and because each atom is indistinguishable, is invariant to permuting the atoms. }
\label{fig:molecule}
\end{figure}

Figure \ref{fig:molecule} gives one example of a function which obeys the same symmetries as a M\"{o}bius strip, in some subsets of its arguments.

\fi
