%
% A header that lets you compile a chapter by itself, or inside a larger document.
% Adapted from http://stackoverflow.com/questions/3655454/conditional-import-in-latex
%
%
%Use \inbpdocument and \outbpdocument in your individual files, in place of \begin{document} and \end{document}. In your main file, put in a \def \ismaindoc {} before including or importing anything.
%
% David Duvenaud
% June 2011
% 
% ======================================
%
%


\ifx\ismaindoc\undefined
	\newcommand{\inbpdocument}{
		\def \ismaindoc {}
		% Use this header if we are compiling by ourselves.
		\documentclass[a4paper,12pt,authoryear,index]{common/PhDThesisPSnPDF}
		\input{common/official-preamble.tex}
		% All my custom preamble stuff.  Shouldn't overlap with anything in official-preamble

\usepackage{nth}
\usepackage{rotating}
\usepackage{array}
%\usepackage{gantt}
\usepackage[hyperpageref]{backref}

		% ************************ Thesis Information & Meta-data **********************

%% The title of the thesis
%\title{Structured Gaussian Process Models} 
%\title{Automatic Model Construction \\ through \\ Structured Gaussian Processes}
%\title{Automatic Model-Building \\ through \\ Structured Gaussian Processes}
%\title{Automatic Modeling \\ with \\ Structured Gaussian Processes}    
\title{Automatic Model Construction \\ with Gaussian Processes}
%\title{Automatic Model Construction}
%\title{Automating Statistical Model Construction}


%\texorpdfstring is used for PDF metadata. Usage:
%\texorpdfstring{LaTeX_Version}{PDF Version (non-latex)} eg.,
%\texorpdfstring{$sigma$}{sigma}

%% The full name of the author
\author{David Kristjanson Duvenaud}

%% Department (eg. Department of Engineering, Maths, Physics)
\dept{Department of Engineering}

%% University and Crest
\university{University of Cambridge}
\crest{\includegraphics[width=0.25\textwidth]{University_Crest}}

%% You can redefine the submission text:
% Default as per the University guidelines: This dissertation is submitted for
% the degree of Doctor of Philosophy
%\renewcommand{\submissiontext}{change the default text here if needed}

%% Full title of the Degree 
\degree{Doctor of Philosophy}
 
%% College affiliation (optional)
\college{Pembroke College}

%% Submission date
\degreedate{June 2014} 

%% Meta information
\subject{LaTeX} \keywords{{LaTeX} {PhD Thesis} {Engineering} {University of
Cambridge}}



		\begin{document}
	}	
	\newcommand{\outbpdocument}[1]{
		%\bibliographystyle{common/CUEDthesis}
		\bibliographystyle{plainnat}
		\bibliography{references.bib}
		\end{document}
	}	
\else
	%If we're inside another document, no need to re-start the document.
	\ifx\inbpdocument\undefined
		\newcommand{\inbpdocument}{}
		\newcommand{\outbpdocument}[1]{}
	\fi
\fi

\inbpdocument


\chapter{Automatically Describing Structured Covariance Functions}
\label{ch:description}





This paper presents the beginnings of an automatic statistician, focusing on regression problems.
Our system explores an open-ended space of possible statistical models to discover a good explanation of the data, and then produces a detailed report with figures and natural-language text.

Our approach treats unknown functions nonparametrically using Gaussian processes, which has two important consequences.
First, Gaussian processes model functions in terms of high-level properties (e.g.\ smoothness, trends, periodicity,
changepoints).
Taken together with the compositional structure of our language of models, this allows us to automatically describe functions through a decomposition into additive parts.
Second, the use of flexible nonparametric models and a rich
language for composing them in an open-ended manner also results
in state-of-the-art extrapolation performance evaluated over 13 real time
series data sets from various domains.


\subsection{Attribution}
Joint work with James Robert Lloyd, Roger Grosse, Joshua B. Tenenbaum and Zoubin Ghahramani.








\section{Automatic description of regression models}
\label{sec:description}

\paragraph{Overview}

In this section, we describe how \procedurename{} generates natural-language descriptions of the models found by the search procedure.
There are two main features of our language of \gp{} models that allow description to be performed automatically.

First, the sometimes complicated kernel expressions found can be simplified into a sum of products.
A sum of kernels corresponds to a sum of functions so each product can be described separately.
Second, each kernel in a product modifies the resulting model in a consistent way.
Therefore, we can choose one kernel to be described as a noun, with all others described using adjectives.

\paragraph{Sum of products normal form} 

We convert each kernel expression into a standard, simplified form.
We do this by first distributing all products of sums into a sum of products.
Next, we apply several simplifications to the kernel expression:
The product of two $\kSE$ kernels is another $\kSE$ with different parameters. Multiplying $\kWN$ by any stationary kernel ($\kC$, $\kWN$, $\kSE$, or $\kPer$) gives another $\kWN$ kernel. Multiplying any kernel by $\kC$ only changes the parameters of the original kernel.

After applying these rules, the kernel can as be written as a sum of terms of the form:
\begin{align}
K \prod_m \kLin^{(m)} \prod_n \boldsymbol\sigma^{(n)},
\label{eq:sop}
\end{align}
where $K$ is one of \kWN, \kC, \kSE, $\prod_k \kPer^{(k)}$ or $\kSE \prod_k \kPer^{(k)}$
and $\prod_i\kernel^{(i)}$ denotes a product of kernels, each with different parameters.


\paragraph{Sums of kernels are sums of functions}
Formally, if $f_1(x) \dist \gp{}(0, \kernel_1)$ and independently $f_2(x) \dist \gp{}(0, \kernel_2)$ then ${f_1(x) + f_2(x) \dist \gp{}(0, \kernel_1 + \kernel_2)}$.
This lets us describe each product of kernels separately.


\paragraph{Each kernel in a product modifies a model in a consistent way}
This allows us to describe the contribution of each kernel in a product as an adjective, or more generally as a modifier of a noun.
We now describe how each kernel modifies a model and how this can be described in natural language:

\begin{itemize}
\item {\bf Multiplication by $\kSE$} removes long range correlations from a model since $\kSE(x,x')$ decreases monotonically to 0 as $|x - x'|$ increases.
This can be described as making an existing model's correlation structure `local' or `approximate'.
\item {\bf Multiplication by $\kLin$} is equivalent to multiplying the function being modeled by a linear function.
If $f(x) \dist \gp{}(0, \kernel)$, then $xf(x) \dist \gp{}\left(0, k \times \kLin \right)$.
This causes the standard deviation of the model to vary linearly without affecting the correlation and can be described as \eg `with linearly increasing standard deviation'.
\item {\bf Multiplication by $\boldsymbol\sigma$} is equivalent to multiplying the function being modeled by a sigmoid which means that the function goes to zero before or after some point.
This can be described as \eg `from [time]' or `until [time]'.
\item {\bf Multiplication by $\kPer$}
modifies the correlation structure in the same way as multiplying the function by an independent periodic function.
Formally, if ${f_1(x) \dist \gp{}(0, \kernel_1)}$ and ${f_2(x) \dist \gp{}(0, \kernel_2)}$ then
\begin{align}
{\textrm{Cov} \left[f_1(x)f_2(x), f_1(x')f_2(x') \right] = k_1(x,x')k_2(x,x')}.\nonumber
\end{align}
This can be loosely described as \eg `modulated by a periodic function with a period of [period] [units]'.
\end{itemize}

\paragraph{Constructing a complete description of a product of kernels}
We choose one kernel to act as a noun which is then described by the functions it encodes for when unmodified \eg `smooth function' for $\kSE$.
Modifiers corresponding to the other kernels in the product are then appended to this description, forming a noun phrase of the form:
\begin{align*}
\textnormal{Determiner}	+	\textnormal{Premodifiers} +	\textnormal{Noun}	+	\textnormal{Postmodifiers}
\end{align*}

As an example, a kernel of the form $\kSE \times \kPer \times  \kLin \times \boldsymbol{\sigma}$ could be described as an
\begin{align*}
\underbrace{\kSE}_{\textnormal{\scriptsize approximately}} \times 
\underbrace{\kPer}_{\textnormal{\scriptsize periodic function}} \times 
\underbrace{\kLin}_{\textnormal{\scriptsize with linearly growing amplitude}} \times 
\underbrace{\boldsymbol{\sigma}}_{\textnormal{\scriptsize until 1700.}}
\end{align*}
where $\kPer$ has been selected as the head noun.

In principle, any assignment of kernels in a product to these different phrasal roles is possible, but in practice we found certain assignments to produce more interpretable phrases than others.
The head noun is chosen according to the following ordering:
\begin{align*}
\kPer > \kWN, \kSE, \kC > \prod_m \kLin^{(m)} > \prod_n \boldsymbol\sigma^{(n)}
\end{align*}
\ie $\kPer$ is always chosen as the head noun when present.

\paragraph{Ordering additive components}

The reports generated by \procedurename{} attempt to present the most interesting or important features of a data set first.
As a heuristic, we order components by always adding next the component which most reduces the 10-fold cross-validated mean absolute error.

\subsection{Worked example}

Suppose we start with a kernel of the form
\begin{align*}
\kSE \times (\kWN \times \kLin + \kCP(\kC, \kPer)).
\end{align*}
This is converted to a sum of products:
\begin{align*}
\kSE \times \kWN \times \kLin + \kSE \times \kC \times \boldsymbol{\sigma} + \kSE \times \kPer \times \boldsymbol{\bar\sigma}.
\end{align*}
which is simplified to
\begin{align*}
\kWN \times \kLin + \kSE \times \boldsymbol{\sigma} + \kSE \times \kPer \times \boldsymbol{\bar\sigma}.
\end{align*}

To describe the first component, the head noun description for $\kWN$, `uncorrelated noise', is concatenated with a modifier for $\kLin$, `with linearly increasing standard deviation'.
%
The second component is described as `A smooth function with a lengthscale of [lengthscale] [units]', corresponding to the $\kSE$, `which applies until [changepoint]', which corresponds to the $\boldsymbol{\sigma}$.
%
Finally, the third component is described as `An approximately periodic function with a period of [period] [units] which applies from [changepoint]'.

%Descriptions are sometimes modified depending on kernel parameters.  
%Also, a separate description procedure is used to produce the detailed descriptions of each component, which operates on the same principles.

%\fTBD{JL: We should also mention that the detailed descriptions are less modular - but not necessarily so}
%\fTBD{JL: Talk about different templates based on parameters?}








Automating the process of statistical modeling would have a tremendous impact on fields that currently rely on expert statisticians, machine learning researchers, and data scientists.
While fitting simple models (such as linear regression) is largely automated by standard software packages, there has been little work on the automatic construction of flexible but interpretable models. 
What are the ingredients required for an artificial intelligence system to be able to perform statistical modeling automatically? 
In this paper we conjecture that the following ingredients may be useful for building an AI system for statistics, and we develop a working system which incorporates them:
\begin{itemize}
\item {\bf An open-ended language of models} expressive enough to
  capture many of the modeling assumptions and model composition
  techniques  applied by human statisticians to capture real-world phenomena
\item {\bf A search procedure} to efficiently explore the space of
  models spanned by the language
\item {\bf A principled method for evaluating models} in terms of their complexity and their degree of fit to the
  data
\item {\bf A procedure for automatically generating reports} which
  explain and visualize different factors underlying the data, make
  the chosen modeling assumptions explicit, and quantify how each
  component improves the predictive power of the model 
\end{itemize}

\begin{figure}[t]
\centering
\fbox{\includegraphics[trim=0cm 4.75cm 0cm 1.0cm, clip, width=0.98\columnwidth]{\grammarfiguresdir/solarpages/pg_0007-crop}}
\caption[Extract from an automatically-generated report]
{Extract from an automatically-generated report describing the model components discovered by automatic model search.  This part of the report isolates and describes the approximately 11-year sunspot cycle, also noting its disappearance during the 16th century, a time known as the Maunder minimum \citep{lean1995reconstruction}.}
\label{fig:periodic}
\end{figure}

The compositional structure of the language allows us to develop a method
for automatically translating components of the model into
natural-language descriptions of patterns in the data.

We show examples of automatically generated reports which highlight
interpretable features discovered in a variety of data sets (e.g.\
figure~\ref{fig:periodic}).  The supplementary material to this paper
includes 13 complete reports automatically generated by \procedurename{}.

Good statistical modeling requires not only
interpretability but predictive accuracy.





\paragraph{Airline passenger data}



Figure \ref{fig:airline_decomp} shows the decomposition produced by applying our method to monthly totals of international airline passengers~\citep{box2011time}.
We observe similar components to the previous dataset: a long term trend, annual periodicity and medium-term deviations.
In addition, the composite kernel captures the near-linearity of the long-term trend, and the linearly growing amplitude of the annual oscillations.
%The discovery of linear structure allows for a growing uncertainty highly plausible extrapolation.\fTBD{Mention heteroskedasticity?}

%Discovering this linearity results in very plausible extrapolations of the data.
%Figure~\ref{fig:airline_grow} demonstrates this by ploting the full posterior of the kernel discovered at sequential levels of the search.
%The first plot shows that the first kernel has discovered onyl short term continuity.
%The second kernel discovers periodicity that varies with time.
%The third kernel has discovered that the periodicity is growing approximately linearly.
%Finally, the fourth kernel has split the data into an approximately linear trend and growing oscillations.

%\begin{figure}[h!]
%\centering
%\newcommand{\wag}{4.8cm}  % width airline growth
%\newcommand{\hag}{4cm}  % height airline growth
%\begin{tabular}{cc}
%\includegraphics[width=\wag,height=\hag]{../figures/decomposition/01-airline-s_max_level_0/01-airline-s_all_small}
%\hspace{-1cm} \includegraphics[width=\wag,height=\hag]{../figures/decomposition/01-airline-s_max_level_1/01-airline-s_all_small} \\
%\hspace{-1cm} \includegraphics[width=\wag,height=\hag]{../figures/decomposition/01-airline-s_max_level_2/01-airline-s_all_small} 
%\hspace{-1cm} \includegraphics[width=\wag,height=\hag]{../figures/decomposition/01-airline-s_max_level_3/01-airline-s_all_small}
%\end{tabular}
%\caption{Posterior mean and variance for different levels of kernel search on the airline dataset. \TBD{RBG: it looks like the figures on the left got clipped.}}
%\label{fig:airline_grow}
%\end{figure}

% Created by the command:
% postprocessing.make_all_1d_figures(folder='../results/31-Jan-1d/', prefix='31-Jan-v3', rescale=False)

%To evaluate the effectiveness of our method we performed two types of experiments.
%First, we examined the ability of our algorithm to discover useful structure in one-dimensional datasets.  Second, we examined the predictive performance of our model on multi-dimensional datasets.
% testing the ability of the algorithm to infer both an accurate regression function estimate and to produce an interpretable decomposition of a regression function.
%The accuracy experiments show that our method consistently matches or beats previous state of the art regression methods (\TBD{not yet statistically significantly within each experiment - more experiments currently running\ldots}).
%The decomposition experiments produce highly interpretable decompositions of time series data and highly plausible extrapolations; a particularly rare property of naively applied (linear) smoothers (\TBD{justify comment - e.g.~local linear is just linear, GP Sq-Exp nose dives towards the mean}).



\paragraph{Solar irradiance Data} 
Finally, we analyzed annual solar irradiation data from 1610 to 2011 \citep{lean1995reconstruction}.
%
\begin{figure}
\newcommand{\wsd}{0.5\columnwidth}  % width solar decomp
\newcommand{\hsd}{4cm}  % height solar decomp
\newcommand{\srd}{\grammarfiguresdir/decomposition/11-Feb-02-solar-s}  % solar decomp results dir
\newcommand{\mbs}{\hspace{-0.3cm}}  % move back
\begin{tabular}{cc}
\mbs \includegraphics[width=\wsd,height=\hsd]{\srd/02-solar-s_all} &
\mbs \includegraphics[width=\wsd,height=\hsd]{\srd/02-solar-s_resid}
\end{tabular}
\caption[Decomposition of model discovered on solar irradiance dataset]
{Full posterior and residuals on the solar irradiance dataset.}
\label{fig:solar_decomp}
\end{figure}
%


%\paragraph{Multidimensional decomposition}  Decomposition of multi-dimensional posteriors into sums of one-dimensional functions is possible as well, and was demonstrated in \cite{duvenaud2011additive11}.  \fTBD{Probably citing myself too much...}






We demonstrate the ability of our procedure to discover and describe a variety of patterns on two time series.
Full automatically-generated reports for 13 data sets are provided as supplementary material.

\subsection{Summarizing 400 Years of Solar Activity}
\label{sec:solar}

\begin{figure}[h]
\centering
\includegraphics[trim=0.2cm 18.0cm 8cm 2cm, clip, width=0.5\columnwidth]{\grammarfiguresdir/solarpages/pg_0002-crop}
\caption[Solar irradiance dataset]
{Solar irradiance data.}
\label{fig:solar}
\end{figure}

We show excerpts from the report automatically generated on annual solar irradiation data from 1610 to 2011 (figure~\ref{fig:solar}).
This time series has two pertinent features: a roughly 11-year cycle of solar activity, and a period lasting from 1645 to 1715 with much smaller variance than the rest of the dataset.
This flat region corresponds to the Maunder minimum, a period in which sunspots were extremely rare \citep{lean1995reconstruction}.
\procedurename{} clearly identifies these two features, as discussed below.


\begin{figure}[h]
\centering
\fbox{\includegraphics[trim=0cm 4.75cm 0cm 1.0cm, clip, width=0.98\columnwidth]{\grammarfiguresdir/solarpages/pg_0007-crop}}
\caption[Extract from an automatically-generated report]
{Extract from an automatically-generated report describing the model components discovered by automatic model search.  This part of the report isolates and describes the approximately 11-year sunspot cycle, also noting its disappearance during the 16th century, a time known as the Maunder minimum \citep{lean1995reconstruction}.}
\label{fig:periodic}
\end{figure}

\begin{figure}[h]
\centering
\fbox{\includegraphics[trim=0cm 10.8cm 0cm 6.3cm, clip, width=0.98\columnwidth]{\grammarfiguresdir/solarpages/pg_0002-crop}}
\caption[Automatically-generated descriptions of the solar irradiance data set]
{Automatically generated descriptions of the components discovered by \procedurename{} on the solar irradiance data set.
The dataset has been decomposed into diverse structures with simple descriptions.}
\label{fig:exec}
\end{figure}
Figure \ref{fig:exec} shows the natural-language summaries of the top four components chosen by \procedurename{}.
From these short summaries, we can see that our system has identified the Maunder minimum (second component) and 11-year solar cycle (fourth component).
These components are visualized in figures~\ref{fig:maunder} and \ref{fig:periodic}, respectively. 
The third component corresponds to long-term trends, as visualized in figure~\ref{fig:smooth}.

\begin{figure}[ht]
\centering
\fbox{\includegraphics[trim=0cm 4.75cm 0cm 0.7cm, clip, width=0.98\columnwidth]{\grammarfiguresdir/solarpages/pg_0005-crop}}
\caption[A learned component corresponding to the Maunder minimum]
{One of the learned components corresponds to the Maunder minimum.}
\label{fig:maunder}
\end{figure}

\begin{figure}[h!]
\centering
\fbox{\includegraphics[trim=0cm 4.75cm 0cm 1cm, clip, width=0.98\columnwidth]{\grammarfiguresdir/solarpages/pg_0006-crop}}
\caption[\procedurename{} isolating the part of the signal explained by a slowly-varying trend]
{Characterizing the medium-term smoothness of solar activity levels.  By allowing other components to explain the periodicity, noise, and the Maunder minimum, \procedurename{} can isolate the part of the signal best explained by a slowly-varying trend.}
\label{fig:smooth}
\end{figure}

\subsection{Finding heteroscedasticity in air traffic data}
\label{sec:airline}

\begin{figure}[h]
\centering
\includegraphics[trim=0.4cm 16.8cm 8cm 2cm, clip, width=0.5\columnwidth]{\grammarfiguresdir/airlinepages/pg_0002-crop}
\caption[International airline passenger monthly volume dataset]
{International airline passenger monthly volume \citep[e.g.][]{box2013time}.}
\label{fig:airline}
\end{figure}

Next, we present the analysis generated by our procedure on international airline passenger data (figure~\ref{fig:airline}).
The model constructed by \procedurename{} has four components: $\kLin + \kSE \times \kPer \times \kLin + \kSE + \kWN \times \kLin$, with descriptions given in figure~\ref{fig:exec-airline}.

\begin{figure}[h]
\centering
\fbox{\includegraphics[trim=0cm 6.8cm 0cm 6cm, clip, width=0.98\columnwidth]{\grammarfiguresdir/airlinepages/pg_0002-crop}}
\caption[Short descriptions of the four components of the airline model]
{Short descriptions and summary statistics for the four components of the airline model.}
\label{fig:exec-airline}
\end{figure}

The second component (figure~\ref{fig:lin_periodic}) is accurately described as approximately ($\kSE$) periodic ($\kPer$) with linearly growing amplitude ($\kLin$).
%
\begin{figure}[h]
\centering
\fbox{\includegraphics[trim=0cm 4.75cm 0cm 0cm, clip, width=0.98\columnwidth]{\grammarfiguresdir/airlinepages/pg_0004-crop}}
\caption[Capturing non-stationary periodicity in the airline data]
{Capturing non-stationary periodicity in the airline data}
\label{fig:lin_periodic}
\end{figure}
%
By multiplying a white noise kernel by a linear kernel, the model is able to express heteroscedasticity (figure~\ref{fig:heteroscedastic}).
%
\begin{figure}[h]
\centering
\fbox{\includegraphics[trim=0cm 0cm 0cm 0cm, clip, width=0.98\columnwidth]{\grammarfiguresdir/airlinepages/pg_0006-crop}}
\caption[Modeling heteroscedasticity in the airline dataset]
{Modeling heteroscedasticity in the airline dataset.}
\label{fig:heteroscedastic}
\end{figure}

\subsection{Comparison to equation learning}
\label{sec:eqn-learning-comp}

We now compare the descriptions generated by \procedurename{} to parametric functions produced by an equation learning system.
We show equations produced by Eureqa \citep{Eureqa} for the data sets shown above, using the default mean absolute error performance metric.

The learned function for the solar irradiance data is
\begin{align*}
\textrm{Irradiance($t$)} = 1361 + \alpha\sin(\beta + \gamma t)\sin(\delta + \epsilon t^2 - \zeta t)
\end{align*}
where $t$ is time and constants are replaced with symbols for brevity.
This equation captures the constant offset of the data, and models the long-term trend with a product of sinusoids, but fails to capture the solar cycle or the Maunder minimum.

The learned function for the airline passenger data is
\begin{align*}
\textrm{Passengers($t$)} = \alpha t + \beta\cos(\gamma - \delta t)\textrm{logistic}(\epsilon t - \zeta) - \eta
\end{align*}
which captures the approximately linear trend, and the periodic component with approximately linearly (logistic) increasing amplitude.
However, the annual cycle is heavily approximated by a sinusoid and the model does not capture heteroscedasticity.






\section{Related work}
\label{sec:related-work}


\subsection{Natural-language output}
To the best of our knowledge, our procedure is the first example of automatic description of nonparametric statistical models.
However, systems with natural language output have been built in the areas of video interpretation \citep{barbu2012video} and automated theorem proving \citep{GanesalingamG13}.





\subsection{Interpretability versus accuracy}

BIC trades off model fit and complexity by penalizing the number of parameters in a kernel expression.
This can result in \procedurename{} favoring kernel expressions with nested products of sums, producing descriptions involving many additive components.
While these models have good predictive performance the large number of components can make them less interpretable.
We experimented with distributing all products over addition during the search, causing models with many additive components to be more heavily penalized by BIC.
We call this procedure \procedurename{}-interpretability, in contrast to the unrestricted version of the search, \procedurename{}-accuracy.


%\fTBD{Josh: Can you write about the Blessing of abstraction, and doing lots with a small amount of data?  This might become apparent from plotting the extraplolations.}
%\fTBD{RBG: We could make this point as part of the learning curves for extrapolation by marking the points at which each additional aspect of the structure is found.}




\section{Conclusion}

Towards the goal of automating statistical modeling we have presented a system which constructs an appropriate model from an open-ended language and automatically generates detailed reports that describe patterns in the data captured by the model.
We have demonstrated that our procedure can discover and describe a variety of patterns on several time series.
Our procedure's extrapolation and interpolation performance on time-series are state-of-the-art compared to existing model construction techniques.
We believe this procedure has the potential to make powerful statistical model-building techniques accessible to non-experts.










\section{Discussion}


%There has developed a large and mature literature on automatic structure discovery in unsupervised settings \fTBD{citations including Josh}.  
%For example, searches over large model classes have been succesfully used in machine vision \cite{pinto2009high}, analogous to high-throughput methods used in biology.
%However, relatively little attention has been paid to automatic structure discovery in supervised learning.
%
%Our work respresents a step towards automated structure discover in supervised settings.
% data-based modeling, as opposed to proposing the model beforehand.

%In particular, this work removes the need for an expert to propose a kernel family in order to effectively use a Gaussian process model.  
%Choosing a kernel family is a stumbling block even for experts.
%Automating the search over kernel structures will make it easier for non-statisticians to construct appropriate models for their data.

%The ability to learn kernel parameters and combination weights automatically has been an important factor in enabling the widespread use of kernel methods.
%For the most part, however, it has been up to the user to choose the proper form for the kernel, a task which requires considerable expertise.

Towards the goal of automating the choice of kernel family, we introduced a space of composite kernels defined compositionally as sums and products of a small number of base kernels.  
The set of models included in this space includes many standard regression models.
We proposed a search procedure for this space of kernels which parallels the process of scientific discovery.

We found that the learned structures are often capable of accurate extrapolation in complex time-series datasets, and are competitive with widely used kernel classes and kernel combination methods on a variety of prediction tasks.
The learned kernels often yield decompositions of a signal into diverse and interpretable components, enabling model-checking by humans.  %provides an additional degree of reassurance that the learned structure reflects the world.
We believe that a data-driven approach to choosing kernel structures automatically can help make nonparametric regression and classification methods accessible to non-experts.




%\paragraph{Source Code}
%Source code to perform all experiments is available on github\footnotemark.
\footnote{
%\url{http://www.github.com/jamesrobertlloyd/gpss-research}. 
All \gp{} parameter optimisation was performed by automated calls to the GPML toolbox available at \url{http://www.gaussianprocess.org/gpml/code/}.}



\outbpdocument{
\bibliographystyle{plainnat}
\bibliography{references.bib}
}


