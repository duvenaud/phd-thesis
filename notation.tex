%
% A header that lets you compile a chapter by itself, or inside a larger document.
% Adapted from http://stackoverflow.com/questions/3655454/conditional-import-in-latex
%
%
%Use \inbpdocument and \outbpdocument in your individual files, in place of \begin{document} and \end{document}. In your main file, put in a \def \ismaindoc {} before including or importing anything.
%
% David Duvenaud
% June 2011
% 
% ======================================
%
%


\ifx\ismaindoc\undefined
	\newcommand{\inbpdocument}{
		\def \ismaindoc {}
		% Use this header if we are compiling by ourselves.
		\documentclass[a4paper,12pt,authoryear,index]{common/PhDThesisPSnPDF}
		\input{common/official-preamble.tex}
		% All my custom preamble stuff.  Shouldn't overlap with anything in official-preamble

\usepackage{nth}
\usepackage{rotating}
\usepackage{array}
%\usepackage{gantt}
\usepackage[hyperpageref]{backref}

		% ************************ Thesis Information & Meta-data **********************

%% The title of the thesis
%\title{Structured Gaussian Process Models} 
%\title{Automatic Model Construction \\ through \\ Structured Gaussian Processes}
%\title{Automatic Model-Building \\ through \\ Structured Gaussian Processes}
%\title{Automatic Modeling \\ with \\ Structured Gaussian Processes}    
\title{Automatic Model Construction \\ with Gaussian Processes}
%\title{Automatic Model Construction}
%\title{Automating Statistical Model Construction}


%\texorpdfstring is used for PDF metadata. Usage:
%\texorpdfstring{LaTeX_Version}{PDF Version (non-latex)} eg.,
%\texorpdfstring{$sigma$}{sigma}

%% The full name of the author
\author{David Kristjanson Duvenaud}

%% Department (eg. Department of Engineering, Maths, Physics)
\dept{Department of Engineering}

%% University and Crest
\university{University of Cambridge}
\crest{\includegraphics[width=0.25\textwidth]{University_Crest}}

%% You can redefine the submission text:
% Default as per the University guidelines: This dissertation is submitted for
% the degree of Doctor of Philosophy
%\renewcommand{\submissiontext}{change the default text here if needed}

%% Full title of the Degree 
\degree{Doctor of Philosophy}
 
%% College affiliation (optional)
\college{Pembroke College}

%% Submission date
\degreedate{June 2014} 

%% Meta information
\subject{LaTeX} \keywords{{LaTeX} {PhD Thesis} {Engineering} {University of
Cambridge}}



		\begin{document}
	}	
	\newcommand{\outbpdocument}[1]{
		%\bibliographystyle{common/CUEDthesis}
		\bibliographystyle{plainnat}
		\bibliography{references.bib}
		\end{document}
	}	
\else
	%If we're inside another document, no need to re-start the document.
	\ifx\inbpdocument\undefined
		\newcommand{\inbpdocument}{}
		\newcommand{\outbpdocument}[1]{}
	\fi
\fi

\inbpdocument

\chapter*{Notation}
\label{ch:notation}
\addcontentsline{toc}{chapter}{Notation}

%Throughout this thesis we use Roman letters in place of greek letters wherever possible.

Unbolded $x$ represents a real number, $\vx$ represents a vector, and $\vX$ represents a matrix.
The $i$th element of a vector $\vx$ is denoted as $x_i$.
A bold lower-case number with an index such as $\vx_j$ represents a particular row of matrix $\vX$.

\vspace{1cm}

\begin{tabular}{lm{12cm}}
Symbol \quad     & Description \\
\hline
$\feat$       & The implicit feature vector corresponding to a kernel. \\
$\mathcal{O}(\cdot)$ & The big-O asymptotic complexity of an algorithm. \\
$A \otimes B$ & The Kronecker product of matrices $A$ and $B$. \\
$\vecf$ & A function represented as an infinite-dimensional vector. \\
$\kSE$ & Squared-exponential kernel, also known as the radial-basis function kernel, or Gaussian kernel. \\
$\kRQ$ & Rational-quadratic kernel. \\
$\kPer$ & Periodic kernel. \\
$\kLin$ & Linear kernel. \\
$\kWN$ & White noise kernel. \\
$\kC$ & constant kernel. \\
$k_1 + k_2$ & Addition of kernels, shorthand for: $k_1(x,x') + k_2(x,x')$ \\
$k_1 \times k_2$& Multiplication of kernels, shorthand for: $k_1(x,x') \times k_2(x,x')$ \\
\end{tabular}

\outbpdocument{
}


