%
% A header that lets you compile a chapter by itself, or inside a larger document.
% Adapted from http://stackoverflow.com/questions/3655454/conditional-import-in-latex
%
%
%Use \inbpdocument and \outbpdocument in your individual files, in place of \begin{document} and \end{document}. In your main file, put in a \def \ismaindoc {} before including or importing anything.
%
% David Duvenaud
% June 2011
% 
% ======================================
%
%


\ifx\ismaindoc\undefined
	\newcommand{\inbpdocument}{
		\def \ismaindoc {}
		% Use this header if we are compiling by ourselves.
		\documentclass[a4paper,12pt,authoryear,index]{common/PhDThesisPSnPDF}
		\input{common/official-preamble.tex}
		% All my custom preamble stuff.  Shouldn't overlap with anything in official-preamble

\usepackage{nth}
\usepackage{rotating}
\usepackage{array}
%\usepackage{gantt}
\usepackage[hyperpageref]{backref}

		% ************************ Thesis Information & Meta-data **********************

%% The title of the thesis
%\title{Structured Gaussian Process Models} 
%\title{Automatic Model Construction \\ through \\ Structured Gaussian Processes}
%\title{Automatic Model-Building \\ through \\ Structured Gaussian Processes}
%\title{Automatic Modeling \\ with \\ Structured Gaussian Processes}    
\title{Automatic Model Construction \\ with Gaussian Processes}
%\title{Automatic Model Construction}
%\title{Automating Statistical Model Construction}


%\texorpdfstring is used for PDF metadata. Usage:
%\texorpdfstring{LaTeX_Version}{PDF Version (non-latex)} eg.,
%\texorpdfstring{$sigma$}{sigma}

%% The full name of the author
\author{David Kristjanson Duvenaud}

%% Department (eg. Department of Engineering, Maths, Physics)
\dept{Department of Engineering}

%% University and Crest
\university{University of Cambridge}
\crest{\includegraphics[width=0.25\textwidth]{University_Crest}}

%% You can redefine the submission text:
% Default as per the University guidelines: This dissertation is submitted for
% the degree of Doctor of Philosophy
%\renewcommand{\submissiontext}{change the default text here if needed}

%% Full title of the Degree 
\degree{Doctor of Philosophy}
 
%% College affiliation (optional)
\college{Pembroke College}

%% Submission date
\degreedate{June 2014} 

%% Meta information
\subject{LaTeX} \keywords{{LaTeX} {PhD Thesis} {Engineering} {University of
Cambridge}}



		\begin{document}
	}	
	\newcommand{\outbpdocument}[1]{
		%\bibliographystyle{common/CUEDthesis}
		\bibliographystyle{plainnat}
		\bibliography{references.bib}
		\end{document}
	}	
\else
	%If we're inside another document, no need to re-start the document.
	\ifx\inbpdocument\undefined
		\newcommand{\inbpdocument}{}
		\newcommand{\outbpdocument}[1]{}
	\fi
\fi

\inbpdocument


\chapter{Deep Gaussian Processes}
\label{ch:deep-limits}

\begin{quotation}
``I asked myself: On any given day, would I rather be wrestling with a sampler, or proving theorems?''

\hspace*{\fill} \emph{Peter Orbanz}, personal communication
\end{quotation}




Choosing appropriate architectures and regularization strategies of deep networks is important for good predictive performance.
In this chapter, we propose to study this problem by viewing deep nets as priors on functions.
By viewing neural networks this way, we can analyze their properties without reference to any particular dataset, loss function, or training method.
Instead, we can ask what sorts of information-processing structures these priors give rise to, and check whether those structures are the same sort which we expect to find in useful models.

%To shed light on this problem, we analyze the analogous problem of constructing useful priors on compositions of functions.
As a starting point, we'll relate neural networks to Gaussian processes, and examine a class of infinitely-wide, deep neural networks called \emph{deep Gaussian processes}.
Deep \gp{}s are an attractive model class to study for several reasons.
\citet{damianou2012deep} showed that the probabilistic nature of deep \gp{}s gaurds against overfitting, and allows us to use the marginal likelihood to automatically tune the model architecture without the need for cross-validation.
\citet{hensman2014deep} showed that stochastic variational inference is possible in deep \gp{}s.
Together, these results suggest that deep \gp{}s are a promising alternative to neural nets.
For the analysis in this chapter, deep \gp{}s are attractive because they abstract away some of the details of finite neural networks. %, such as the number of hidden units

Our analysis will show that in standard architectures, the representational capacity of standard deep networks tends to decrease as the number of layers increases, retaining only a single degree of freedom in the limit.
We'll propose an alternate network architecture -- one which connects the input to each layer -- which does not suffer from this pathology.
We'll also examine \emph{deep kernels}, obtained by composing infinitely-many fixed feature transforms.

The ideas contained in this chapter were developed through discussions with Oren Rippel, Ryan Adams and Zoubin Ghahramani, and appear in \citep{DuvRipAdaGha14}.




\section{Relating Neural Networks to Deep Gaussian Processes}
\label{sec:relating}

Deep \gp{}s are simply priors on compositions of vector-valued functions, where each function in each layer is drawn independently from a GP prior:
%
\begin{align}
\vf^{(1:L)}(\vx) = \vf^{(L)}(\vf^{(L-1)}(\dots \vf^{(2)}(\vf^{(1)}(\vx)) \dots)) \\
%\nonumber\\
\textnormal{with each} \quad \vf_d^{(\layerindex)}  \simind \gp{} \left( 0, k^\layerindex_d(\vx, \vx') \right) \nonumber
\label{eq:deep-gp}
\end{align}
%

A multilayer neural network also implements a composition of vector-valued functions, one per layer.
%Therefore, understanding properties of function compositions helps us gain insight into deep networks.
In this section, we'll show the precise relationship between Gaussian processes and neural nets, then show two equivalent ways of constructing neural networks which give rise to deep \gp{}s.



%\subsection{A neural net with one hidden layer}
%\vspace{-0.05in}
\subsection{Single-layer Models}

%To mmake the link between deep \gp{}s and deep neural nets more explicit, we'll first
First, we'll relate neural networks with one hidden layer to Gaussian processes, using the
standard neural network architecture known as the multi-layer perceptrion (\MLP{}).
In the typical definition of an \MLP{}, the hidden units of the first layer are defined as:
%
\begin{align}
\vh^{(1)}(\vx) = \sigma \left( \vb^{(1)} + \vW^{(1)}\vx \right)
\end{align}
%
where $\vh$ are the hidden unit activations, $\vb$ is a bias vector, $\vW$ is a weight matrix and $\sigma$ is a one-dimensional nonlinear function applied element-wise. The output vector $f(\vx)$ is simply a weighted sum of these hidden unit activations:
%
\begin{align}
f(\vx) = \vV^{(1)} \sigma \left( \vb^{(1)} + \vW^{(1)} \vx \right)  = \vV^{(1)} \vh^{(1)}(\vx)
\label{eq:one-layer-nn}
\end{align}
%
where $\vV^{(1)}$ is another weight matrix.

%There exists a correspondence between one-layer \MLP{}s and \gp{}s 
\citet{neal1995bayesian} showed that \gp{}s can be viewed as a prior on neural networks with infinitely many hidden units and unknown weights.
More precisely, for any model of the form
%
\begin{align}
f(\vx) = \frac{1}{K}{\mathbf \vnetweights}\tra \hPhi(\vx) = \frac{1}{K} \sum_{i=1}^K \netweights_i \hphi_i(\vx),
\label{eq:one-layer-gp}
\end{align}
%
with fixed features $\left[ \hphi_1(\vx), \dots, \hphi_K(\vx) \right]\tra = \hPhi(\vx)$ and i.i.d. $\netweights$'s with zero mean and finite variance $\sigma^2$, the central limit theorem implies that as the number of features $K$ grows, any two function values $f(\vx), f(\vx')$ have a joint distribution approaching a Gaussian:
%
\begin{align}
\lim_{K \to \infty} p\left( \colvec{f(\vx)}{f(\vx')} \right) = \Nt{\colvec{0}{0}}{
\frac{\sigma^2}{K} \left[ \begin{array}{cc}
\sum_{i=1}^K \hphi_i(\vx)\hphi_i(\vx) &
\sum_{i=1}^K \hphi_i(\vx)\hphi_i(\vx') \\
\sum_{i=1}^K \hphi_i(\vx')\hphi_i(\vx) &
\sum_{i=1}^K \hphi_i(\vx')\hphi_i(\vx')
\end{array} \right] }
\end{align}
A joint Gaussian distribution between any set of function values is the definition of a Gaussian process.
%\TBD{DD: any two or any set?}

The result is surprisingly general:
it puts no constraints on the features (other than having uniformly bounded activation), nor does it require that the feature weights $\vnetweights$ be Gaussian distributed.  

We can also work backwards to derive a one-layer \MLP{} from any \gp{}.
Mercer's theorem implies that any positive-definite kernel function corresponds to an inner product of features: $k(\vx, \vx') = \hPhi(\vx) \tra \hPhi(\vx')$.
%
Thus in the one-hidden-layer case, the correspondence between \MLP{}s and \gp{}s is simple:
the features $\hPhi(\vx)$ of the kernel correspond to the hidden units of the \MLP{}.




% For tikz figures in deep limits
\newcommand{\numdims}[0]{3}
\newcommand{\numhidden}[0]{3}
\newcommand{\upnodedist}[0]{1cm}
\newcommand{\bardist}[0]{\hspace{-0.2cm}}

\def\layersep{2.3cm}
\def\nodesep{1.3cm}
\def\nodesize{1cm}


\newcommand{\neuronfunc}[2]{
\FPeval{\result}{clip(#1+#2)}
\includegraphics[width=1cm, clip, trim=0mm 0mm 0mm 0mm]{../figures/deep-limits/two-d-draws/sqexp-draw-\result}
}

\tikzstyle{input neuron}=[neuron, fill=green!15]
\tikzstyle{output neuron}=[neuron, fill=red!15]
\tikzstyle{hidden neuron}=[neuron, fill=blue!15]

\newcommand{\indfeat}{h}
%\newcommand{\indfeat}{\phi}

\begin{figure}[t]
\begin{tabular}{c|c}
Neural net corresponding to a \gp{} & Net corresponding to a \gp{} with a deep kernel \\
\\
\null\hspace{-0.25cm}
\begin{tikzpicture}[shorten >=1pt,->,draw=black!50, node distance=\layersep]
    \tikzstyle{every pin edge}=[<-,shorten <=1pt]
    \tikzstyle{neuron}=[circle,fill=black!25,minimum size=17pt,inner sep=0pt]
    \tikzstyle{annot} = [text width=4em, text centered]

    % Draw the input layer nodes
    \foreach \name / \y in {1,...,\numdims}
    % This is the same as writing \foreach \name / \y in {1/1,2/2,3/3,4/4}
        \node[input neuron, minimum size=\nodesize
        %, pin=left:Input \#\y
        ] (I-\name) at (0,-\nodesep*\y) {$x_\y$};

    % Draw the hidden layer nodes
    % Draw the hidden layer nodes
    \foreach \name / \y in {1,2}
        \path[yshift=0.5cm]
            node[hidden neuron, minimum size=\nodesize] (H-\name) at (\layersep,-\nodesep*\y) {$\indfeat_{\y}$};
    
	\foreach \name / \y in {3}
	    \path[yshift=0.5cm]
    	    node[hidden neuron, minimum size=\nodesize] (H-\name) at (\layersep,-\nodesep*4) {$\indfeat_K$};

    % Draw the output layer node
    \foreach \name / \y in {1,...,\numdims}
    	\node[output neuron, minimum size=\nodesize] (O-\name) at (2*\layersep,-\nodesep*\y) {$f_{\y}$};

    % Connect every node in the input layer with every node in the hidden layer.
    \foreach \source in {1,...,\numdims}
        \foreach \dest in {1,...,\numhidden}
            \path (I-\source) edge (H-\dest);

    % Connect every node in the hidden layer with the output layer
    \foreach \source in {1,...,\numhidden}
        \foreach \dest in {1,...,\numdims}
    	    \path (H-\source) edge (O-\dest);

    % Annotate the layers
    \node[annot,below of=H-2, node distance=1.15cm] {$\vdots$};
    \node[annot,above of=I-1, node distance=\upnodedist] {Inputs};
    \node[annot,above of=H-1, node distance=\upnodedist] {Fixed};
    \node[annot,above of=O-1, node distance=\upnodedist] {Random};
\end{tikzpicture}
&
\begin{tikzpicture}[shorten >=1pt,->,draw=black!50, node distance=\layersep]
    \tikzstyle{every pin edge}=[<-,shorten <=1pt]
    \tikzstyle{neuron}=[circle,fill=black!25,minimum size=17pt,inner sep=0pt]
    \tikzstyle{annot} = [text width=4em, text centered]

    % Draw the input layer nodes
    \foreach \name / \y in {1,...,\numdims}
    % This is the same as writing \foreach \name / \y in {1/1,2/2,3/3,4/4}
        \node[input neuron, minimum size=\nodesize
        %, pin=left:Input \#\y
        ] (I-\name) at (0,-\nodesep*\y) {$x_\y$};

    % Draw the first hidden layer nodes
    \foreach \name / \y in {1,2}%...,\numhidden}
        \path[yshift=0.5cm]
            node[hidden neuron, minimum size=\nodesize] (H-\name) at (\layersep,-\nodesep*\y) {$\indfeat^{(1)}_{\y}$};
	\foreach \name / \y in {3}
	    \path[yshift=0.5cm]
    	    node[hidden neuron, minimum size=\nodesize] (H-\name) at (\layersep,-\nodesep*4) {$\indfeat^{(1)}_K$};
            

    % Draw the secdond hidden layer nodes
    \foreach \name / \y in {1,2}%...,\numhidden}
        \path[yshift=0.5cm]
            node[hidden neuron, minimum size=\nodesize] (H2-\name) at (2*\layersep,-\nodesep*\y) {$\indfeat^{(2)}_{\y}$};
	\foreach \name / \y in {3}
	    \path[yshift=0.5cm]
    	    node[hidden neuron, minimum size=\nodesize] (H2-\name) at (2*\layersep,-\nodesep*4) {$\indfeat^{(2)}_K$};

    % Draw the output layer node
    \foreach \name / \y in {1,...,\numdims}
    	\node[output neuron, minimum size=\nodesize
    	%,pin={[pin edge={->}]right:Output }
    	] (O-\name) at (3*\layersep,-\nodesep*\y) {$f_{\y}$};

    % Connect every node in the input layer with every node in the
    % hidden layer.
    \foreach \source in {1,...,\numdims}
        \foreach \dest in {1,...,\numhidden}
            \path (I-\source) edge (H-\dest);
            
    \foreach \source in {1,...,\numhidden}
        \foreach \dest in {1,...,\numhidden}
            \path (H-\source) edge (H2-\dest);            

    % Connect every node in the hidden layer with the output layer
    \foreach \source in {1,...,\numhidden}
        \foreach \dest in {1,...,\numdims}
    	    \path (H2-\source) edge (O-\dest);

    % Annotate the layers
    \node[annot,above of=I-1, node distance=\upnodedist] {Inputs};
    \node[annot,below of=H-2, node distance=1.15cm] {$\vdots$};    
    \node[annot,above of=H-1, node distance=\upnodedist] {Fixed};
    \node[annot,below of=H2-2, node distance=1.15cm] {$\vdots$};
    \node[annot,above of=H2-1, node distance=\upnodedist] {Fixed};
    \node[annot,above of=O-1, node distance=\upnodedist] {Random};
\end{tikzpicture}
\end{tabular}
\caption[Neural network architectures giving rise to \sgp{}s]
{
\emph{Left:} \gp{}s can be understood as a one-hidden-layer \MLP{} with infinitely many fixed hidden units having unknown weights.
\emph{Right:} Multiple layers of fixed hidden units gives rise to a \gp{} with a deep kernel, but not a deep \gp{}.
}
\label{fig:gp-architectures}
\end{figure}





\subsection{Multiple Hidden Layers}

%\paragraph{Fixed nonlinearities}
In an \MLP{} with multiple hidden layers, activation of the $\layerindex$th layer units are given by
% the recurrence
%
\begin{align}
\vh^{(\layerindex)}(\vx) = \sigma \left( \vb^{(\layerindex)} + \vW^{(\layerindex)} \vh^{(\layerindex-1)}(\vx) \right) \;.
\label{eq:nextlayer}
\end{align}
This architecture is shown on the right of \cref{fig:gp-architectures}.
%In this model, each hidden layer's output feeds directly into the next layer's input, weighted by the corresponding element of $\vW^{(\layerindex)}$.  
%
For example, if we extend the model given by \cref{eq:one-layer-nn} to have two layers of feature mappings,  the resulting model would become
%
\begin{align}
f(\vx) = \frac{1}{K}{\mathbf \vnetweights}\tra \hPhi^{(2)}\left( \hPhi^{(1)}(\vx) \right) \;.
% = \frac{1}{K} \sum_{i=1}^K \netweights_i \hphi_i(\vx),
\label{eq:mutli-layer-nn}
\end{align}

If the features $\vh^{(1)}(\vx)$ and $\vh^{(2)}(\vx)$ are considered fixed with only the last-layer weights ${\vnetweights}$ unknown, this model corresponds to a (non-deep) \gp{} with a ``deep kernel'', given by
\begin{align}
k(\vx, \vx') = \left( \hPhi^{(2)} ( \hPhi^{(1)}(\vx) ) \right) \tra \hPhi^{(2)} (\hPhi^{(1)}(\vx') )
\end{align}
\gp{}s with deep kernels, explored in \cref{sec:deep_kernels}, imply a fixed representation as opposed to a prior over representations.
Thus, unless we richly parameterize these kernels, their capacity to learn an appropriate representation is limited in comparison to more flexible models such as deep neural networks, or deep \gp{}s. %since only the output weights ${\vnetweights}$ can adapt to the problem  which is what we wish to analyze in this chapter.


%In a neural net with multiple hidden layers, the correspondence is a little more complicated.
 
\subsection{Two Network Architectures Equivalent to Deep \sgp{}s}

\def\halfshift{0.0cm}

\begin{figure}[t!]
\centering
\begin{tabular}{c}
%\multicolumn{2}{c}{} \\
%\multicolumn{2}{c}{
A neural net with fixed activation functions corresponding to a 3-layer deep \gp{}\\ %$\vy = \vf^{(2)} \left( \vf^{(1)}(\vx) \right)$ \\
\\
%\multicolumn{2}{c}{
\begin{tikzpicture}[shorten >=1pt,->,draw=black!50, node distance=\layersep]
    \tikzstyle{every pin edge}=[<-,shorten <=1pt]
    \tikzstyle{neuron}=[circle,fill=black!25,minimum size=17pt,inner sep=0pt]
    \tikzstyle{annot} = [text width=4em, text centered]

    % Draw the input layer nodes
    \foreach \name / \y in {1,...,\numdims}
    % This is the same as writing \foreach \name / \y in {1/1,2/2,3/3,4/4}
        \node[input neuron, minimum size=\nodesize
        %, pin=left:Input \#\y
        ] (I-\name) at (0,-\nodesep*\y) {$x_\y$};

    % Draw the hidden layer nodes
    \foreach \name / \y in {1,2}%...,\numhidden}
        \path[yshift=0.5cm]
            node[hidden neuron, minimum size=\nodesize] (H-\name) at (\layersep,-\nodesep*\y) {$\indfeat^{(1)}_{\y}$};
   	\foreach \name / \y in {3}
	    \path[yshift=0.5cm]
    	    node[hidden neuron, minimum size=\nodesize] (H-\name) at (\layersep,-\nodesep*4) {$\indfeat^{(1)}_K$};


    % Draw the hidden layer nodes
    \foreach \name / \y in {1,2}%...,\numhidden}
        \path[yshift=0.5cm]
            node[hidden neuron, minimum size=\nodesize] (H2-\name) at (3*\layersep,-\nodesep*\y) {$\indfeat^{(2)}_{\y}$};
   	\foreach \name / \y in {3}
	    \path[yshift=0.5cm]
    	    node[hidden neuron, minimum size=\nodesize] (H2-\name) at (3*\layersep,-\nodesep*4) {$\indfeat^{(2)}_K$};
    	    
    % Draw the hidden layer nodes
    \foreach \name / \y in {1,2}%...,\numhidden}
        \path[yshift=0.5cm]
            node[hidden neuron, minimum size=\nodesize] (H3-\name) at (5*\layersep,-\nodesep*\y) {$\indfeat^{(3)}_{\y}$};
   	\foreach \name / \y in {3}
	    \path[yshift=0.5cm]
    	    node[hidden neuron, minimum size=\nodesize] (H3-\name) at (5*\layersep,-\nodesep*4) {$\indfeat^{(3)}_K$};    	    

    % Draw the output layer node
    \foreach \name / \y in {1,...,\numdims}
    	\node[output neuron, minimum size=\nodesize
    	%,pin={[pin edge={->}]right:Output }
    	] (O1-\name) at (2*\layersep,-\nodesep*\y) {$f^{(1)}_{\y}$};

    % Draw the output layer node
    \foreach \name / \y in {1,...,\numdims}
    	\node[output neuron, minimum size=\nodesize
    	%,pin={[pin edge={->}]right:Output }
    	] (O2-\name) at (4*\layersep,-\nodesep*\y) {$f^{(2)}_{\y}$};
    	
    % Draw the output layer node
    \foreach \name / \y in {1,...,\numdims}
    	\node[output neuron, minimum size=\nodesize
    	%,pin={[pin edge={->}]right:Output }
    	] (O3-\name) at (6*\layersep,-\nodesep*\y) {$f^{(3)}_{\y}$};    	

    % Connect every node in the input layer with every node in the
    % hidden layer.
    \foreach \source in {1,...,\numdims}
        \foreach \dest in {1,...,\numhidden}
            \path (I-\source) edge (H-\dest);
            
    \foreach \source in {1,...,\numhidden}
        \foreach \dest in {1,...,\numdims}
            \path (H-\source) edge (O1-\dest);         
            
    \foreach \source in {1,...,\numdims}
        \foreach \dest in {1,...,\numhidden}
            \path (O1-\source) edge (H2-\dest);        
            
    \foreach \source in {1,...,\numdims}
        \foreach \dest in {1,...,\numhidden}
            \path (O2-\source) edge (H3-\dest);                      
            
    \foreach \source in {1,...,\numhidden}
        \foreach \dest in {1,...,\numdims}
    	    \path (H3-\source) edge (O3-\dest);            

    % Connect every node in the hidden layer with the output layer
    \foreach \source in {1,...,\numhidden}
        \foreach \dest in {1,...,\numdims}
    	    \path (H2-\source) edge (O2-\dest);

    % Annotate the layers
    \node[annot,above of=I-1, node distance=\upnodedist] {Inputs};
    \node[annot,below of=I-3, node distance=\upnodedist] {$\vx$};
    \node[annot,above of=H-1, node distance=\upnodedist] {Fixed};
    \node[annot,below of=O1-3, node distance=\upnodedist] {$\vf^{(1)}(\vx)$};
    \node[annot,above of=O1-1, node distance=\upnodedist] {Random};
    \node[annot,above of=H2-1, node distance=\upnodedist] {Fixed};
    \node[annot,below of=O2-3, node distance=\upnodedist] {$\vf^{(1:2)}(\vx)$};
    \node[annot,above of=O2-1, node distance=\upnodedist] {Random};
    \node[annot,above of=H3-1, node distance=\upnodedist] {Fixed};
    \node[annot,above of=O3-1, node distance=\upnodedist] {Random};
    \node[annot,below of=O3-3, node distance=\upnodedist] {$\vy$};
    \node[annot,below of=H-2, node distance=1.15cm] {$\vdots$};    
    \node[annot,below of=H2-2, node distance=1.15cm] {$\vdots$};    
    \node[annot,below of=H3-2, node distance=1.15cm] {$\vdots$}; 
\end{tikzpicture}
%}
\\
\hline
\\
A net with nonpararametric activation functions corresponding to a 3-layer deep \gp{}\\
\\
%\begin{tabular}{c}
\begin{tikzpicture}[shorten >=1pt,->,draw=black!50, node distance=\layersep]
%    \tikzstyle{interarrows}=[->, thick, black!50]
    \tikzstyle{every pin edge}=[<-,shorten <=1pt]
%    \tikzstyle{input neuron}=[neuron, minimum size=\nodesize];
    \tikzstyle{neuron}=[circle, draw = black, inner sep=0pt, line width = 1pt]
    \tikzstyle{input neuron}=[circle, minimum size=\nodesize, fill=green!15]
    \tikzstyle{output neuron}=[neuron, minimum size=\nodesize, text width = 1cm];
    \tikzstyle{hidden neuron}=[neuron, minimum size=\nodesize, text width = 1cm];
    \tikzstyle{annot} = [text width=4em, text centered]

    % Draw the input layer nodes
    \foreach \name / \y in {1,...,\numdims}
        \node[input neuron] (I-\name) at (0,-\nodesep*\y) {$x_\y$};

    % Draw the hidden layer nodes
    \foreach \name / \y in {1,...,\numhidden}
        \path[yshift=\halfshift] node[hidden neuron] (H-\name) at (\layersep,-\nodesep*\y) { \neuronfunc{\y}{0}};

    % Draw the hidden layer nodes
    \foreach \name / \y in {1,...,\numhidden}
        \path[yshift=\halfshift] node[hidden neuron] (H2-\name) at (2*\layersep,-\nodesep*\y) {\neuronfunc{\y}{4}};

    % Draw the output layer node
    \foreach \name / \y in {1,...,\numdims}
    	\node[output neuron] (O-\name) at (3*\layersep,-\nodesep*\y) {\neuronfunc{\y}{8}};

    % Connect every node in the input layer with every node in the hidden layer.
    \foreach \source in {1,...,\numdims}
        \foreach \dest in {1,...,\numhidden}
            \path (I-\source) edge (H-\dest);
            
    \foreach \source in {1,...,\numhidden}
        \foreach \dest in {1,...,\numhidden}
            \path (H-\source) edge (H2-\dest);            

    % Connect every node in the hidden layer with the output layer
    \foreach \source in {1,...,\numhidden}
        \foreach \dest in {1,...,\numdims}
    	    \path (H2-\source) edge (O-\dest);
    	    %arrows={-angle 90}

    % Annotate the layers
    \node[annot,above of=I-1, node distance=\upnodedist] {Inputs};
    \node[annot,below of=I-\numdims, node distance=\upnodedist] {$\vx$};    
    \node[annot,above of=H-1, node distance=\upnodedist, text width = 2cm] {\gp{}};
    \node[annot,above of=H2-1, node distance=\upnodedist, text width = 2cm] {\gp{}};
    \node[annot,below of=H-\numhidden, node distance=\upnodedist, text width = 2cm] {$\vf^{(1)}(\vx)$};
    \node[annot,below of=H2-\numhidden, node distance=\upnodedist, text width = 2cm] {$\vf^{(2)}(\vf^{(1)}(\vx))$};
    \node[annot,above of=O-1, node distance=\upnodedist] {\gp{}};
    \node[annot,below of=O-\numdims, node distance=\upnodedist, text width = 1cm] {$\vy$};
\end{tikzpicture}
%\end{tabular}
\end{tabular}
\caption[Neural network architectures giving rise to deep \sgp{}s]
{
Two equivalent views of deep \gp{}s as neural networks.
\emph{Top:} A neural network whose every second layer is a weighted sum of an infinite number of fixed hidden units, where the weights are initially unknown.
\emph{Bottom:} A neural network with a finite number of hidden units, each with a different unknown non-parametric activation function.
}
\label{fig:deep-gp-architectures}
\end{figure}

There are two equivalent neural net architectures which correspond to deep \gp{}s: one with fixed non-linearities, and another with \gp{}-distributed nonlinearties.

To construct a neural network with fixed nonlinearities that corresponds to a deep \gp{}, 
%with $D_\layerindex$ units for each layer corresponding to the outputs compositions of functions drawn from a \gp{} prior, 
one can start with the infinitely-wide deep \gp{} shown in \cref{fig:gp-architectures}(right), and introduce a finite set of nodes in between each infinitely-wide set of fixed basis functions.
This architecture is shown in the top of \cref{fig:deep-gp-architectures}.
The $D^{(\layerindex)}$ outputs $\vf^{(\layerindex)}(\vx)$ in between each layer are weighted sums (with random weights) of the fixed hidden units of the layer below, and the next layer's hidden units depend only on these $D^{(\layerindex)}$ outputs.
%Thus, the network architecture with fixed nonlinearities corresponding to deep \gp{}s has an extra set of layers that a \MLP{} doesn't have.
%

This alternating-layer architecture has an interpretation as a series of linear information bottlenecks.
To see this, we can simply substitute \cref{eq:one-layer-nn} into \cref{eq:nextlayer} to get
%
\begin{align}
\hPhi^{(\layerindex)}(\vx) = \sigma \left( \vb^{(\layerindex)} + \vW^{(\layerindex)} \vV^{(\layerindex-1)} \hPhi^{(\layerindex-1)}(\vx) \right) \;.
\end{align}
%
Thus, ignoring the intermediate outputs $\vf^{(\layerindex)}(\vx)$, a deep \gp{} is an infinitely-wide, deep \MLP{} with each pair of layers connected by random, rank-$D_\layerindex$ matrices $\vW^{(\layerindex)} \vV^{(\layerindex-1)}$. %connecting the layers.



%There are two ways to relate deep \gp{}s to \MLP{}s.
The second, more direct way to construct a network architecture corresponding to a deep \gp{} is to integrate out all $\vV^{(\layerindex)}$, and view deep \gp{}s as a neural network with a finite number of nonparametric, \gp{}-distributed basis functions at each layer, in which $\vf^{(1:\layerindex)}(\vx)$ represent the output of the hidden nodes at the $\layerindex^{th}$ layer.
This second view lets us compare deep \gp{} models to standard neural net architectures more directly.
\Cref{fig:gp-architectures}(bottom) shows this architecture.
%, examining the activations and shapes of the finite number of basis functions at each layer.


%Figure \ref{fig:architectures} compares these two architectures.  If we stick to the correspondence between neural nets and \gp{}s as defined in \eqref{eq:one-layer-nn}, we notice that the deep \gp{} architecture forces the linear connections between hidden layers go be squeezed through a finite set of nodes, corresponding to the output of the independent \gp{}s at each layer.  




%\subsection{Relating initialization strategies, regularization, and priors}

%Traditional training of deep networks requires both an an initialization strategy, and a regularization method.  In a Bayesian model, the prior implicitly defines both of these things.

%For example, one feature of \gp{} models with a clear analogue to initialization strategies in deep networks is automatic relevance determination. 
%\paragraph{The effects of automatic relevance determination}  One feature of Gaussian process models that doesn't have a clear anologue in the neural-net literature is automatic relevance determination (ARD).  Recall that 
%In the ARD procedure, the lengthscales of each dimension are scaled to maximize the marginal likelihood of the GP.  In standard one-layer GP regression, it has the effect of smoothing out the posterior over functions in directions in which the data does not change.

%\paragraph{Sparse Initialization}
%\cite{martens2010deep} used “sparse initialization”, in which each hidden unit had 15 non-zero incoming connections.  %They say ``this allows the units to be both highly differentiated as well as unsaturated, avoiding the problem in dense initializations where the connection weights must all be scaled very small in order to prevent saturation, leading to poor differentiation between units''.
%
%While it is not clear if deep GPs have analogous problems with saturation of connection weights, we note that 
%An anologous initialization strategy would be to put a heavy-tailed prior on the lengthscales in the squared-exp kernel.


%\subsection{What kinds of functions do deep GPs prefer?}
%Isotropic kernels such as the squared-exp encode the assumption that the function varies independently in all directions.  
%This implies that using an isotropic kernel to model a function that does \emph{not} vary in all direction independently is 'wasting marginal likelihood' by not making strong enough assumptions.  
%
%Thus, deep GPs prefer to have hidden layers on whose outputs the higher-layer functions vary independently.  In this way, we can say that the deep GP model prefers independent features.

%Combining these two properties of isotropic kernels, we can say that a deep GP will attempt to transform its input into a representation with as few degrees of freedom as possible, and for which the output function varies independently across each dimension.




\section{Characterizing Deep Gaussian Processes}
\label{sec:characterizing-deep-gps}

In this section, we develop several theoretical results that explore the behavior of deep \gp{}s as a function of their depth.
This will allow us in \cref{sec:formalizing-pathology} to formally identify a pathology that emerges in very deep networks.

Specifically, we will show that the size of the derivative of a one-dimensional deep \gp{} becomes log-normal distributed as the network becomes deeper.
We'll also show that the Jacobian of a multivariate deep \gp{} is a product of independent Gaussian matrices with independent entries.


\subsection{One-dimensional Asymptotics}
\label{sec:1d}

%One way to understand the properties of functions drawn from deep \gp{}s and deep networks is by looking at the distribution of the derivative of these functions. We first focus on the one-dimensional case.

In this section, we derive the limiting distribution of the derivative of an arbitrarily deep, one-dimensional \gp{} with a squared-exp kernel:  %in order to shed light on the ways in which a deep \gp{}s are non-Gaussian, and the ways in which lower layers affect the computation performed by higher layers.
%
\newcommand{\onedsamplepic}[1]{
\hspace{-0.25in}
\includegraphics[trim=2mm 2mm 2mm 6.4mm, clip, width=0.255\columnwidth]{\deeplimitsfiguresdir/1d_samples/latent_seed_0_1d_large/layer-#1}} 
%
\newcommand{\onedsamplepiccon}[1]{
\hspace{-0.25in}
\includegraphics[trim=2mm 2mm 2mm 6.4mm, clip, width=0.255\columnwidth]{\deeplimitsfiguresdir/1d_samples/latent_seed_0_1d_large_connected/layer-#1}} 
%
\begin{figure}
\centering
\begin{tabular}{cccc}
\hspace{-0.1in} 1 layer & \hspace{-0.2in} 2 Layers & 5 Layers & \hspace{-0.25in} 10 Layers \\
\hspace{0.03in}
\onedsamplepic{1} &
\onedsamplepic{2} &
\onedsamplepic{5} &
\onedsamplepic{10}
%& 6 layers \\ & largest , for unconnected layers & largest, for connected layers
\end{tabular}
\caption[One-dimensional draws from a deep \sgp{} prior]
{One-dimensional draws from a deep gp prior.
After a few layers, the functions begin to be either nearly flat, or highly varying, everywhere.
This is a consequence of the distribution on derivatives becoming heavy-tailed.}
\label{fig:deep_draw_1d}
\end{figure}
%
\begin{align}
\kSE(x, x') = \sigma^2 \exp \left( \frac{-(x - x')^2}{2\lengthscale^2} \right) \;.
\label{eq:se_kernel}
\end{align}
%
The parameter $\sigma^2$ controls the variance of functions drawn from the prior, and the lengthscale parameter $\lengthscale$ controls the smoothness.  
The derivative of a \gp{} with a squared-exp kernel is point-wise distributed as $\Nt{0}{\nicefrac{\sigma^2}{\lengthscale^2}}$.  
Intuitively, a draw from a \gp{} is likely to have large derivatives if the kernel has high variance and small lengthscales.
 
 By the chain rule, the derivative of a one-dimensional deep \gp{} is simply a product of the derivatives of each layer, which are drawn independently.
 The distribution of the absolute value of this derivative is a product of half-normals, each with mean $\sqrt{\nicefrac{2 \sigma^2}{\pi \lengthscale^2}}$.
%
%\begin{align}
%\frac{\partial f(x)}{\partial x} & \distas{iid} \Nt{0}{\frac{\sigma^2_f}{\lengthscale^2}} \\
%\implies 
%\left| \frac{\partial f(x)}{\partial x} \right| & \distas{iid} \textnormal{half}\Nt{\sqrt{\frac{2 \sigma^2_f}{\pi\lengthscale^2}}}{\frac{\sigma^2_f}{\lengthscale^2} \left( 1 - \frac{2}{\pi} \right)}
%\end{align}
%
If we choose kernel parameters so that ${\nicefrac{\sigma^2}{\lengthscale^2} = \nicefrac{\pi}{2}}$, then the expected magnitude of the derivative remains constant regardless of the depth.

%then ${\expectargs{}{\left| \nicefrac{\partial f(x)}{\partial x} \right|} = 1}$, and so ${\expectargs{}{\left| \nicefrac{\partial f^{(1:L)}(x)}{\partial x} \right|} = 1}$ for any $L$.
%  If $\nicefrac{\sigma^2_f}{\lengthscale^2}$ is less than $\nicefrac{\pi}{2}$, the expected derivative magnitude goes to zero, and if it is greater, the expected magnitude goes to infinity as a function of $L$.  

The log of the magnitude of the derivatives has moments
\begin{align}
m_{\log} = \expectargs{}{\log \left| \frac{\partial f(x)}{\partial x} \right|} & = 2 \log \left( \frac{\sigma}{\lengthscale} \right) - \log2 - \gamma \nonumber \\
%\varianceargs{}{\log \left| \frac{\partial f(x)}{\partial x} \right|} & = \frac{1}{4}\left[ \pi^2 + 2 \log^2 2  - 2 \gamma( \gamma + log4 ) + 8 \left( \gamma + \log2 - \log \left(\frac{\sigma_f}{\lengthscale}\right)\right) \log\left(\frac{\sigma_f}{\lengthscale}\right) \right]
v_{\log} = \varianceargs{}{\log \left| \frac{\partial f(x)}{\partial x} \right|} & = \frac{\pi^2}{4} + \frac{\log^2 2}{2}  - \gamma^2 - \gamma \log4 + 2 \log\left(\frac{\sigma}{\lengthscale}\right) \left[ \gamma + \log2 - \log \left(\frac{\sigma}{\lengthscale}\right) \right]
\end{align}
where $\gamma \approxeq 0.5772$ is Euler's constant.  Since the second moment is finite, by the central limit theorem, the limiting distribution of the size of the gradient approaches log-normal as L grows:
\begin{align}
\log \left| \frac{\partial f^{(1:L)}(x)}{\partial x} \right| 
& = \sum_{\layerindex=1}^L \log \left| \frac{\partial f^{(\layerindex)}(x)}{\partial x} \right| 
% \implies
%\log \left| \frac{\partial f^{1:L}(x)}{\partial x} \right| & \distas{L \rightarrow \infty} \Nt{L \sqrt{\frac{2 \sigma^2_f}{\pi\lengthscale^2}}}{L \frac{\sigma^2_f}{\lengthscale^2} \left( 1 - \frac{2}{\pi} \right)}
%\log \left| \frac{\partial f^{(1:L)}(x)}{\partial x} \right| & 
\distas{L \rightarrow \infty} \Nt{ L m_{\log} }{L^2 v_{\log}}
\end{align}
%
Even if the expected magnitude of the derivative remains constant, the variance of the log-normal distribution grows without bound as the depth increases.

Because the log-normal distribution is heavy-tailed and its domain is bounded below by zero, the derivative will become very small almost everywhere, with rare but very large jumps.  
\Cref{fig:deep_draw_1d} shows this behavior in a draw from a 1D deep \gp{} prior, at varying depths.
This figure also shows that once the derivative in one region of the input space becomes very large or very small, it is likely to remain that way in subsequent layers.
%
%Does it convege to some sort of jump process?  
%Note that there is another limit - if $\nicefrac{\sigma^2_f}{\lengthscale_{d_1}^2} < \nicefrac{\pi}{2}$, then the mean of the size of the gradient goes to zero, but the variance approaches a finite constant.







%\section{The Jacobian of deep GP is a product of independent normal matrices}
\subsection{Distribution of the Jacobian}
\label{sec:theorem}

Next, we characterize the distribution on Jacobians of multivariate functions drawn from a deep \gp{} prior.

\begin{lemma}
\label{thm:deriv-ind}
The partial derivatives of a function mapping $\mathbb{R}^D \rightarrow \mathbb{R}$ drawn from a \gp{} prior with a product kernel are independently Gaussian distributed.
\end{lemma}
%
%\vspace{-0.25in}
\begin{proof}
Because differentiation is a linear operator, the derivatives of a function drawn from a \gp{} prior are also jointly Gaussian distributed.
The covariance between partial derivatives with respect to input dimensions $d_1$ and $d_2$ of vector $\vx$ are given by \citet{Solak03derivativeobservations}:
%
\begin{align}
\cov \left( \frac{\partial f(\vx)}{\partial x_{d_1}}, \frac{\partial f(\vx)}{\partial x_{d_2}} \right) 
= \frac{\partial^2 k(\vx, \vx')}{\partial x_{d_1} \partial x_{d_2}'} \bigg|_{\vx=\vx'}
\label{eq:deriv-kernel}
\end{align}
%
If our kernel is a product over individual dimensions $k(\vx, \vx') = \prod_d^D k_d(x_d, x_d')$, 
%as in the case of the squared-exp kernel, 
then the off-diagonal entries are zero, implying that all elements are independent.
\end{proof}

%\vspace{-0.1in}
For example, in the case of the multivariate squared-exp kernel, the covariance between derivatives has the form:
%
\begin{align}
%k_{SE}(\vx, \vx') = 
f(\vx) \sim \textnormal{GP}\left( 0, 
\sigma^2 \prod_{d=1}^D \exp \left(-\frac{1}{2} \frac{(x_d - x_d')^2}{\lengthscale_d^2} \right) \right) \nonumber \\
 \implies 
\cov \left( \frac{\partial f(\vx)}{\partial x_{d_1}}, \frac{\partial f(\vx)}{\partial x_{d_2}} \right) =
%\frac{\partial^2 k_{SE}(\vx, \vx')}{\partial x_{d_1} \partial x_{d_2}'} \bigg|_{\vx=\vx'} =
\begin{cases} 
\frac{\sigma^2}{\lengthscale_{d_1}^2} & \mbox{if } d_1 = d_2 \\ 
0 & \mbox{if } d_1 \neq d_2 \end{cases}
\end{align}


\begin{lemma}
\label{thm:matrix}
The Jacobian of a set of $D$ functions $\mathbb{R}^D \rightarrow \mathbb{R}$ drawn from independent \gp{} priors, each having product kernel is a $D \times D$ matrix of independent Gaussian R.V.'s
\end{lemma}
%
%\vspace{-0.2in}
\begin{proof}
The Jacobian of the vector-valued function $\vf(\vx)$ is a matrix $J$ with elements ${J_{ij} = \frac{ \partial \vf_i (\vx) }{\partial x_j}}$.
%
%\begin{align}
%J_{ij} = \dfrac{\partial f_i (\vx) }{\partial x_j}
%\end{align}
%\begin{align}
%\Jx^\layerindex(\vx) =\begin{bmatrix} \dfrac{\partial f^\layerindex_1 (\vx) }{\partial x_1} & \cdots & \dfrac{\partial f^\layerindex_1 (\vx)}{\partial x_D} \\ \vdots & \ddots & \vdots \\ \dfrac{\partial f^\layerindex_D (\vx)}{\partial x_1} & \cdots & \dfrac{\partial f^\layerindex_D (\vx)}{\partial x_D}  \end{bmatrix}
%\end{align}
%
%
Because the \gp{}s on each output dimension $\vf_d(\vx)$ are independent, it follows that each row of $J$ is independent.
Lemma \ref{thm:deriv-ind} shows that the elements of each row are independent Gaussian.
Thus all entries in the Jacobian of a \gp{}-distributed transform are independent Gaussian R.V.'s.
\end{proof}



\begin{theorem}
\label{thm:prodjacob}
The Jacobian of a deep \gp{} with a product kernel is a product of independent Gaussian matrices, with each entry in each matrix being drawn independently.
\end{theorem}
%
%\vspace{-0.2in}
\begin{proof}
When composing $L$ different functions, we'll denote the \emph{immediate} Jacobian of the function mapping from layer $\layerindex -1$ to layer $\layerindex$ as $J^{\layerindex}(\vx)$, and the Jacobian of the entire composition of $L$ functions by $J^{1:L}(\vx)$.
%
By the multivariate chain rule, the Jacobian of a composition of functions is simply the product of the immediate Jacobian matrices of each function.  
%
Thus the Jacobian of the composed (deep) function $\vf^{(L)}(\vf^{(L-1)}(\dots \vf^{(3)}( \vf^{(2)}( \vf^{(1)}(\vx)))\dots))$ is
%
% $J^L(f^{(L-1)}(\vx) \ldots J^2(f^{(1)}(\vx))J^1(\vx)$
%
\begin{align}
 J^{1:L}(\vx) 
% = \frac{\partial \fdeep(x) }{\partial x} 
%= \frac{\partial f^1(x) }{\partial x} \frac{\partial f^{(2)}(x) }{\partial f^{(1)}(x)} \cdots \frac{\partial f^L(x) }{\partial f^({L-1})(x)}
%= \prod_{\layerindex = 1}^{L} J^L (x)
= J^L J^{(L-1)} \dots J^3 J^2 J^1. 
\end{align}
%
By lemma \ref{thm:matrix}, each ${J^\layerindex_{i,j} \simind \mathcal{N}}$, so the complete Jacobian is a product of independent Gaussian matrices, with each entry of each matrix drawn independently.
\end{proof}

\vspace{-0.1in}
\Cref{thm:prodjacob} allows us to analyze the representational properties of a deep Gaussian process by examining the properties of products of independent Gaussian matrices.
%, a well-studied object.






\section{Formalizing a Pathology}
\label{sec:formalizing-pathology}

\citet{rifai2011higher} argue that good representations of data manifolds are invariant in directions orthogonal to the manifold.
Conversely, a good representation must also change in directions tangent to the data manifold, in order to preserve relevant information.
\Cref{fig:hidden} visualizes this idea.

%\newcolumntype{L}[1]{>{\raggedright\let\newline\\\arraybackslash\hspace{0pt}}m{#1}}
\begin{figure}[h]
\centering
%\begin{tabular}{L{\columnwidth}}
%\includegraphics[width=0.45\columnwidth]{\deeplimitsfiguresdir/hidden_good} &
\begin{tikzpicture}[pile/.style={thick, ->, >=stealth'}]
    \node[anchor=south west,inner sep=0] at (0,0) {
    	\includegraphics[clip, trim = 0cm 12cm 0cm 0.0cm, width=0.6\columnwidth]{\deeplimitsfiguresdir/hidden_good}
    };
    \coordinate (D) at (1.6,1.5);
    \coordinate (Do) at (2.5, 0.8);
    \coordinate (Dt) at (3,3);
    
    \draw[pile] (D) -- (Dt) node[above, text width=5em] { tangent };
    \draw[pile] (D) -- (Do) node[right, text width=5em] { orthogonal };
\end{tikzpicture}
%A noise-tolerant representation  of \\ a one-dimensional manifold (white)
%\includegraphics[clip, trim = 0cm 12cm 0cm 0.0cm, width=0.9\columnwidth]{\deeplimitsfiguresdir/hidden_bad} \\
%b) A na\"{i}ve representation (colors) \\ of a one-dimensional manifold (white)
%\end{tabular}
\caption[Desirable properties of representations of manifolds]
{Representing a 1-D data manifold.
%A representation is a function mapping the input space to some set of outputs.
Colors correspond to the computed representation as a function of the input space.
The representation (blue \& green) varies in directions tangent to the data manifold (white), preserving information for later layers. 
The representation changes little in directions orthogonal to the manifold, making it robust to noise in those directions.
%, and reducing the number of parameters needed to represent a datapoint.
%Representation b) changes in all directions, preserving potentially useless information.}% The representation on the right might be useful if the data were spread out in over plane.
}
\label{fig:hidden}
\end{figure}

As in \citep{rifai2011contractive}, we characterize the representational properties of a function by the singular value spectrum of the Jacobian.
The number of relatively large singular values of the Jacobian roughly correspond to the number of directions in which the representation varies.
\citet{rifai2011contractive} analyzed the Jacobian at location of the training points, but because the priors we are examining are stationary, the distribution of the Jacobian is identical everywhere.
%
\newcommand{\spectrumpic}[1]{%
\includegraphics[trim=4mm 1mm 4mm 2.5mm, clip, width=0.475\columnwidth]{\deeplimitsfiguresdir/spectrum/layer-#1}}%
\begin{figure}
\centering
\begin{tabular}{ccc}
& 2 Layers & 6 Layers \\
\hspace{-0.3cm} \begin{sideways} { \quad Normalized singular value} \end{sideways} & \hspace{-0.2in} \spectrumpic{2} & \hspace{-0.2in} \spectrumpic{6} \\
 & { Singular value index} & { Singular value index}
\end{tabular}
\caption[Distribution of singular values of the Jacobian of a deep \sgp{}]
{%The normalized singular value spectrum of the Jacobian of a deep \gp{} warping $\Reals^5 \to \Reals^5$.
The distribution of normalized singular values of the Jacobian of a function drawn from a 5-dimensional deep \gp{} prior 25 layers deep~(\emph{Left}) and 50 layers deep~(\emph{Right}).
As nets get deeper, the largest singular value tends to become much larger than the others.
This implies that with high probability, these functions vary little in all directions but one, making them insuitable for computing representations of manifolds of more than one dimension.
%As depth increases, the distribution on singular values also becomes heavy-tailed.
}
\label{fig:deep_spectrum}
\end{figure}%
%
\begin{figure}
\centering
\begin{tabular}{cc}
$p(\vx)$ & $p(\vf^{(1)}(\vx))$ \\
\gpdrawbox{1} &
\gpdrawbox{2} \\
$p(\vf^{(1:4)}(\vx))$ &  $p(\vf^{(1:6)}(\vx))$ \\
\gpdrawbox{4} & 
\gpdrawbox{6}
\end{tabular}
\caption[Points warped by a draw from a deep \sgp{}]
{Points warped by a function drawn from a deep \gp{} prior.
\emph{Top left:} Points drawn from a 2-dimensional Gaussian distribution, color-coded by their location.
\emph{Subsequent panels:} Those same points, successively warped by functions drawn from a \gp{} prior.
As the number of layers increases, the density concentrates along one-dimensional filaments.}
\label{fig:filamentation}
\end{figure}
%
\Cref{fig:deep_spectrum} shows the singular value spectrum for 5-dimensional deep GPs of different depths.
As the net gets deeper, the largest singular value dominates, implying there is usually only one effective degree of freedom in representation being computed.

\Cref{fig:filamentation} demonstrates a related pathology that arises when composing functions to produce a deep density model.
The density in the observed space eventually becomes locally concentrated onto one-dimensional manifolds, or \emph{filaments}, again implying that such models are unsuitable to model manifolds whose underlying dimensionality is greater than one.

\newcommand{\mappic}[1]{\hspace{-0.05in}\includegraphics[width=0.465\columnwidth]{\deeplimitsfiguresdir/map/latent_coord_map_layer_#1}} 
\newcommand{\mappiccon}[1]{\hspace{-0.05in} \includegraphics[width=0.465\columnwidth]{\deeplimitsfiguresdir/map_connected/latent_coord_map_layer_#1}}
\begin{figure}
\centering
\begin{tabular}{cc}
Identity Map: $\vy = \vx$ & 1 Layer: $\vy = f^{(1)}(\vx)$ \\
\mappic{0} & \mappic{1} \\
10 Layers: $\vy = f^{(1:10)}(\vx)$ & 40 Layers: $\vy = f^{(1:40)}(\vx)$ \\
\mappic{10} & \mappic{40}
\end{tabular}
\caption[Visualization of a feature map drawn from a deep \sgp{}]
{A vizualization of the feature map implied by a draw from a deep \gp{}.
Colors correspond to the location $\vy = \vf(\vx)$ that each point is mapped to after being warped by a deep \gp{}.
The number of directions in which the color changes rapidly corresponds to the number of large singular values in the Jacobian.
Just as the densities in \cref{fig:filamentation} became locally one-dimensional, there is usually only one direction that one can move $\vx$ in locally to change $\vy$.
This means that $\vf$ is unlikely to be a suitable representation for decision tasks that depend on more than one aspect of $\vx$.}
\label{fig:deep_map}
\end{figure}
%
To visualize this pathology in another way, \cref{fig:deep_map} illustrates a color-coding of the representation computed by a deep \gp{}, evaluated at each point in the input space.
After 10 layers, we can see that locally, there is usually only one direction that one can move in $\vx$-space in order to change the value of the computed representation.
This means that such representations are likely to be unsuitable for decision tasks that depend on more than one property of the input.

To what extent are these pathologies present in the types of neural networks commonly used in practice?
In simulations, we found that for deep functions with a fixed latent dimension $D$, the singular value spectrum remained relatively flat for hundreds of layers as long as $D > 100$.
Thus, these pathologies are unlikely to severely effect the relatively shallow, wide networks most commonly used in practice.





\section{Fixing the Pathology}
\label{sec:fix}


Following \citet{neal1995bayesian}, we can fix the pathologies exhibited in figures \cref{fig:filamentation} and \ref{fig:deep_map} by simply making each layer depend not only on the output of the previous layer, but also on the original input $\vx$.  
We refer to these models as \emph{input-connected} networks, and denote deep functions having this architecture with the subscript $C$, as in $f_C(\vx)$.
\Cref{fig:input-connected} shows a graphical representation of the two connectivity architectures.
Similar connections between non-adjacent layers can also be found the primate visual cortex \citep{maunsell1983connections}.
Formally, this functional dependence can be written as
\begin{align}
\vf_C^{(1:L)}(\vx) = \vf^{(L)} \left( \vf_C^{(1:L-1)}(\vx), \vx \right), \quad \forall L
\end{align}
%
Visualizations of the resulting prior on functions are shown in \cref{fig:deep_draw_1d_connected,fig:no_filamentation,fig:deep_map_connected}.

\begin{figure}[h]
\def\nodeseptwo{1.9cm}
\def\nodesize{.35cm}
\def\numhiddentwo{3}
\centering
\begin{tabular}{ccc}
a) Standard \MLP{} connectivity & &
b) Input-connected architecture\\
\hspace{-4mm}
\begin{tikzpicture}[draw=black!80]
    \tikzstyle{neuron}=[circle,minimum size=17pt, draw = black!80, fill = white, thick]
    \tikzstyle{input neuron}=[neuron, fill=green!50];
    \tikzstyle{output neuron}=[neuron, fill=red!50];
    \tikzstyle{hidden neuron}=[neuron, fill=blue!50];
    \tikzstyle{pile} =[thick, ->, >=stealth', shorten <=7pt, shorten >=8pt];

    % Define the input layer node
    \coordinate (I) at (0, 0);

    % Define the hidden layer nodes
    \foreach \name / \y in {1,...,\numhiddentwo} {
        \coordinate (H-\name) at (\nodeseptwo*\y, 0);
    }

    \path[pile] (I) edge (H-1) {};
    % Connect every node            
    \foreach \name in {2,...,\numhiddentwo} {
	 	\pgfmathsetmacro\hindex{\name - 1}
		\path[pile] (H-\hindex) edge (H-\name) {};
    }

    \draw (I) node[neuron] {};
    \draw (I) node[below = 0.5cm]  {$\vx$};

    % Draw the hidden layer nodes
    \foreach \name / \y in {1,...,\numhiddentwo} {
		\draw (H-\name) node[neuron]  {};
        \draw (H-\name) node[below = 0.34cm] {$\vf^{(\y)}(\vx)$};
    }
\end{tikzpicture} &
\hspace{0.5cm} &
\begin{tikzpicture}[draw=black!80]
    \tikzstyle{neuron}=[circle,minimum size=17pt, draw = black!80, fill = white, thick]
    \tikzstyle{input neuron}=[neuron, fill=green!50];
    \tikzstyle{output neuron}=[neuron, fill=red!50];
    \tikzstyle{hidden neuron}=[neuron, fill=blue!50];
    \tikzstyle{pile} =[thick, ->, >=stealth', shorten <=7pt, shorten >=8pt];

    % Define the input layer node
    \coordinate (I) at (0, 0);

    % Define the hidden layer nodes
    \foreach \name / \y in {1,...,\numhiddentwo} {
        \coordinate (H-\name) at (\nodeseptwo*\y, 0);
    }

    % Connect every node            
    \path[pile] (I) edge (H-1) {};
    \foreach \name in {2,...,\numhiddentwo} {
		\pgfmathsetmacro\hindex{\name - 1}
		\path[pile] (H-\hindex) edge (H-\name) {};
        \path[pile] (I) edge [bend left] (H-\name) {};
    }

    \draw (I) node[neuron] {};
    \draw (I) node[below = 0.5cm]  {$\vx$};

    % Draw the hidden layer nodes
    \foreach \name / \y in {1,...,\numhiddentwo} {
		\draw (H-\name) node[neuron]  {};
       	\draw (H-\name) node[below = 0.34cm] {$\vf_C^{(\y)}(\vx)$};
    }
\end{tikzpicture}
\end{tabular}
\caption[Two different architectures for deep neural networks]
{Two different architectures for deep neural networks.
\emph{Left:} The standard architecture connects each layer's outputs to the next layer's inputs.
\emph{Right:} The input-connected architecture also connects the original input $\vx$ to each layer.}
\label{fig:input-connected}
\end{figure}



\begin{figure}[h]
\centering
\begin{tabular}{cccc}
\hspace{-0.1in} One layer & \hspace{-0.2in} 2 Layers & \hspace{-0.2in}  5 Layers & \hspace{-0.25in} 10 Layers \\
\hspace{0.03in}
\onedsamplepiccon{1} &
\onedsamplepiccon{2} &
\onedsamplepiccon{5} &
\onedsamplepiccon{10}
\end{tabular}
\caption[Draws from a 1D deep \sgp{} prior with each layer connected to the input]
{Draws from a 1D deep \gp{} prior with each layer connected to the input.
Even after many layers, the functions remain smooth in some regions, while varying rapidly in other regions.
Compare to standard-connectivity deep \gp{} draws shown in \cref{fig:deep_draw_1d}.}
\label{fig:deep_draw_1d_connected}
\end{figure}
%
\newcommand{\gpdrawboxcon}[1]{
\setlength\fboxsep{0pt}
\hspace{-0.2in} 
\fbox{
\includegraphics[width=0.464\columnwidth]{\deeplimitsfiguresdir/deep_draws_connected/deep_sample_connected_layer#1}
}}
%
\begin{figure}
\centering
\begin{tabular}{cc}
3 Layers & 6 Layers \\
\gpdrawboxcon{3} &
\gpdrawboxcon{6}
\end{tabular}
\caption[Points warped by a draw from an input-connected deep \sgp{}]
{Points warped by a draw from a deep \sgp{}, with each layer connected to the input $\vx$.
%\emph{Top left:} Points drawn from a 2-dimensional Gaussian distribution, color-coded by their location.
%\emph{Subsequent panels:} Those same points, successively warped by functions drawn from a \gp{} prior.
As depth increases, the density becomes more complex without concentrating everywhere along one-dimensional filaments.}
\label{fig:no_filamentation}
\end{figure}
%
\begin{figure}
\centering
\newcommand{\spectrumpiccon}[1]{
\includegraphics[trim=4mm 1mm 4mm 2.5mm, clip, width=0.475\columnwidth]{\deeplimitsfiguresdir/spectrum/con-layer-#1}} 
\begin{tabular}{ccc}
 & 25 layers &  50 layers \\
\hspace{-0.5cm} \begin{sideways} {\scriptsize \quad Normalized singular value} \end{sideways} & \hspace{-0.2in} \spectrumpiccon{25} & \hspace{-0.16in} \spectrumpiccon{50} \\
 & {\scriptsize Singular value number} & {\scriptsize Singular value number}
\end{tabular}
\caption[Distribution of singular values of an input-connected deep \sgp{}]
{The distribution of singular values drawn from 5-dimensional input-connected deep GP priors, 25 and 50 layers deep.
The singular values remain roughly the same scale as one another.}
\label{fig:good_spectrum}
\end{figure}
%
%
\begin{figure}
\centering
\begin{tabular}{cc}
Identity map: $\vy = \vx$ & 2 Connected layers \\
\hspace{-0.07in} \mappic{0} & \mappiccon{2} \\
 10 Connected layers & 20 Connected layers \\
\hspace{-0.07in} \mappiccon{10} & \mappiccon{20}
\end{tabular}
\caption[Feature mapping of an input-connected deep \sgp{}]
{Feature mapping of a deep \gp{} with each layer connected to the input $\vx$.
Compare to the mappings shown in \cref{fig:deep_map}.
Just as the densities in \cref{fig:no_filamentation} remained locally two-dimensional even after many warpings, in the mapping shown here there are sometimes two directions that one can move locally in $\vx$ to in order to change the values of $\vf(\vx)$.
This means that the input-connected prior puts mass on a greater variety of types of representations, some of which depend on all aspects of the input.
}
\label{fig:deep_map_connected}
\end{figure}


The Jacobian of an input-connected deep function is defined by the recurrence
%
\newcommand{\sbi}[2]{\left[ \! \begin{array}{c} #1 \\ #2 \end{array} \! \right]} 
%\newcommand{\sbi}[2]{\left[ #1 \quad \!\! #2 \right]} 
\begin{align}
{J_C^{1:L}(\vx) = J^L \sbi{ J_C^{1:L-1}}{I_D}}.
\end{align}
%
%So the entire Jacobian has the form:
%
%\begin{align}
%J^{1:L}(x) = J^L \sbi{ J^{L-1} \sbi{ \dots J^{4} \sbi{ J^{3} \sbi{ J^2 J^1 }{ I_D }}{ I_D } \dots }{ I_D \\ %\vdots }}{ I_D}
%\end{align}
%
which, in the case of a deep \gp{}, is still a product of independent Gaussian matrices.
\Cref{fig:good_spectrum} shows that with this architecture, even 50-layer deep \gp{}s have well-behaved singular value spectra.

The pathology examined in this section is an example of the sort of analysis made possible by a well-defined prior on functions.




\section{Deep Kernels}
\label{sec:deep_kernels}


\cite{ bengio2006curse} showed that kernel machines have limited generalization ability when they use ``local'' kernels, such as the squared-exp.
However, as we saw in \cref{ch:kernels,ch:grammar}, structured kernels can be constructed through addition and multiplication which allow non-local extrapolation.

We can also build non-local kernels by composing fixed feature maps.
To return to an example given in \cref{sec:expressing-symmetries}, periodic kernels can be viewed as a 2-layer-deep kernel, in which the first layer maps $x \rightarrow [\sin(x), \cos(x)]$, and the second layer maps through basis functions corresponding to the \humble{SE} kernel.

%In addition to analyzing \MLP{}s with random weights, we can also analyze fixed feature mappings with different connectivity architectures.
 
Can we construct other useful kernels by composing fixed feature maps several times, creating deep kernels?  \citet{cho2012kernel} constructed kernels of this form, repeatedly applying multiple layers of feature mappings.
%Given a kernel $k_1(\vx, \vx') = \hPhi(\vx) \tra \hPhi(\vx')$, 
In principle, we can compose the feature mapping of any two kernels to get a new one:
%
\begin{align}
k_{1}(\vx, \vx') & = \hPhi_1(\vx) \tra \hPhi_1(\vx') \\
k_{2}(\vx, \vx') & = \hPhi_2(\vx) \tra \hPhi_2(\vx') \\
\left( k_2 \circ k_1 \right)(\vx, \vx') & = k_2 \left(\hPhi_1(\vx), \hPhi_1(\vx') \right) \\
& = \left[ \hPhi_2 \left( \hPhi_1(\vx) \right)\right] \tra \hPhi_2 \left(\hPhi_1(\vx') \right) 
\end{align}
%
However, this composition might not always have a closed form.

Fortunately, composing the squared-exp kernel with any implicit mapping $\hPhi(\vx)$ has a simple closed form:
%, this composition operation has a closed form for any starting kernel:% for any set of starting features $\hPhi_n(\vx)$:
%
\begin{align}
%k_1(\vx, \vx') & = \exp \left( -\frac{1}{2} ||\vx - \vx'||_2^2 \right) \\
\left( \kSE \circ k \right)(\vx, \vx') \left( \vx, \vx' \right) & = k_{SE} \left( \hPhi(\vx), \hPhi(\vx') \right) =  \\
%& = \left( \hPhi^{SE} \left(\hPhi^{1}(\vx) \right) \right) \tra \hPhi^{SE} \left( \hPhi^{1}(\vx') \right) \\
& = \exp \left( -\frac{1}{2} || \hPhi(\vx) - \hPhi(\vx')||_2^2 \right) \nonumber\\
%k_{n+1}(\vx, \vx') 
%& = \exp \left( -\frac{1}{2} \sum_i \left[ \hphi_n^{(i)}(\vx) - \hphi_n^{(i)}(\vx') \right]^2 \right) \\
& = \exp\left ( -\frac{1}{2} \left[ \hPhi(\vx) \tra \hPhi(\vx) - 2 \hPhi(\vx) \tra \hPhi(\vx') + \hPhi(\vx') \tra \hPhi(\vx') \right] \right) \nonumber \\
%k_2(\vx, \vx') & = \exp \left( -\frac{1}{2} \left[ \sum_i \hphi_i(\vx)^2 - 2 \sum_i \hphi_i(\vx) \hphi_i(\vx') + \sum_i \hphi_i(\vx')^2 \right] \right) \\
%k_{n+1}(\vx, \vx') 
& = \exp \left( -\frac{1}{2} \left[ k(\vx, \vx) - 2 k(\vx, \vx') + k(\vx', \vx') \right] \right) \nonumber
%k_{n+1}(\vx, \vx') 
%& = \exp \left( k_1(\vx, \vx') - 1 \right) \qquad \textnormal{(if $k_1(\vx, \vx) = 1$)} \nonumber
\end{align}
%
%\begin{figure}
%\centering
%\begin{tabular}{ccc}
%\includegraphics[width=0.5\columnwidth, clip, trim = 0cm 0cm 0cm 0.61cm]{\deeplimitsfiguresdir/deep_kernel} &
%\includegraphics[width=0.5\columnwidth, clip, trim = 0cm 0cm 0cm 0.61cm]{\deeplimitsfiguresdir/deep_kernel_draws} \\
%Kernel derived from iterated feature transforms & Draws from the corresponding kernel
%\end{tabular}
%\caption{A degenerate kernel produced by repeatedly applying a feature transform.}
%\label{fig:deep_kernel}
%\end{figure}
%
%Thus, if $k_1(x,y) = e^{-||x - y||2}$, then the two-layer kernel is simply $k_2(x,y) = e^{k_1(x, y) - 1}$.  This formula is true for every layer: $k_{n+1}(x,y) = e^{k_n(x, y) - 1}$.
%
%Note that nothing in this derivation depends on details of $k_n$, except that $k_n( \vx, \vx) = 1$.
%Note that this result holds for any base kernel $k_n$, as long as $k_n( \vx, \vx) = 1$.
%  Because this is true for $k_2$ as well, this recursion holds in general, and we have that $k_{n+1}(x,y) = e^{k_n(x, y) - 1}$.  
which lets us express the composed kernel, $\kSE \circ k$, exactly in terms of evaluations of $k$.

\subsection{Infinitely Deep Kernels}
What happens when we repeat this composition of feature maps many times, starting with the squared-exp kernel?
If the output variance is set to 1, then in the infinite limit, this operation converges to $\left(\kSE \circ \kSE \circ \kSE \circ \ldots \kSE\right)(\vx,\vx') = 1$ for all pairs of inputs, which corresponds to a prior on constant functions $f(\vx) = c$.

%Figure \ref{fig:deep_kernel_connected} shows this kernel at different depths, including the degenerate limit.  
%
%One interpretation of why repeated feature transforms lead to this degenerate prior is that each layer can only lose information about the previous set of features.  
%In the limit, the transformed features contain no information about the original input $\vx$.  Since the function doesn't depend on its input, it must be the same everywhere.

%\subsubsection{A non-degenerate construction}

As before, we can overcome this degeneracy by connecting the inputs $\vx$ to each layer.  To do so, we simply augment the feature vector at each layer, $\hPhi^{(\ell)}(\vx)$, with the input vector $\vx$:
%
\begin{align}
%k_1(\vx, \vx') & = \exp \left( -\frac{1}{2} ||\vx - \vx'||_2^2 \right) \\
 k^{(\ell + 1)}(\vx, \vx') % = \nonumber\\
& = \exp \left( -\frac{1}{2} \left|\left| \left[ \! \begin{array}{c} \hPhi^{(\ell)}(\vx) \\ {\color{black} \vx} \end{array} \! \right]  - \left[ \! \begin{array}{c} \hPhi^{(\ell)}(\vx') \\ {\color{black} \vx'} \end{array} \! \right] \right| \right|_2^2 \right) \nonumber \\
%k_{n+1}(\vx, \vx') 
%& = \exp \left( -\frac{1}{2} \sum_i \left[ \hphi_i(\vx) - \hphi_i(\vx') \right]^2 -\frac{1}{2} || \vx - \vx' ||_2^2 \right) \\
%k_{n+1}(\vx, \vx') & = \exp\left ( -\frac{1}{2} \sum_i \left[ \hphi_i(\vx)^2 - 2 \hphi_i(\vx) \hphi_i(\vx') + \hphi_i(\vx')^2 \right]  -\frac{1}{2} || \vx - \vx' ||_2^2 \right) \\
%k_2(\vx, \vx') & = \exp \left( -\frac{1}{2} \left[ \sum_i \hphi_i(\vx)^2 - 2 \sum_i \hphi_i(\vx) \hphi_i(\vx') + \sum_i \hphi_i(\vx')^2 \right] \right) \\
%k_2(\vx, \vx') & = \exp \left( -\frac{1}{2} \left[ k_1(\vx, \vx) - 2 k_1(\vx, \vx') + k_1(\vx', \vx') \right] \right) \\
%k_{n+1}(\vx, \vx') 
%& = \exp \left( k_n(\vx, \vx') - 1 -\frac{1}{2} || \vx - \vx' ||_2^2 \right)
& = \exp \Big( -\frac{1}{2} \big[ k^{(\ell)}(\vx, \vx) - 2 k^{(\ell)}(\vx, \vx') % \nonumber \\ 
 + k^{(\ell)}(\vx', \vx') {\color{black} - || \vx - \vx' ||_2^2} \big] \Big)
\end{align}
%
Starting with the squared-exp kernel, this repeated mapping satisfies
\begin{align}
k^{(\infty)}(\vx, \vx') - \log \left( k^{(\infty)}(\vx, \vx') \right) = 1 + \frac{1}{2} || \vx - \vx' ||_2^2
\end{align}
%
The solution to this recurrence has no closed form, but has a similar shape to the Ornstein-Uhlenbeck covariance ${\OU(x,x') = \exp( -|x - x'| )}$ but with lighter tails.
%
Samples from a \gp{} prior with this kernel are not differentiable, and are locally fractal.
\Cref{fig:deep_kernel_connected} shows this kernel at different depths, as well as samples from the resulting \gp{} prior.
%\item This kernel has smaller correlation than the squared-exp everywhere except at $\vx = \vx'$.  
%\item The tails have the same form as the squared-exp.

\begin{figure}
\centering
\begin{tabular}{cc}
Input-connected deep kernels & Draws from corresponding \gp{}s \\
\hspace{-0.3cm}
\rotatebox{90}{$\qquad \cov(f(x), f(x'))$}
\includegraphics[width=0.465\columnwidth, clip, trim = 1.3cm 0.4cm 0.9cm 0.3cm]{\deeplimitsfiguresdir/deep_kernel_connected} &
\hspace{-0.3cm}
\rotatebox{90}{$\qquad \qquad f(x)$}
\includegraphics[width=0.44\columnwidth, clip, trim = 1.29cm 0.1cm 0.9cm 0.35cm]{\deeplimitsfiguresdir/deep_kernel_connected_draws} \\
$ x - x'$ &  $ x - x'$
\end{tabular}
\caption[Infinitely deep kernels]{
\emph{Left:}  Input-connected deep kernels of different depths.
By connecting the inputs $\vx$ to each layer, the kernel can still depend on its input even after arbitrarily many layers of computation.
\emph{Right:} Draws from \gp{}s with deep input-connected kernels.}
\label{fig:deep_kernel_connected}
\end{figure}

We can also consider two other connectivity architectures: one in which each layer is connected to the output layer, and another in which every layer is connected to all subsequent layers.
It is easy to show that in the limit of infinite depth, both these architectures converge to $k(\vx, \vx') = \delta( \vx, \vx' )$, a white noise kernel.



%\subsection{A fully-connected kernel}
%However, connecting every layer to every subsequent layer leades to a pathology:
%\begin{align}
%k_1(\vx, \vx') & = \exp \left( -\frac{1}{2} ||\vx - \vx'||_2^2 \right) \\
%k_{n+1}(\vx, \vx') & = \exp \left( -\frac{1}{2} \left|\left| \left[ \! \begin{array}{c} \hPhi_n(\vx) \\ \hPhi_{n-1}(\vx') \end{array} \! \right]  - \left[ \! \begin{array}{c} \hPhi_n(\vx') \\ \hPhi_{n-1}(\vx') \end{array} \! \right] \right| \right|_2^2 \right) \\
%k_{n+1}(\vx, \vx') & = \exp \left( -\frac{1}{2} \sum_i \left[ \hphi^{(i)}_n(\vx) - \hphi^{(i)}_n(\vx') \right]^2 -\frac{1}{2} \sum_i \left[ \hphi^{(i)}_{n-1}(\vx) - \hphi^{(i)}_{n-1}(\vx') \right]^2 \right)\\
%k_{n+1}(\vx, \vx') & = \exp\left ( -\frac{1}{2} \sum_i \left[ \hphi_i(\vx)^2 - 2 \hphi_i(\vx) \hphi_i(\vx') + \hphi_i(\vx')^2 \right]  -\frac{1}{2} || \vx - \vx' ||_2^2 \right) \\
%k_2(\vx, \vx') & = \exp \left( -\frac{1}{2} \left[ \sum_i \hphi_i(\vx)^2 - 2 \sum_i \hphi_i(\vx) \hphi_i(\vx') + \sum_i \hphi_i(\vx')^2 \right] \right) \\
%k_2(\vx, \vx') & = \exp \left( -\frac{1}{2} \left[ k_1(\vx, \vx) - 2 k_1(\vx, \vx') + k_1(\vx', \vx') \right] \right) \\
%k_{n+1}(\vx, \vx') & = \exp \left( k_n(\vx, \vx') - 1 \right) \exp \left( k_{n-1}(\vx, \vx') - 1 \right)
%k_{n+1}(\vx, \vx') & = \prod_{i=1}^n \prod_{j=1}^i \exp \left( k_i(\vx, \vx') - 1 \right)
%\end{align}
%Which has the solution $k_\infty = \delta( \vx = \vx' )$, a white noise kernel.


%\subsection{Connecting every layer to the end}
%There is a fourth possibilty (suggested by Carl), of connecting every layer to the output:
%\begin{align}
%k_n(\vx, \vx') & = \exp \left( -\frac{1}{2} || \hPhi_n(\vx) - \hPhi_n(\vx')||_2^2 \right) \\
%k_{n+1}(\vx, \vx') & = \exp \left( k_n(\vx, \vx') - 1 \right) \\
%k_{L}(\vx, \vx') & = \prod_{i=1}^L \exp \left( k_i(\vx, \vx') - 1 \right)
%\end{align}
%Which also has the solution $k_\infty = \delta( \vx = \vx' )$, a white noise kernel.


\subsection{When are Deep Kernels Useful Models?}

Kernels correspond to fixed feature maps, and so kernel learning is an example of implicit representation learning. %can compute %useful representations.
As we saw in \cref{ch:kernels,ch:grammar}, kernels can capture rich structure,
% \citep{DuvLloGroetal13}
 and can enable many types of generalization.
 %, such as affine invariance in images \citep{kondor2008group}.
%\cite{salakhutdinov2008using} used a deep neural network to learn feature transforms for kernels, which learn invariants in an unsupervised manner.
The relatively uninteresting properties of the kernels derived in this section simply reflect the fact that an arbitrary computation, even if it is ``deep'', is not likely to give rise to a useful representation, unless combined with learning.
To put it another way, any fixed representation is unlikely to be useful unless it has been chosen specifically for the problem at hand.






\section{Related Work}

\subsubsection{Deep Gaussian Processs}
\citet{neal1995bayesian} discussed the properties of arbitrarily deep random networks, including those that would give rise to deep \gp{}s.
However, deep \gp{}s were first clearly defined and developed by \cite{lawrence2007hierarchical}, as a model of hierarchical generative relationships.
Deep \gp{}s were first referred to as such by \citet{damianou2012deep}, who developed a variational inference scheme and analyzed the effect of automatic relevance determination in that model.

\subsubsection{Nonparametric Neural Networks}
%Other Bayesian deep neural network models have been proposed by, for example, . % and Sum-product networks \cite{poon2011sum}  
\citet{adams2010learning} proposed a prior on arbitrarily deep Bayesian networks.
Their architecture has connections only between adjacent layers, and may also be expected to have similar pathologies to those of deep \gp{}s as the number of layers increases.

Deep Density Networks \citep{rippel2013high} were constructed with invertibility in mind, with penalty terms encouraging the preservation of information about lower layers.
This is another promising approach to fixing the pathology discussed above.

\citet{wilson2012gaussian} introduced \gp{} Regression Networks (\gprn{}), which define a matrix product of \gp{}s, rather than a composition.
The form of a \gprn{} is
%
\begin{align}
\vy(\vx) & = \vW(\vx) \vf(\vx) \qquad \textnormal{with each} \; f_{d}, W_{d,j} \simiid \GPt{\vzero}{\kSE + \kWN} .
%\textnormal{each} \; W_{d,j} & \simiid \GPt{\vzero}{\kSE + \kWN}
\end{align}
%
We can easily define a ``deep'' \gprn{}:
%
\begin{align}
\vy(\vx) = \vW^{(3)}(\vx) \vW^{(2)}(\vx) \vW^{(1)}(\vx) \vf(\vx)
\end{align}
%
which repeatedly adds and multiplies functions drawn from \gp{}s, in contrast to deep \gp{}s which repeatedly compose functions.
This prior on functions might be amenable to a similar analysis to that of section \ref{sec:characterizing-deep-gps}.

%
%This model has a similar form to the Jacobian of deep \gp{}, (since each row of the Jacobian is jointly a \gp{} but with with a kernel given by \eqref{eq:deriv-kernel}) except that the Jacobian is evaluated at the output from previous layers:
%
%\begin{align}
%\vy(\vx) = J^3(\vf^2(\vx)) J^2(\vf^1(\vx)) J^1(\vx)
%\end{align}
%




\subsubsection{Recurrent Networks}
\cite{bengio1994learning} and \cite{pascanu2012understanding} analyze a related problem with gradient-based learning in recurrent nets, the ``exploding-gradients'' problem.
Specifically, they note that in recurrent neural networks, the size of the training gradient can grow or shrink exponentially as it is back-propagated, making gradient-based training difficult.

\subsubsection{Deep Kernels}

% TODO: Flesh out Cho reference.
%In addition to \citet{cho2012kernel}, 

\cite{hermans2012recurrent} construct deep kernels in a time-series setting, constructing kernels corresponding to infinite-width \emph{recurrent} neural networks.
They also propose concatenating the implicit feature vectors from previous time-steps with the current inputs, resulting in an architecture analogous to the input-connected architecture proposed by \citet{neal1995bayesian}.

\subsubsection{Analyses of Deep Learning}
\cite{montavon2010layer} perform a layer-wise analysis of deep networks, and note that the performance of MLPs degrades as the number of layers with random weights increases.
The importance of network architectures relative to learning has been examined by \cite{saxe2011random}.
Later, \cite{saxedynamics} looked at the dynamics of learning in deep linear models, as a tractable approximation to deep neural networks.  




%\section{Discussion}


%\paragraph{Recursive learning method}
%Just as layer-wise unsupervised pre-training encourages the projection of the data into a representation with independent features in the higher layers, so does the procedure outlined here.  This is because the isotropic kernel does not penalize independence between different dimensions, only the number of dimensions.


\section{Conclusions}

%In this work, we established a number of propositions which help us gain insight into the properties of very deep models, and allow making informed choices regarding their architecture.

In this chapter, we showed that well-defined priors allow us to explicitly examine the assumptions being made about functions we may wish to learn.
As an example of the sort of analysis made possible this way, we attempted to gain insight into the properties of deep neural networks by characterizing the sorts of functions likely to be obtained under different choices of priors on compositions of functions.
%established a number of propositions which help us gain insight into the properties of very deep models, and allow making informed choices regarding their architecture.

%First, we identified equivalences between multi-layer perceptions and deep \gp{}s --- namely, that a deep \gp{} can be written as an \MLP{}s with a finite number of nonparametric hidden units, or as an \MLP{} with infinitely-many parametric hidden units.
First, we identified deep Gaussian processes as an easy-to-analyze model corresponding to multi-layer preceptrons with nonparametric activation functions.
%  We also showed several other connections between deep \MLP{}s and different Gaussian process models. %equivalences between multi-layer perceptions and deep \gp{}s --- namely, that a deep \gp{} can be written as an \MLP{}s with a finite number of nonparametric hidden units, or as an \MLP{} with infinitely-many parametric hidden units.
%
%Second, we characterized the derivatives and Jacobians of deep \gp{}s through products of independent Gaussian matrices. % can be characterized using random matrix theory, which we applied to establish results regarding the distribution over the Jacobian of the composition transformation. 
We then showed that representations based on repeated composition of independent functions exhibit a pathology where the representations becomes invariant to all directions of variation but one. % This leads to extremely restricted expressiveness of such deep models in the limit of increasing number of layers. 
Finally, we showed that this problem could be alleviated by connecting the input to each layer.
%proposed a way to alleviate this problem: connecting the input to each layer of a deep representation allows us to construct priors on deep functions that do not exhibit the information-capacity pathology.
%
We also examined properties of deep kernels, corresponding to arbitrarily many compositions of fixed features.
%Finally, we derived models obtained by performing dropout on Gaussian processes, finding a tractable approximation to exact dropout in \gp{}s.

Much recent work on deep networks has focused on weight initialization \citep{martens2010deep}, regularization \citep{lee2007sparse} and network architecture \citep{gens2013learning}.
However, the interactions between these different design decisions can be complex and difficult to characterize.
%We propose to approach the design of deep architectures by examining the problem of assigning priors to nested compositions of functions.
% Although inference in fully Bayesian models is generally challenging, well-defined priors allow us to encode our beliefs into models in a data-independent way. 
If we can identify classes of priors that give our models desirable properties, these in turn may suggest regularization, initialization, and architecture choices that also provide such properties.

\subsubsection{Source Code}
Source code to produce all figures is available at \url{github.com/duvenaud/deep-limits/}. 



\outbpdocument{
%This will be ignored, it's just so that Gummi can find the bibliography.
\bibliographystyle{plainnat}
\bibliography{references.bib}
qwerty
}

