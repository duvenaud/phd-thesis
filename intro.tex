%
% A header that lets you compile a chapter by itself, or inside a larger document.
% Adapted from http://stackoverflow.com/questions/3655454/conditional-import-in-latex
%
%
%Use \inbpdocument and \outbpdocument in your individual files, in place of \begin{document} and \end{document}. In your main file, put in a \def \ismaindoc {} before including or importing anything.
%
% David Duvenaud
% June 2011
% 
% ======================================
%
%


\ifx\ismaindoc\undefined
	\newcommand{\inbpdocument}{
		\def \ismaindoc {}
		% Use this header if we are compiling by ourselves.
		\documentclass[a4paper,12pt,authoryear,index]{common/PhDThesisPSnPDF}
		\input{common/official-preamble.tex}
		% All my custom preamble stuff.  Shouldn't overlap with anything in official-preamble

\usepackage{nth}
\usepackage{rotating}
\usepackage{array}
%\usepackage{gantt}
\usepackage[hyperpageref]{backref}

		% ************************ Thesis Information & Meta-data **********************

%% The title of the thesis
%\title{Structured Gaussian Process Models} 
%\title{Automatic Model Construction \\ through \\ Structured Gaussian Processes}
%\title{Automatic Model-Building \\ through \\ Structured Gaussian Processes}
%\title{Automatic Modeling \\ with \\ Structured Gaussian Processes}    
\title{Automatic Model Construction \\ with Gaussian Processes}
%\title{Automatic Model Construction}
%\title{Automating Statistical Model Construction}


%\texorpdfstring is used for PDF metadata. Usage:
%\texorpdfstring{LaTeX_Version}{PDF Version (non-latex)} eg.,
%\texorpdfstring{$sigma$}{sigma}

%% The full name of the author
\author{David Kristjanson Duvenaud}

%% Department (eg. Department of Engineering, Maths, Physics)
\dept{Department of Engineering}

%% University and Crest
\university{University of Cambridge}
\crest{\includegraphics[width=0.25\textwidth]{University_Crest}}

%% You can redefine the submission text:
% Default as per the University guidelines: This dissertation is submitted for
% the degree of Doctor of Philosophy
%\renewcommand{\submissiontext}{change the default text here if needed}

%% Full title of the Degree 
\degree{Doctor of Philosophy}
 
%% College affiliation (optional)
\college{Pembroke College}

%% Submission date
\degreedate{June 2014} 

%% Meta information
\subject{LaTeX} \keywords{{LaTeX} {PhD Thesis} {Engineering} {University of
Cambridge}}



		\begin{document}
	}	
	\newcommand{\outbpdocument}[1]{
		%\bibliographystyle{common/CUEDthesis}
		\bibliographystyle{plainnat}
		\bibliography{references.bib}
		\end{document}
	}	
\else
	%If we're inside another document, no need to re-start the document.
	\ifx\inbpdocument\undefined
		\newcommand{\inbpdocument}{}
		\newcommand{\outbpdocument}[1]{}
	\fi
\fi

\inbpdocument

\chapter{Introduction}
\label{ch:intro}


\begin{quotation}
``I only work on intractable nonparametrics - Gaussian processes don't count.''

\hspace*{\fill} Sinead Williamson, personal communication
\end{quotation}


\section{Regression}

% These two paras written with James L and Zoubin
The general problem of regression consists of learning a function $f$ mapping from some input space $\mathcal{X}$ to some output space $\mathcal{Y}$.
We would like an expressive language which can represent both simple parametric forms of $f$ such as linear, polynomial, \etc and also complex nonparametric functions specified in terms of properties such as smoothness, periodicity, \etc~ 
Fortunately, Gaussian processes (\gp{}s) provide a very general and analytically tractable way of capturing both simple and complex functions. 

\section{Gaussian process models}

Gaussian processes are distributions over functions such that any finite subset of function evaluations, $(f(x_1), f(x_2), \ldots f(x_N))$, have a joint Gaussian distribution \citep{rasmussen38gaussian}.
A \gp{} is completely specified by its mean function, $\mu(x)=\mathbb{E}(f(x))$ and kernel (or covariance) function $\kernel(x,x') = \Cov(f(x),f(x'))$.
It is common practice to assume zero mean, since marginalizing over an unknown mean function can be equivalently expressed as a zero-mean \gp{} with a new kernel.
The structure of the kernel captures high-level properties of the unknown function, $f$, which in turn determines how the model generalizes or extrapolates to new data.
We can therefore define a language of regression models by specifying a language of kernels.

\subsection{Useful properties of Gaussian process models}

\begin{itemize}
\item {\bf Tractable inference} Given a kernel function, the posterior distribution can be computed exactly in closed form.  This is a rare property for nonparametric models to have.
\item {\bf Expressivity} by choosing different covariance functions, we can express a very wide range of modeling assumptions.
\item {\bf Integration over hypotheses} the fact that a \gp{} posterior lets us exactly integrate over a wide range of hypotheses means that overfitting is less of an issue than in comparable model classes - for example, neural nets.
\item {\bf Marginal likelihood} A side benefit of being able to integrate over all hyoptheses is that we compute the \emph{marginal likelihood} of the data given the model.  This gives us a principled way of comparing different Gaussian process models.
\item {\bf Closed-form posterior} The posterior predictive distribution of a \gp{} is another \gp{}.  This means that \gp{}s can easily be composed with other models or decision procedures.  For example, \NA{Carl's reinforcement learning work.}
\end{itemize}


\begin{figure}[t]
\begin{centering}
\begin{tabular}{cc}
\includegraphics[width=0.48\textwidth]{\introfigsdir/fuzzy-0} & 
\includegraphics[width=0.48\textwidth]{\introfigsdir/fuzzy-1} \\
\includegraphics[width=0.48\textwidth]{\introfigsdir/fuzzy-2} & 
\includegraphics[width=0.48\textwidth]{\introfigsdir/fuzzy-3}
\end{tabular}
\end{centering}
\caption[One-dimensional Gaussian process posterior]{A visual representation of a one-dimensional Gaussian process posterior.
Red isocountours show the marginal density at each input location.
Coloured lines are samples from the posterior.}
\label{fig:gp-post}
\end{figure}

Figure \ref{fig:gp-post} shows a Gaussian process posterior.  Typically, it's rendered with the mean and +- 2SD, but there's nothing special about mean.




\section{Latent Variable Models}

Besides being useful for modeling functions, a simple extension allows \gp{}s to be useful for general density modeling.  Unfortunately, this extension causes many of the useful properties of the \gp{} not to hold.


\begin{figure}[t]
\begin{centering}
\includegraphics[width=\textwidth]{\gplvmfiguresdir/gplvm_1d_draw_9}
\end{centering}
\caption[One-dimensional Gaussian process latent variable model]{A visual representation of the Gaussian process latent variable model.  Bottom: density and samples from a 1D Gaussian, specifying the distribution $p(\vX)$ in the latent space.  Top Right: A function drawn from a \gp{} prior.  Left: A nonparametric density defined by warping the latent density through the function drawn from a \gp{} prior.}  
\label{fig:oned-gplvm}
\end{figure}



\begin{figure}
\begin{centering}
{\begin{tabular}{cccc}
\phantom{h} & Latent space $p(\vX)$ & & Observed space $p(\vY)$ \\
& \fbox{\includegraphics[height=15em,width=15em]{\gplvmfiguresdir/gplvm-latent-crop}} &
\raisebox{7em}{$\overset{\mathlarger{f(x)}}{\mathlarger{\mathlarger{\mathlarger{\mathlarger{\mathlarger{\mathlarger{\mathlarger{\rightarrow}}}}}}}}$} &
%\raisebox{7em}{\Huge $\overset{\textnormal{\Large{GP}}}\rightarrow$} &
\fbox{\includegraphics[height=15em,width=15em]{\gplvmfiguresdir/gplvm-draw-crop}}
\end{tabular}}
\end{centering}
\caption[Two-dimensional Gaussian process latent variable model]{A visual representation of the Gaussian process latent variable model.  Left:  Isocontours and samples from a 2D Gaussian, specifying the distribution $p(\vX)$ in the latent space.  Right: Density and samples from a nonparametric density defined by warping the latent density through a function drawn from a GP prior.}  
\label{fig:twod-gplvm}
\end{figure}




\section{Covariance functions}

Kernels specify similarity between function values of two objects, not between similarity of objects





\section{Structure through additivity}

A theme throughout this thesis is exploring the idea that a lot of the expressivity of \gp{} models comes from the fact that these models can be combined and decomposed additively.


\subsection{Derivation of Component Marginal Variance}

In this section, we derive the posterior marginal variance and covariance of the additive components of a \gp{}.  These formulas let us plot the marginal variance of each component separately.  These formulas can also be used to examine the posterior covariance between pairs of components.

Let us assume that our function $\vf$ is a sum of two functions, $\vf_1$ and $\vf_2$, where $\vf = \vf_1 + \vf_2$.  If $\vf_1$ and $\vf_2$ are a priori independent, and $\vf_1 \sim \gp( \vmu_1, k_1)$ and $\vf_2 \sim \gp( \vmu_2, k_2)$, then
\begin{align}
\left[ \begin{array}{c} \vf_1 \\ \vf_1^\star \\ \vf_2 \\ \vf_2^\star \\ \vf \\ \vf^\star \end{array} \right]
\sim
\Nt{
\left[ \begin{array}{c} \vmu_1 \\ \vmu_1^\star \\ \vmu_2 \\ \vmu_2^\star \\ \vmu_1 + \vmu_2 \\ \vmu_1^\star + \vmu_2^\star \end{array} \right]
}
{\left[ \begin{array}{cccccc} 
\vk_1 & \vk_1^\star & 0 & 0 & \vk_1 & \vk_1^\star \\ 
\vk_1^\star & \vk_1^{\star\star} & 0 & 0 & \vk_1^\star & \vk_1^{\star\star} \\
0 & 0 & \vk_2 & \vk_2^\star & \vk_2 & \vk_2^\star \\ 
0 & 0 & \vk_2^\star & \vk_2^{\star\star} & \vk_2^\star & \vk_2^{\star\star} \\
\vk_1 & \vk_1^\star & \vk_2 & \vk_2^\star & \vk_1 + \vk_2 & \vk_1^\star + \vk_2^\star \\ 
\vk_1^\star & \vk_1^{\star\star}  & \vk_2^\star & \vk_2^{\star\star}  & \vk_1^\star + \vk_2^\star & \vk_1^{\star\star} + \vk_2^{\star\star}\\
\end{array} \right]
}
\end{align}
where $\vk_1 = k_1( \vX, \vX )$ and $\vk_1^\star = k_1( \vX^\star, \vX )$. 

By the formula for Gaussian conditionals:
\begin{align}
\vx_A | \vx_B \sim \Nt{\vmu_A + \vSigma_{AB} \vSigma_{BB}\inv \left( \vx_B - \vmu_B \right) }
{\vSigma_{AA} - \vSigma_{AB} \vSigma_{BB}\inv \vSigma_{BA} },
\end{align}
we get that the conditional variance of a Gaussian conditioned on its sum with another Gaussian is given by
\begin{align}
%\vf_1^\star | \vf \sim \Nt{\vmu_1^\star + \vk_1^{\star\tra} (\vK_1 + \vK_2)\inv \left( \vf - \vmu_1 - \vmu_2 \right) }
\vf_1(\vx^\star) | \vf(\vx) \sim \mathcal{N}\big( 
& \vmu_1(\vx^\star) + \vk_1(\vx^{\star}, \vx)  \left[ \vK_1(\vx, \vx) + \vK_2(\vx, \vx) \right]\inv \left( \vf(\vx) - \vmu_1(\vx) - \vmu_2(\vx) \right) , \nonumber \\
& \vk_1(\vx^{\star}, \vx^{\star}) - \vk_1(\vx^{\star}, \vx)  \left[ \vK_1(\vx, \vx) + \vK_2(\vx, \vx) \right] \inv \vk_1(\vx, \vx^{\star}) 
\big).
\end{align}

%The covariance between the two components, conditioned on their sum is given by:
%\begin{align}
%\cov(\vf_1^\star, \vf_2^\star) | \vf = - \vk_1^{\star\tra} (\vK_1 + \vK_2)\inv \vk_2^\star
%\covargs{\vf_1(\vx^\star)}{\vf_2(\vx^\star) | \vf(\vx) }= - \vk_1(\vx^{\star}, \vx)  \left[ \vK_1(\vx, \vx) + \vK_2(\vx, \vx) \right] \inv \vk_1(\vx, \vx^{\star}) 
%\end{align}

These formulae express the posterior model uncertainty about different components of the signal, integrating over the possible configurations of the other components.





\outbpdocument{
\bibliographystyle{plainnat}
\bibliography{references.bib}
}


