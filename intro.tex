%
% A header that lets you compile a chapter by itself, or inside a larger document.
% Adapted from http://stackoverflow.com/questions/3655454/conditional-import-in-latex
%
%
%Use \inbpdocument and \outbpdocument in your individual files, in place of \begin{document} and \end{document}. In your main file, put in a \def \ismaindoc {} before including or importing anything.
%
% David Duvenaud
% June 2011
% 
% ======================================
%
%


\ifx\ismaindoc\undefined
	\newcommand{\inbpdocument}{
		\def \ismaindoc {}
		% Use this header if we are compiling by ourselves.
		\documentclass[a4paper,12pt,authoryear,index]{common/PhDThesisPSnPDF}
		\input{common/official-preamble.tex}
		% All my custom preamble stuff.  Shouldn't overlap with anything in official-preamble

\usepackage{nth}
\usepackage{rotating}
\usepackage{array}
%\usepackage{gantt}
\usepackage[hyperpageref]{backref}

		% ************************ Thesis Information & Meta-data **********************

%% The title of the thesis
%\title{Structured Gaussian Process Models} 
%\title{Automatic Model Construction \\ through \\ Structured Gaussian Processes}
%\title{Automatic Model-Building \\ through \\ Structured Gaussian Processes}
%\title{Automatic Modeling \\ with \\ Structured Gaussian Processes}    
\title{Automatic Model Construction \\ with Gaussian Processes}
%\title{Automatic Model Construction}
%\title{Automating Statistical Model Construction}


%\texorpdfstring is used for PDF metadata. Usage:
%\texorpdfstring{LaTeX_Version}{PDF Version (non-latex)} eg.,
%\texorpdfstring{$sigma$}{sigma}

%% The full name of the author
\author{David Kristjanson Duvenaud}

%% Department (eg. Department of Engineering, Maths, Physics)
\dept{Department of Engineering}

%% University and Crest
\university{University of Cambridge}
\crest{\includegraphics[width=0.25\textwidth]{University_Crest}}

%% You can redefine the submission text:
% Default as per the University guidelines: This dissertation is submitted for
% the degree of Doctor of Philosophy
%\renewcommand{\submissiontext}{change the default text here if needed}

%% Full title of the Degree 
\degree{Doctor of Philosophy}
 
%% College affiliation (optional)
\college{Pembroke College}

%% Submission date
\degreedate{June 2014} 

%% Meta information
\subject{LaTeX} \keywords{{LaTeX} {PhD Thesis} {Engineering} {University of
Cambridge}}



		\begin{document}
	}	
	\newcommand{\outbpdocument}[1]{
		%\bibliographystyle{common/CUEDthesis}
		\bibliographystyle{plainnat}
		\bibliography{references.bib}
		\end{document}
	}	
\else
	%If we're inside another document, no need to re-start the document.
	\ifx\inbpdocument\undefined
		\newcommand{\inbpdocument}{}
		\newcommand{\outbpdocument}[1]{}
	\fi
\fi

\inbpdocument

\chapter{Introduction}
\label{ch:intro}


\begin{quotation}
``I only work on intractable nonparametrics - Gaussian processes don't count.''

\hspace*{\fill} Sinead Williamson, personal communication
\end{quotation}


\section{Regression}

% These two paras written with James L and Zoubin
The general problem of regression consists of learning a function $f$ mapping from some input space $\mathcal{X}$ to some output space $\mathcal{Y}$.
We would like an expressive language which can represent both simple parametric forms of $f$ such as linear, polynomial, \etc and also complex nonparametric functions specified in terms of properties such as smoothness, periodicity, \etc~ 
Fortunately, Gaussian processes (\gp{}s) provide a very general and analytically tractable way of capturing both simple and complex functions. 

\section{Gaussian process models}


Gaussian processes are a flexible and tractable prior over functions, useful for solving regression and classification tasks\cite{rasmussen38gaussian}.  The kind of structure which can be captured by a GP model is mainly determined by its \emph{kernel}: the covariance function.  One of the main difficulties in specifying a Gaussian process model is in choosing a kernel which can represent the structure present in the data.  For small to medium-sized datasets, the kernel has a large impact on modeling efficacy.

Gaussian processes are distributions over functions such that any finite subset of function evaluations, $(f(x_1), f(x_2), \ldots f(x_N))$, have a joint Gaussian distribution \citep{rasmussen38gaussian}.
A \gp{} is completely specified by its mean function, $\mu(x)=\mathbb{E}(f(x))$ and kernel (or covariance) function $\kernel(x,x') = \Cov(f(x),f(x'))$.
It is common practice to assume zero mean, since marginalizing over an unknown mean function can be equivalently expressed as a zero-mean \gp{} with a new kernel.
The structure of the kernel captures high-level properties of the unknown function, $f$, which in turn determines how the model generalizes or extrapolates to new data.
We can therefore define a language of regression models by specifying a language of kernels.

\subsection{Useful properties of Gaussian process models}

\begin{itemize}
\item {\bf Tractable inference} Given a kernel function, the posterior distribution can be computed exactly in closed form.  This is a rare property for nonparametric models to have.
\item {\bf Expressivity} by choosing different covariance functions, we can express a very wide range of modeling assumptions.
\item {\bf Integration over hypotheses} the fact that a \gp{} posterior lets us exactly integrate over a wide range of hypotheses means that overfitting is less of an issue than in comparable model classes - for example, neural nets.
\item {\bf Marginal likelihood} A side benefit of being able to integrate over all hyoptheses is that we compute the \emph{marginal likelihood} of the data given the model.  This gives us a principled way of comparing different Gaussian process models.
\item {\bf Closed-form posterior} The posterior predictive distribution of a \gp{} is another \gp{}.  This means that \gp{}s can easily be composed with other models or decision procedures.  For example, \NA{Carl's reinforcement learning work.}
\end{itemize}


\begin{figure}[t]
\begin{centering}
\begin{tabular}{cc}
\includegraphics[width=0.48\textwidth]{\introfigsdir/fuzzy-0} & 
\includegraphics[width=0.48\textwidth]{\introfigsdir/fuzzy-1} \\
\includegraphics[width=0.48\textwidth]{\introfigsdir/fuzzy-2} & 
\includegraphics[width=0.48\textwidth]{\introfigsdir/fuzzy-3}
\end{tabular}
\end{centering}
\caption[One-dimensional Gaussian process posterior]{A visual representation of a one-dimensional Gaussian process posterior.
Red isocountours show the marginal density at each input location.
Coloured lines are samples from the posterior.}
\label{fig:gp-post}
\end{figure}

Figure \ref{fig:gp-post} shows a Gaussian process posterior.  Typically, it's rendered with the mean and +- 2SD, but there's nothing special about mean.


\subsection{Gaussian process models in practice}

The class of models that could be called Gaussian processes is extremely broad.
Examples of commonly used models not usually cast as \gp{}s are linear regression, splines, some forms of generalized additive models, and Kalman filters.

Applying \gp{} regression as a non-expert is a daunting task.
The first problem one encounters when attempting to apply \gp{}s to a particular task is that the choice of kernel is very important, but it's usually not clear which covariance function is appropriate.

In some instances, using a 'default' kernel yields acceptable performance.
Many frequentist methods assume a characteristic kernel, such as the squared-exp.
This choice is motivated by the fact that, in the limit of infinite data, and a shrinking lengthscale, the estimate of the function will converge asymptotically to the truth. [citation needed]

As we shall see in this thesis, in the small-to-medium data range, the choice of of kernel is extremely important.
This is especially true when one wishes to do extrapolation.


\paragraph{Discovering interpretable structure}
Besides allowing faster learning and extrapolation, learning a more structured kernel sometimes has the added benefit of making the resulting model more intepretable.
This is a similar motivation as for the use of sparsity-inducing methods:  on many real datasets, the signal can be well-predicted by some small subset of the inputs.  Identifying this subset allows both better generalization, and a more interpretable model.




\subsection{Why assume zero mean?}

It is common practice to assume zero mean, since marginalizing over an unknown mean function can be equivalently expressed as a different \gp{} with zero-mean, and another term added to the kernel.
Specifically, if we wish to model an unknown function $f(\vx)$ with known mean $m(\vx)$, (with unknown magnitude $c \sim \Nt{0}{\sigma^2_c}$), we can equivalently express this model using another \gp{} with zero mean:
%$related modelquivalently (\vx) a(\vx)$ with $f \sim \GPt{0}{k(\vx,\vx')}$, then this 
%
\begin{align}
f \sim \GPt{c m(\vx)}{k(\vx,\vx')}, \quad c \sim \Nt{0}{\sigma^2_c}
\iff f \sim \GPt{ \vzero }{ c^2 m(\vx) m(\vx') + k(\vx,\vx')}
\end{align}

%This correspondence means that, b
By moving the mean function into the covariance function, we get the same model, but we can integrate over the magnitude of the mean function at no additional cost.
This is one advantage of moving as much structure as possible into the covariance function.
In fact, we can view \gp{} regression as simply implicitly integrating over the magnitudes of (possibly uncountably many) different mean functions all summed together.
%TODO: provide a link to the GPs as neural nets discussion.

\subsection{Why not simply learn the mean function?}
One might ask: besides integrating over the magnitude, what is the advantage of moving the mean function into the covariance function?
After all, mean functions are certainly more interpretable than a posterior distribution over functions.

Instead of searching over a large class of covariance functions, which seems strange and unnatural, we might consider simply searching over a large class of structured mean functions, assuming a simple \iid noise model.
This is the approach taken by practically every other regression technique: neural networks, decision trees, boosting, \etc.
If we could integrate over a wide class of possible mean functions, we would have a very powerful learning an inference method.
The problem faced by all of these methods is the well-known problem \emph{overfitting}.
If we are forced to choose just a single function with which to make predictions, we must carefully control the flexibility of the model we learn, generally prefering ``simple'' functions, or to choose a function from a restricted set.

If, on the other hand, we are allowed to keep in mind many possible explanations of the data, \emph{there is no need to penalize complexity}. [cite Occam's razor paper?]
The power of putting structure into the covariance function is that doing so allows us to implictly integrate over many functions, maintaining a posterior distribution over infinitely many functions, instead of choosing just one.
In fact, each of the functions being considered can be infinitely complex, without causing any form of overfitting.
For example, each of the samples shown in figure \ref{fig:gp-post} varies randomly over the whole real line, never repeating, each one requiring an infinite amount of information to describe.
Choosing the one function which best fits the data will almost certainly cause overfitting.
However, if we integrate over many such functions, we will end up with a posterior putting mass on only those functions which are compatible with the data.
In other words, the parts of the function that we can determine from the data will be predicted with certainty, but the parts which are still uncertain will give rise to a wide range of predictions.

To repeat: \emph{there is no need to assume that the function being modeled is simple, or to prefer simple explanations} in order to avoid overfitting, if we integrate over many possible explanations rather than choosing just the one.



\section{Latent Variable Models}

Besides being useful for modeling functions, a simple extension allows \gp{}s to be useful for general density modeling.  



Unfortunately, this extension causes many of the useful properties of the \gp{} not to hold.

The \gplvm{} can also be thought of as a method for modeling the covariance matrix between all rows of $Y$ using a number of parameters which grows linearly with $N$.


\begin{figure}[t]
\begin{centering}
\includegraphics[width=\textwidth]{\gplvmfiguresdir/gplvm_1d_draw_9}
\end{centering}
\caption[One-dimensional Gaussian process latent variable model]{A visual representation of the Gaussian process latent variable model.  Bottom: density and samples from a 1D Gaussian, specifying the distribution $p(\vX)$ in the latent space.  Top Right: A function drawn from a \gp{} prior.  Left: A nonparametric density defined by warping the latent density through the function drawn from a \gp{} prior.}  
\label{fig:oned-gplvm}
\end{figure}



\begin{figure}
\begin{centering}
{\begin{tabular}{cccc}
\phantom{h} & Latent space $p(\vX)$ & & Observed space $p(\vY)$ \\
& \fbox{\includegraphics[height=15em,width=15em]{\gplvmfiguresdir/gplvm-latent-crop}} &
\raisebox{7em}{$\overset{\mathlarger{f(x)}}{\mathlarger{\mathlarger{\mathlarger{\mathlarger{\mathlarger{\mathlarger{\mathlarger{\rightarrow}}}}}}}}$} &
%\raisebox{7em}{\Huge $\overset{\textnormal{\Large{GP}}}\rightarrow$} &
\fbox{\includegraphics[height=15em,width=15em]{\gplvmfiguresdir/gplvm-draw-crop}}
\end{tabular}}
\end{centering}
\caption[Two-dimensional Gaussian process latent variable model]{A visual representation of the Gaussian process latent variable model.  Left:  Isocontours and samples from a 2D Gaussian, specifying the distribution $p(\vX)$ in the latent space.  Right: Density and samples from a nonparametric density defined by warping the latent density through a function drawn from a GP prior.}  
\label{fig:twod-gplvm}
\end{figure}



\section{Outline}

This thesis aims to present a set of related results about how the probabilistic nature of Gaussian process models allows them to be easily extended or composed with other models.
Furthermore, the fact that the marginal likelihood is often available (or easily approximable) means that we can evaluate how much evidence the data provides for one structure over another.


\paragraph{Chapter \ref{ch:kernels}} describes, in detail, the ways in which different sorts of structure can be introduced into a \gp{} model through the kernel.

\paragraph{Chapter \ref{ch:grammar}} shows how to construct a general, open-ended language over kernels - which implies a corresponding language over models.
Given a wide variety of structures, plus the ability to evaluate the suitability of each one, it's straightforward to automatically search over models.

\paragraph{Chapter \ref{ch:description}} shows that, for the particular language of models constructed in chapter \ref{ch:grammar}, it's relatively easy to automatically generate english-language descriptions of the models discovered.
Combined with 


\section{Contributions}

My doctorate was made possible, and enjoyable by the company of the many co-authors I was fortunate to work with.
In this section, I detail the novel contributions of this thesis, and attempt to give proper credit to my tireless co-authors.

\paragraph{Grammar on topologies}
Chapter \ref{ch:kernels} contains a section on describes, in detail, the ways in which different sorts of structure can be introduced into a \gp{} model through the kernel.

\paragraph{Deep kernels}
Section \ref{sec:deep_kernels}

The research upon which \paragraph{chapter \ref{ch:grammar}} is based was done in collaboration with James Robert Lloyd, Roger Grosse, Joshua B. Tenenbaum, and Zoubin Ghahramani.
Specifically, myself, James Lloyd and Roger Grosse wrote the

\paragraph{Chapter \ref{ch:description}} was written in collaboration with James Robert Lloyd, Roger Grosse, Joshua B. Tenenbaum, Zoubin Ghahramani.
Specifically, James Lloyd wrote most of the code to automatically generate reports, and ran all of the experiments.
The idea of the correspondence between kernels and adjectives grew out of many extended discussions between myself and James.
The text was written mainly by myself and James Lloyd, with many helpful contributions and suggestions from Roger Grosse, Zoubin Ghaharamani, and Josh Tenenbaum.


\outbpdocument{
\bibliographystyle{plainnat}
\bibliography{references.bib}
}


