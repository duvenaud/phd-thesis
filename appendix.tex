%
% A header that lets you compile a chapter by itself, or inside a larger document.
% Adapted from http://stackoverflow.com/questions/3655454/conditional-import-in-latex
%
%
%Use \inbpdocument and \outbpdocument in your individual files, in place of \begin{document} and \end{document}. In your main file, put in a \def \ismaindoc {} before including or importing anything.
%
% David Duvenaud
% June 2011
% 
% ======================================
%
%


\ifx\ismaindoc\undefined
	\newcommand{\inbpdocument}{
		\def \ismaindoc {}
		% Use this header if we are compiling by ourselves.
		\documentclass[a4paper,12pt,authoryear,index]{common/PhDThesisPSnPDF}
		\input{common/official-preamble.tex}
		% All my custom preamble stuff.  Shouldn't overlap with anything in official-preamble

\usepackage{nth}
\usepackage{rotating}
\usepackage{array}
%\usepackage{gantt}
\usepackage[hyperpageref]{backref}

		% ************************ Thesis Information & Meta-data **********************

%% The title of the thesis
%\title{Structured Gaussian Process Models} 
%\title{Automatic Model Construction \\ through \\ Structured Gaussian Processes}
%\title{Automatic Model-Building \\ through \\ Structured Gaussian Processes}
%\title{Automatic Modeling \\ with \\ Structured Gaussian Processes}    
\title{Automatic Model Construction \\ with Gaussian Processes}
%\title{Automatic Model Construction}
%\title{Automating Statistical Model Construction}


%\texorpdfstring is used for PDF metadata. Usage:
%\texorpdfstring{LaTeX_Version}{PDF Version (non-latex)} eg.,
%\texorpdfstring{$sigma$}{sigma}

%% The full name of the author
\author{David Kristjanson Duvenaud}

%% Department (eg. Department of Engineering, Maths, Physics)
\dept{Department of Engineering}

%% University and Crest
\university{University of Cambridge}
\crest{\includegraphics[width=0.25\textwidth]{University_Crest}}

%% You can redefine the submission text:
% Default as per the University guidelines: This dissertation is submitted for
% the degree of Doctor of Philosophy
%\renewcommand{\submissiontext}{change the default text here if needed}

%% Full title of the Degree 
\degree{Doctor of Philosophy}
 
%% College affiliation (optional)
\college{Pembroke College}

%% Submission date
\degreedate{June 2014} 

%% Meta information
\subject{LaTeX} \keywords{{LaTeX} {PhD Thesis} {Engineering} {University of
Cambridge}}



		\begin{document}
	}	
	\newcommand{\outbpdocument}[1]{
		%\bibliographystyle{common/CUEDthesis}
		\bibliographystyle{plainnat}
		\bibliography{references.bib}
		\end{document}
	}	
\else
	%If we're inside another document, no need to re-start the document.
	\ifx\inbpdocument\undefined
		\newcommand{\inbpdocument}{}
		\newcommand{\outbpdocument}[1]{}
	\fi
\fi

\inbpdocument

\chapter{Gaussian Conditionals}
\label{ch:appendix-gaussians}

%\section{Formula for Gaussian Conditionals}

A standard result shows how to condition on knowing a subset of the dimensions $\vy_B$ of a vector $\vy$ having a multivariate Gaussian distribution.
If
%
\begin{align}
\vy = \colvec{\vy_A}{\vy_B} \sim \Nt{\colvec{\vmu_A}{\vmu_B}}{\left[ \begin{array}{cc}\vSigma_{AA} & \vSigma_{AB} \\ \vSigma_{BA} & \vSigma_{BB} \end{array} \right]}
%\left[ \begin{array} \vx_A \sim \Nt{\vmu_A}{
%| \vx_B \sim \Nt{\vmu_A + \vSigma_{AB} \vSigma_{BB}\inv \left( \vx_B - \vmu_B \right) }
%{\vSigma_{AA} - \vSigma_{AB} \vSigma_{BB}\inv \vSigma_{BA} }
%\label{eq:gauss_conditional}
\end{align}
%
then
%
\begin{align}
\vy_A | \vy_B \sim \Nt{\vmu_A + \vSigma_{AB} \vSigma_{BB}\inv \left( \vx_B - \vmu_B \right) }
{\vSigma_{AA} - \vSigma_{AB} \vSigma_{BB}\inv \vSigma_{BA} }.
\label{eq:gauss_conditional}
\end{align}

This result can be used in the context of Gaussian process regression, where $\vy_B = [f(\vx_1), f(\vx_2), \dots, f(\vx_N)]$ represents a set of function values observed at some subset of locations $[\vx_1, \vx_2, \dots, \vx_N]$, while $\vy_A = [f(\vx_1\star), f(\vx_2\star), \ldots, f(\vx_N\star)]$ represents test points whose predictive distribution we'd like to know.
In this case, the necessary covariance matrices are given by:
%
\begin{align}
\vSigma_{AA} & = k(\vX, \vX) \\
\vSigma_{AB} & = k(\vX, \vX^\star) \\
\vSigma_{BA} & = k(\vX^\star, \vX) \\
\vSigma_{BB} & = k(\vX^\star, \vX^\star)
\end{align}
and similarily for the mean vectors.






\chapter{Kernel Definitions}
\label{ch:appendix-kernels}
\label{sec:kernel-definitions}

%\newcommand{\scalefactor}{\sigma_f^2}
\newcommand{\scalefactor}{}

%\subsection{Base kernels}

%For scalar-valued inputs, the white noise ($\kWN$), constant ($\kC$), linear ($\kLin$), squared exponential ($\kSE$), and periodic kernels ($\kPer$) are defined as follows:
Here we give the formulas for all one-dimensional base kernels mentioned in the thesis.
Each of these formulas is multiplied by a scale factor $\sigma_f^2$, which we omit for clarity.
\begin{align}
\kC(\inputVar, \inputVar') & = \scalefactor 1 \\
\kWN(\inputVar, \inputVar') & = \scalefactor \delta(\inputVar - \inputVar') \label{eq:appendix-wn}\\
\kLin(\inputVar, \inputVar') & = \scalefactor (\inputVar - c)(\inputVar' - c)  \label{eq:appendix-lin} \\
\kSE(\inputVar, \inputVar') & = \scalefactor \exp\left(-\frac{(\inputVar - \inputVar')^2}{2\ell^2}\right)  \label{eq:appendix-se}\\
\kRQ(x, x') & = \scalefactor \left( 1 + \frac{(\inputVar - \inputVar')^2}{2\alpha\ell^2}\right)^{-\alpha}  \label{eq:appendix-rq}\\
\kPer(\inputVar, \inputVar') & =  \sigma_f^2 \frac{\exp\left(\frac{1}{\ell^2}\cos 2 \pi  \frac{(\inputVar - \inputVar')}{p}\right) - I_0\left(\frac{1}{\ell^2}\right)}{\exp\left(\frac{1}{\ell^2}\right) - I_0\left(\frac{1}{\ell^2}\right)} \label{eq:generalized-periodic} \\
%\kPer(\inputVar, \inputVar') & =  \scalefactor \exp\left(\frac{1}{\ell^2}\cos 2 \pi  \frac{(\inputVar - \inputVar')}{p}\right) - I_0\left(\frac{1}{\ell^2}\right) \\
\cos(x, x') & = \scalefactor \cos\left(\frac{2 \pi (x - x')}{p}\right) \label{eq:appendix-cos} \\
\kCP(\kernel_1, \kernel_2)(x, x') & = \scalefactor \sigma(x) k_1(x,x')\sigma(x') + (1-\sigma(x)) k_2(x,x')(1-\sigma(x')) \\
\boldsymbol\sigma & = \scalefactor \sigma(x)\sigma(x') \\
\boldsymbol{\bar\sigma} & = \scalefactor (1-\sigma(x))(1-\sigma(x'))
\end{align}
where $\delta_{\inputVar, \inputVar'}$ is the Kronecker delta function, $I_0$ is the modified Bessel function of the first kind of order zero, and other symbols are kernel parameters.
\Cref{eq:appendix-se,eq:generalized-periodic,eq:appendix-lin} are plotted in \cref{fig:basic_kernels}, and \cref{eq:appendix-wn,eq:appendix-rq,eq:appendix-cos} are plotted in \cref{fig:basic_kernels_two}.
Draws from \gp{} priors with changepoint kernels are shown in \cref{fig:changepoint_examples}.



\subsubsection{The Generalized Periodic Kernel}

\citet{lloyd-periodic} showed that the standard periodic kernel due to \citet{mackay1998introduction} can be decomposed into a periodic and a constant component.
He derived the equivalent periodic kernel without any constant component, shown in \cref{eq:generalized-periodic}.
He further showed that its limit as the lengthscale grows is the cosine kernel:
\begin{equation}
\lim_{\ell \to \infty} \kPer(x, x') = \cos\left(\frac{2 \pi (x - x')}{p}\right).
\end{equation}

Separating out the constant component allows us to express negative prior covariance, as well as increasing the interpretability of the resulting models.






\chapter{Search Operators}
\label{ch:appendix-search}
\label{sec:search-operators}

%\subsection{Overview}

The model construction phase of \procedurename{} starts with the noise kernel, $\kWN$.
New kernel expressions are generated by applying search operators to the current kernel, which replace some part of the existing kernel expression with a new kernel expression.
%When new base kernels are proposed by the search operators, their parameters are randomly initialised with several restarts.
%Parameters are then optimized by conjugate gradients to maximise the likelihood of the data conditioned on the kernel parameters.
%The kernels are then scored by the Bayesian information criterion and the top scoring kernel is selected as the new kernel.
%The search then proceeds by applying the search operators to the new kernel \ie this is a greedy search algorithm.

%In all experiments, 10 random restarts were used for parameter initialisation and the search was run to a depth of 10.

%\subsection{Search operators}

The search used in the multidimensional regression experiments in \cref{sec:synthetic,sec:additive-experiments} used only the following search operators:
%
\begin{eqnarray}
\mathcal{S} &\to& \mathcal{S} + \mathcal{B} \\
\mathcal{S} &\to& \mathcal{S} \times \mathcal{B} \label{eq:search-multiply}\\
\mathcal{B} &\to& \mathcal{B'}
\end{eqnarray}
%
where $\mathcal{S}$ represents any kernel subexpression and $\mathcal{B}$ is any base kernel within a kernel expression.
These search operators represent addition, multiplication and replacement.
When the multiplication operator is applied to a subexpression which includes a sum of subexpressions, parentheses () are introduced.
For instance, if rule \eqref{eq:search-multiply} is applied to the subexpression $k_1 + k_2$, the resulting expression is $(k_1 \kernplus k_2) \kerntimes \mathcal{B}$.

Afterwards, we added several more search operators in order to speed up the search.
These new operators do not change the set of possible models.

To accommodate changepoints and changewindows, we introduced the following additional operators to our search:
%
\begin{eqnarray}
\mathcal{S} &\to& \kCP(\mathcal{S},\mathcal{S}) \\
\mathcal{S} &\to& \kCW(\mathcal{S},\mathcal{S}) \\
\mathcal{S} &\to& \kCW(\mathcal{S},\kC) \\
\mathcal{S} &\to& \kCW(\kC,\mathcal{S})
\end{eqnarray}
%
where $\kC$ is the constant kernel.
The last two operators result in a kernel only applying outside, or within, a certain region.

To allow the search to simplify existing expressions, we introduced the following operators:
%
\begin{eqnarray}
\mathcal{S} &\to& \mathcal{B}\\
\mathcal{S} + \mathcal{S'} &\to& \mathcal{S}\\
\mathcal{S} \times \mathcal{S'} &\to& \mathcal{S}
\end{eqnarray}
%
where $\mathcal{S'}$ represents any other kernel expression.
%Their introduction is currently not rigorously justified.
We also introduced the operator
%
\begin{eqnarray}
\mathcal{S} &\to& \mathcal{S} \times (\mathcal{B} + \kC)
\end{eqnarray}
%
Which allows a new base kernel to be added along with the constant kernel, for cases when multiplying by a base kernel by itself would restrict the model too much.





\chapter{Example Automatically-Generated Report}
\label{ch:example-solar}

The following pages of this appendix contain an entire automatically-generated report, run on a dataset measuring annual solar irradiation data from 1610 to 2011.
This dataset was previously analyzed by \citet{lean1995reconstruction}.

The structure search was run using the \procedurename-interpretable variant, with base kernels $\kSE$, $\kLin$, $\kC$, $\kPer$, $\vsigma$, and $\kWN$.

Other example reports can be found at \url{mlg.eng.cam.ac.uk/Lloyd/abcdoutput/}, including anlyses of wheat prices, temperature records, call centre volumes, radio interference, gas production, unemployment, number of births, and wages over time.

\newcommand{\solarreportpage}[1]{\includegraphics[width=\columnwidth]{figures/solarpages/02-solar-seperate-pages-#1}}

\clearpage

%\solarreportpage{1}

\solarreportpage{2}
\\ \vspace{1cm} \\
\solarreportpage{3}
\\ \vspace{1cm} \\
\solarreportpage{4}
\\ \vspace{1cm} \\
\solarreportpage{5}
\\ \vspace{1cm} \\
\solarreportpage{6}
\\ \vspace{1cm} \\
\solarreportpage{7}
\\ \vspace{1cm} \\
\solarreportpage{8}
\\ \vspace{1cm} \\
\solarreportpage{9}
\\ \vspace{1cm} \\
\solarreportpage{10}
\\ \vspace{1cm} \\
\solarreportpage{11}
\\ \vspace{1cm} \\
\solarreportpage{12}
\\ \vspace{1cm} \\
\solarreportpage{13}

%\solarreportpage{14}

%\solarreportpage{15}

%\solarreportpage{16}

%\solarreportpage{17}

%\solarreportpage{18}

%\solarreportpage{19}

%\solarreportpage{20}

%\solarreportpage{21}

%\solarreportpage{22}


\chapter{Inference in the Warped Mixture Model}
\label{ch:warped-appendix}

\subsubsection{Detailed Definition of Model}

The \iwmm{} assumes that the latent density is an infinite mixture of Gaussians:
\begin{align}
p(\vx ) = \sum_{c=1}^{\infty} \lambda_c \, {\cal N}(\vx|\bm{\mu}_{c},\vR_{c}\inv)
\end{align}
where $\lambda_{c}$, $\bm{\mu}_{c}$ and $\vR_{c}$ is the mixture weight, mean, and precision matrix of the $c^{\text{\tiny th}}$ mixture component.
%
To ensure a conjugate likelihood, we place Gaussian-Wishart priors on the Gaussian parameters
$\{\bm{\mu}_{c},\vR_{c}\}$,
\begin{align}
p(\bm{\mu}_{c},\vR_{c})
= {\cal N}(\bm{\mu}_{c}|\vu,(r\vR_{c})\inv)
{\cal W}(\vR_{c}|\vS\inv,\nu),
\end{align}
%
where $\vu$ is the mean of $\vmu_c$, $r$ is the relative precision of $\vmu_{c}$, $\vS\inv$ is the scale matrix for $\vR_c$, and $\nu$ is the number of degrees of freedom for $\vR_c$.

The Wishart distribution is defined as follows:

\begin{align}
{\cal W}(\vR|\vS\inv,\nu)=\frac{1}{G}|\vR|^{\frac{\nu-Q-1}{2}}\exp\left(-\frac{1}{2}{\rm tr}(\vS\vR)\right),
\end{align}

where $G$ is the normalizing constant.
Because we use conjugate Gaussian-Wishart priors for the parameters of the Gaussian mixture components, we can analytically integrate out those parameters, given the assignments of points to components.
Let $z_{n}$ be the latent assignment of the $n^{\text{\tiny th}}$ point.
The probability of latent coordinates $\vX$ given latent assignments $\vZ=(z_1, z_2, \dots, z_N)$ is obtained by integrating out the Gaussian parameters $\{\vmu_c,\vR_c\}$ as follows:
%
\begin{align}
p(\vX|\vZ,\vS,\nu,r) &= \prod_{c=1}^{\infty}
\pi^{-\frac{N_{c}Q}{2}}\frac{r^{Q/2}|\vS|^{\nu/2}}{r_{c}^{Q/2}|\vS_{c}|^{\nu_{c}/2}}
\times \prod_{q=1}^{Q}\frac{\Gamma(\frac{\nu_{c}+1-q}{2})}{\Gamma(\frac{\nu+1-q}{2})},
\label{eq:px_z}
\end{align}
%
where
$N_c$ is the number of data points assigned to the $c^{\text{\tiny th}}$ component,
$\Gamma(\cdot)$ is the Gamma function, and
%
\begin{align}
r_{c}=r+N_{c}, \hspace{2em}
\nu_{c}=\nu+N_{c}, 
\quad
%\nonumber
%\end{align}
%\begin{align}
\vu_{c}=\frac{r\vu+\sum_{n:z_{n}=c}\vx_{n}}{r+N_{c}}, 
\nonumber
\end{align}
%
\begin{align}
\vS_{c}=\vS+\sum_{n:z_{n}=c}\vx_{n}\vx_{n}\tra + r\vu\vu\tra
 - r_{c}\vu_{c}\vu_{c}\tra,
\end{align}
%
are the posterior Gaussian-Wishart parameters of the $c^{\text{\tiny th}}$ component.
We use a Dirichlet process with concentration parameter $\eta$ for infinite mixture modeling~\citep{maceachern1998estimating} in the latent space.
Then, the probability of $\vZ$ is given as follows:
%
\begin{align}
p(\vZ|\eta) = 
\frac{\eta^{C}\prod_{c=1}^{C}(N_{c}-1)!}
{\eta(\eta+1)\cdots(\eta+N-1)},
\label{eq:pz}
\end{align}
%
where $C$ is the number of components for which $N_{c}>0$.
The joint distribution is given by
%
\begin{align}
p(\vY,\vX,\vZ|\bm{\theta},\bm{S},\nu,\vu,r,\eta)
 = p(\vY|\vX,\bm{\theta})
 p(\vX|\vZ,\bm{S},\nu,\vu,r)p(\vZ|\eta),
\label{eq:joint}
\end{align}
%
where factors in the right hand side can be calculated by equations \eqref{eq:py_x}, \eqref{eq:px_z} and \eqref{eq:pz}, respectively.

\subsubsection{Generative Model}

In summary, the infinite warped mixture model generates observations $\vY$ according to the following generative process:
%
\begin{enumerate}
\item Draw mixture weights $\bm{\lambda} \sim \GEM(\eta)$
\item For each component $c=1, 2, \dots, \infty$
\begin{enumerate}
\item Draw precision $\vR_c \sim {\mathcal W}(\vS\inv, \nu)$
\item Draw mean $\vmu_c \sim {\mathcal N}(\vu,(r\vR_c)\inv)$
\end{enumerate}
\item For each observed dimension $d=1, 2, \dots, D$
\begin{enumerate}
\item Draw function $f_{d}(\vx) \sim {\rm GP}(m(\vx),k(\vx,\vx'))$
\end{enumerate}
\item For each observation $n=1, 2, \dots,N$
\begin{enumerate}
\item Draw latent assignment $z_n \sim {\rm Mult}(\bm{\lambda})$
\item Draw latent coordinates $\vx_n \sim {\mathcal N}(\vmu_{z_n},\vR_{z_n}\inv)$
\item For each observed dimension $d=1, 2, \dots, D$
\begin{enumerate}
\item Draw feature $y_{nd} \sim {\cal N}(f_d(\vx_n), \beta\inv)$
\end{enumerate}
\end{enumerate}
\end{enumerate}
%
Here, $\GEM(\eta)$ is the stick-breaking process \citep{sethuraman94} that generates mixture weights for a Dirichlet process with parameter $\eta$, %${\rm Mult}(\cdot)$ represents a multinomial distribution,
${\rm Mult}(\bm{\lambda})$ represents a multinomial distribution with parameter $\bm{\lambda}$,
$m(\vx)$ is the mean function of the Gaussian process, and $\vx,\vx'\in{\mathbb R}^{Q}$.



\subsubsection{Details of Inference}
\label{sec:iwmm-inference-details}

By placing conjugate Gaussian-Wishart priors on the parameters of the Gaussian mixture components, we can analytically integrate out those parameters given the assignments of points to clusters.
The only remaining variables to infer will be the latent points $\vX$, the cluster assignments $\vZ$, and the kernel parameters $\vtheta$.

We can then infer the posterior distribution of the latent coordinates $\vX$ and cluster assignments $\vZ$ using Markov chain Monte Carlo (\mcmc{}).
In particular, we'll alternate collapsed Gibbs sampling of $\vZ$ and Hamiltonian Monte Carlo sampling of $\vX$.
Given $\vX$,
%the proposed model is simply a standard infinite mixture model with conjugate priors. Therefore, 
we can efficiently sample $\vZ$ using collapsed Gibbs sampling, integrating out the mixture parameters.
Given $\vZ$, we can calculate the gradient of the un-normalized posterior distribution of $\vX$, integrating over warping functions.
This gradient allows us to sample $\vX$ using Hamiltonian Monte Carlo.

First, we explain collapsed Gibbs sampling for $\vZ$.
Given a sample of $\vX$, $p(\vZ | \vX, \vS, \nu, \vu, r, \eta)$ does not depend on $\vY$.
This lets us resample cluster assignments, integrating out the \iGMM{} likelihood in closed form.
Given the current state of all but one latent component $z_n$, a new value for $z_n$ is sampled with the following probability:
%
\begin{align}
p(z_{n}=c|\vX,\vZ_{\setminus n},\bm{S},\nu,\vu,r,\eta)
 & \propto\!
\left\{
\begin{array}{ll}
\!\!N_{c\setminus n}\cdot p(\vx_{n}|\vX_{c\setminus n},\bm{S},\nu,\vu,r) & \text{{\small existing components}}\\
\!\!\eta\cdot p(\vx_{n}|\bm{S},\nu,\vu,r) & \text{{\small a new component}}
\end{array}
\right.
\label{eq:gibbs}
\end{align}
%
where $\vX_{c}=\{\vx_{n}|z_{n}=c\}$ 
is the set of latent coordinates assigned to the $c^{\text{\tiny th}}$ component,
and $\setminus n$ represents the value or set
when excluding the $n^{\text{\tiny th}}$ data point.
We can analytically calculate $p(\vx_{n}|\vX_{c\setminus n},\bm{S},\nu,\vu,r)$
as follows:
%
\begin{align}
p(\vx_{n}|\vX_{c\setminus n},\bm{S},\nu,\vu,r)
 & =\pi^{-\frac{N_{c\setminus n}Q}{2}}\frac{r_{c\setminus n}^{Q/2}|\vS_{c\setminus n}|^{\nu_{c\setminus n}/2}}{r_{c\setminus n}'^{Q/2}|\vS_{c\setminus n}'|^{\nu_{c\setminus n}'/2}}\prod_{d=1}^{Q}\frac{\Gamma(\frac{\nu_{c\setminus n}'+1-d}{2})}{\Gamma(\frac{\nu_{c\setminus n}+1-d}{2})},
\end{align}
%
where $r_{c}'$, $\nu_{c}'$, $\vu_{c}'$ and $\vS_{c}'$ represent the posterior Gaussian-Wishart parameters of the $c^{\text{\tiny th}}$ component when the $n^{\text{\tiny th}}$ data point is assigned to the $c^{\text{\tiny th}}$ component.
We can efficiently calculate the determinant by using the rank one Cholesky update.
In the same way, we can analytically calculate the likelihood for a new component $p(\vx_{n}|\bm{S},\nu,\vu,r)$.

\subsubsection{Gradients for Hamiltonian Monte Carlo}

Hamiltonian Monte Carlo (\HMC{}) sampling of $\vX$ from posterior ${p(\vX|\vZ,\vY,\bm{\theta},\bm{S},\nu,\vu,r)}$, requires computing the gradient of the log of the unnormalized posterior
%
\begin{align}
\log p(\vY|\vX,\bm{\theta}) + \log p(\vX|\vZ,\bm{S},\nu,\vu,r)
\label{eq:warped-hmc1}
\end{align}
%
The first term of the gradient can be calculated by
%
\begin{align}
\pderiv{\log p(\vY | \vX,\bm{\theta})}{\vX} =  \pderiv{\log p(\vY | \vX,\bm{\theta})}{\vK} \pderiv{\vK}{\vX} = \left[ -\frac{1}{2}D\vK\inv+\frac{1}{2}\vK\inv \vY \vY^{T} \vK\inv \right]\left[\pderiv{\vK}{\vX}\right],
\label{eq:warped-hmc2}
\end{align}
%
where for an $\kSE + \kWN$ kernel with the same lengthscale $\ell$ on all dimensions,
%
\begin{align}
\pderiv{k(\vx_{n},\vx_{m})}{\vx_n}
 & = -\frac{\sigma^2_f}{\ell^2} \exp \left( - \frac{1}{2 \ell^2} (\vx_n - \vx_m)\tra (\vx_n - \vx_m) \right) (\vx_n - \vx_m).
 \label{eq:warped-hmc3}
\end{align}
%
%using the chain rule.
The second term can be calculated as follows:
\begin{align}
\frac{\partial \log p(\vX|\vZ,\bm{S},\nu,\vu,r)}{\partial \vx_{n}} 
= -\nu_{z_{n}}\bm{S}_{z_{n}}\inv(\vx_{n}-\vu_{z_{n}}).
\label{eq:warped-hmc4}
\end{align}
We also infer kernel parameters $\vtheta$ via \HMC{}, using the gradient of the log unnormalized posterior with respect to the kernel parameters, and an improper uniform prior.




\subsubsection{Posterior Predictive Density}
\label{sec:iwmm-predictive-density}

In the \gplvm{}, the predictive density of at test point $\vy_\star$ is usually computed by finding the point $\vx_\star$ which which is most likely to be mapped to $\vy_\star$, then using the density of $p(\vx_\star)$ and the Jacobian of the warping at that point to approximately compute the density at $\vy_\star$.
When inference is done by optimizing the location of the latent points, this estimation method only requires solving a single optimization for each $\vy_\star$.  

For our model, we use approximate integration to estimate $p(\vy_\star)$.
This is done for two reasons:
First, multiple latent points (possibly from different clusters) can map to the same observed point, meaning the standard method can underestimate $p(\vy_\star)$.
Second, because we do not optimize the latent coordinates but rather sample them, we would need to perform optimizations for each $p(\vy_\star)$ seperately for each sample.
Our method gives estimates for all $p(\vy_\star)$ at once, but may not be accurate in very high dimensions.

The posterior density in the observed space given the training data is
\begin{align}
p(\vy_\star | \vY)
& = \int \!\!\! \int p(\vy_\star,\vx_\star, \vX | \vY)d\vx_\star d\vX \nonumber\\
& = \int\!\!\! \int p(\vy_\star | \vx_\star, \vX, \vY)p(\vx_\star|\vX,\vY)p(\vX|\vY)d\vx_\star d\vX.
\label{eq:density}
\end{align}
We approximate $p(\vX | \vY)$ using the samples from the Gibbs and Hamiltonian Monte Carlo samplers.
We approximate $p(\vx_\star | \vX, \vY)$ by sampling points from the latent mixture and warping them, using the following procedure:
\begin{enumerate}
\item Draw latent assignments
$z_\star \sim {\rm Mult}(\frac{N_{1}}{N+\eta},\cdots,\frac{N_{C}}{N+\eta},\frac{\eta}{N+\eta})$
\item Draw precision matrix
$\vR_\star \sim {\cal W}(\vS\inv_{z_\star},\nu_{z_\star})$
\item Draw mean
$\vmu_\star \sim {\cal N}(\vu_{z_\star},(r_{z_\star}\vR_\star)\inv)$
\item Draw latent coordinates
$\vx_\star \sim {\cal N}(\vmu_\star,\vR_\star\inv)$
\end{enumerate}
%
When a new component $C+1$ is assigned to $z_\star$, the prior Gaussian-Wishart distribution is used for sampling in steps 2 and 3.
The first factor of (\ref{eq:density}) can be calculated by
%
\begin{align}
p(\vy_\star | \vx_\star, \vX, \vY)
& = {\cal N}(\vk_\star\tra \vK\inv \vY, k(\vx_\star,\vx_\star)-\vk_\star\tra \vK\inv \vk_\star),
\end{align}
%
where
%
$\vk_\star=( k(\vx_\star, \vx_1), \cdots, k(\vx_\star, \vx_N))\tra$.
%  
Each step of this sampling procedure draws from the exact conditional distribution, so the Monte Carlo estimate of the predictive density $p(\vy_\star | \vX, \vY)$ will converge to the true marginal distribution.
Since the observations $\vy_\star$ are conditionally normally distributed, each one adds a smooth contribution to the empirical Monte Carlo estimate of the posterior density, as opposed to a collection of point masses.

\subsubsection{Source Code}

A reference implementation of the above algorithms is available at \url{github.com/duvenaud/warped-mixtures}.

\outbpdocument{
\bibliographystyle{plainnat}
\bibliography{references.bib}
}

